\renewcommand\evenpagerightmark{{\scshape\small Nederlandse Samenvatting}}
\renewcommand\oddpageleftmark{{\scshape\small Dutch Summary}}

\hyphenation{}

\chapter[Nederlandse samenvatting]{Nederlandse samenvatting\\\makebox[2.82in]{--Dutch Summary--}}
\label{samenvatting}

	De upgrade van de Large Hadron Collider (LHC) naar de High Luminosity LHC (‘hoge-luminositeit LHC’, HL-LHC) werd begonnen in 2018 met de start van de Second Long Shutdown (‘tweede lange stillegging’, LS2). Het doel hierbij is de luminositeit van de deeltjesversneller te verhogen om zo het potentieel om nieuwe deeltjesfysica voorbij het Standaardmodel (SM) te ontdekken op te drijven. Vele uitbreidingen van het SM omvatten zulke zaken als Heavy Stable Charged Particles (‘zware stabiele geladen deeltjes’, HSCP’s) waarbij, binnen de context van het Compacte Muon Solenoïde-experiment (CMS), het muonsysteem een belangrijke rol zou kunnen spelen om deze deeltjes te identificeren. Een toename in de onmiddellijke luminositeit zal aanleiding geven tot een verhoging van de achtergrondruis evenals die van de stralingsniveaus waaraan de detectors zullen blootgesteld worden. Bovendien, hoewel de huidige system ontworpen waren om de LHC tien jaar te laten lopen volgens de ontwerpparameters, wordt er nu van ze verwacht om nog eens een verdere tien jaar te mee te gaan bij verhoogde luminositeit. Daarom is er een uitgebreid upgradeprogramma gelanceerd voor alle muonsubsystemen van CMS, met het oog op de HL-LHC. Een eerste probleem in deze upgrade is het feit dat de systemen gecertifieerd moeten worden voor de HL-LHC periode.\\
	Daarnaast moet het muonsysteem uitgebreid worden tot in de nieuw-geïnstalleerde onderdelen, inclusief twee nieuwe, verbeterde Resistive Plate Chamberstations (‘resistieve plaatkamers’, RPCs) - RE3/1 en RE4/1 genaamd – in de voorwaartse regio, m.a.w. dichtbij de beam (‘straal’). Het doel is om een muontrigger van de beste kwaliteit te verzekeren door achtergrondeffecten te minimaliseren en de redundantie van het systeem te verhogen (dit zal het volgen van muonen eveneens misschien ten goede komen). Een upgrade zal ook plaatsvinden op niveau van het Link-Boardsysteem dat de front-endelectronica (FEE) van de CMS RPCs verbindt met de triggerprocessors, om zo de robuustheid te verbeteren en beter gebruik te maken van de effectieve tijdsresolutie van de plate chambers, met een grootteorde van \SI{1}{ns}. De CMS RPC-upgrade omvat een uitgebreid en spannend onderzoeksprogramma. De samenwerking kwam tot oplossingen die zullen toegepast worden in het perspectief van HL-LHC. Hoewel de consolidatie van de huidige CMS RPC-infrastructuur en de certificatie van nieuwe technologieën die de redundantie van de muonsystemen zullen vervolledigen nog steeds lopende is, is de toekomst van het experiment betreffende de RPCs zeer duidelijk. Om dit punt te bereiken is mijn bijdrage tijdens de R\&D-fase van het RPC-upgradeprogramma cruciaal geweest op twee vlakken: (i) de consolidatie van de huidige detectors, en (ii) de keuze van de FEEs waarmee de verbeterde RPCs (of: iRPCs) zullen uitgerust worden. Bij elke stap speelde ik een belangrijke rol in het opzetten van de experimenten, maar eveneens in het vergaren en analyseren van de data.\\
	De certificatie van de resulterende technologieën, aangepast aan de noden van de CMS in de RE3/1- en RE4/1-regio’s van de zgn. ‘muoneindkappen’ (muon endcaps), en die van het huidige RPC-systeem, wordt momenteel uitgevoerd in de nieuwe Gamma Irradiation Facility (‘gammabestralingsfaciliteit’, GIF++) waar een stralingsbunker rond de nieuwe muonenstraal werd gebouwd. Deze faciliteit beschikt over een cesiumbron van \SI{14}{TBq}. Het doel van deze nieuwe faciliteit is het uitvoeren van verscheidene langetermijnstudies uit te voeren binnen het perspectief van de HL-LHC. Het prototype van de verbeterde RPCs zijn trapeziumvormige ‘double-gap’ kamers met High-Pressure Laminate (‘hogedruklaminaat’, HPL) elektrodes van \SI{1.4}{mm} dik, en twee even dikke gasruimtes. De verdunning van de deze diktes werd gedreven door de lagere versterking of ‘gain’ van zulke detectoren bij gelijkaardige elektrisch veldsterktes. De PCB die de data uitleest bestaat uit 96 longitudinale strips met een onderlinge afstand die varieert tussen de \SI{4}{mm} aan het smalle uiteinde en \SI{8}{mm} aan het brede. Dit ontwerp definieert de basislijn van wat er aan de CMS zal geïnstalleerd worden in de toekomst.\vspace*{5mm}
	
	Mijn bijdrage aan de CMS upgrade was het valideren van twee geselecteerde FEE technologieën: (i) een voorversterker met lage ruis die gebruik maakt van SiGe-technologie ontwikkeld aan het INFN Tor Vergata, en (ii) FEEs gebaseerd op de PETIROC ASIC, eveneens gebruik makend van SiGe-technologie ontwikkeld door de OMEGA samenwerking en gebruikt voor timing-gerelateerde toepassingen. Om de veroudering van de geïnstalleerde detectors in de moeilijke regio dicht tegen de beam tegen te gaan, zullen gevoeligere elektronica die werken bij lagere drempelwaarden (\SI{10}{fC} in plaats van de huidige \SI{140}{fC}) helpen bij de versterking van de detector en dus ook bij de verwante ladingsdepositie binnenin de plaatkamer.\\
	De INFN Tor Vergata voorversterker is tot tweemaal toe getest geweest en vergeleken met de huidige CMS FEEs. Bij een eerste experiment werd de voorversterker rechtstreeks verbonden met vier uitleesstrips van een reserve-RPC, en het uitgangssignaal gedigitaliseerd door middel van een discriminator. De voorversterker werd gebruikt met een lage drempelwaarde voor de lading van \SI{3}{fC}, een toonde een werkzame spanningsverschuiving van \SI{475}{V} ten opzichte van een RPC in gebruik met huidige CMS-elektronica. Om de vergelijking van INFN-technologie met de CMS FEB te verbeteren werd een FEB zonder voorversterker ontworpen om te verbinden en te werken met de voorversterker. Dit liet ons toe een directe vergelijking te maken tussen de nieuwe voorversterker met lage ruis en degene die momenteel door CMS gebruikt wordt. De elektronica werd wederom gemonteerd op een reserve-PRC om het voorgaande experiment te herhalen. De verschuiving in de geobserveerde werkzame spanning liep op tot \SI{410}{V} bij een drempelwaarde van \SI{5}{fC}, wat consistent was met het eerste resultaat.\\
	De FEEs ontworpen door OMEGA werden getest op een enkele RPC (een enkele ‘gap’) uit de RE4-productie. Om een vergelijking te doen werd dit onderdeel in een RPC-behuizing gemonteerd en werd de detector gebruikt in single-gap modus met de CMS FEB. Het resultaat toonde aan dat de HARDROC2 een reductie in de werkzame spanning van bijna \SI{500}{V} kan teweegbrengen in single-gap modus bij een ladingsgevoeligheid van \SI{121.4}{fC}, vergelijkbaar met de drempelwaarde van \SI{146}{fC} bij CMS FEEs en dit met een ruis lager dan \SIrate{0.1}. Deze technologie was al eens gecertificeerd geweest bij andere experimenten die gebruik maakten van detectors zoals scintillators en RPCs. Ten slotte had het eveneens het voordeel van een tweedimensionale uitlezing te hebben die de spatiale resolutie van de detector in de radiale richting sterk zou verbeteren. Het uitbuiten van de goede tijdsresolutie van de elektronica en het uitlezen van de strips van de iRPCS langs beide kanten zal resulteren in een spatiale resolutie van ongeveer \SI{2}{cm} langs de richting van de strips, wat de bijdrage van de RPC aan de muontracking enorm zal versterken ten opzichte van huidige systemen.\\
	Verschillende prototypes werden samengesteld en worden gebruikt om de CMS RPCROC FEEs te testen: zowel deze afgeleid van de PETIROC technologie alsook het INFN Tor Vergata alternatief aangepast aan de noden van de CMS RPC. Ter vergelijking zijn kosmische testen zonder bestraling uitgevoerd op de iRPC prototypes gebruik makende van de huidige CMS FEBs. De detector bereikte zijn werkzame spanning op \SIerror{7383}{70}{C} met een efficiëntie in de buurt van 97\% bij gebruik aan een drempelwaarde van \SI{133}{fC}. De gemiddelde muonclustergrootte werd gemeten op \numerror{2.4}{0.1} strips en de ruis was van de orde \Ord{-1}\,\sirate.\\
	Twee versies (FEBv0 en FEBv1) van CMS RPCROC gebaseerd op de PETIROC ASIC gekoppeld aan een FPGA zijn tot nog toe getest geweest, met FEBv2 in voorbereiding. De CMS RPCROC is efficiënt ontworpen om de 96 strips langs beide uiteinden te lezen, wat tweedimensionale informatie geeft over de ontvangen signalen. FEBv0 werd gebruikt aan een drempelwaarde van \SI{100}{fC} terwijl FEBv1 werd gebruikt aan een lagere waarde van \SI{50}{fC} dankzij een PETIROC met verminderde overspraak (‘cross-talk’). Zonder bestraling bereikte het prototype voorzien van FEBv1 een efficiëntie van 99\% bij een werkzame spanning van \SI{7250}{V}. Met een achtergrondruis van \SIkrate{2} liep de werkzame spanning op tot \SI{7340}{V} met een efficiëntie van 95\%. Echter, zowel de Cyclone V FPGA en PETIROC2B ASIC hadden last van problemen in verband met stralingshardheid. De nieuwe FEBv2 zal een nieuwe versie van de Cyclone V FPGA bevatten, speciaal ontworpen om toepassingen in bestraalde omgevingen tegemoet te komen. De nieuwe PETIROC2C wordt momenteel getest in de Louvain-la-Neuve (LLN) neutronenbeam met een fluentie van \Ord{14}\,\si{pC/cm^2}, wat vijfmaal de verwachte waarde bij CMS zou moeten zijn.\\
	De INFN FEEs bevatten acht SiGe voorversterkers, alsook twee discriminator-ASICs. Aldus werden twaalf van deze FEEs geïntegreerd en gesoldeerd op de uitleesprintplaat. De detector werd gebruikt aan een drempelwaarde van \SI{5}{fC} en werd bestraald met een achtergrondruis tot \SIkrate{4}, waarbij het een efficiëntie aanhield van meer dan 92\%. Zonder bestraling had het prototype een efficiënte die hoger lag dan 99\% aan een werkzame spanning onder de \SI{7000}{V}. De gemiddelde muonclustergrootte werd gemeten op \numerror{3.4}{0.1}, groter dan in het geval van de CMS FEB ten gevolge van de verhoogde gevoeligheid. Met een ruis van \SIkrate{2} lag de efficiëntie op 97\% en de werkzame spanning op \SI{7250}{V}. Ten gevolge van straling zakte de muonclustergrootte tot \numerror{2.6}{0.1}.\vspace*{5mm}
	
	Mijn grootste bijdrage aan de CMS RPC-upgrade had te maken met de langdurigheid en de certificatie van het huidige systeem. Volgens extrapolaties van actuele CMS-data zouden de detectors van het huidige systeem gecertificeerd moeten worden voor een achtergrond van \SIrate{600} en zonder noemenswaardig prestatieverlies aan een totale geïntegreerde lading van \SI{840}{mC/cm^2}. Wat overeenstemt met driemaal de ergste achtergrondruis zoals geëxtrapoleerd uit CMS-data uit het RPC-systeem. Ik sloot mij aan bij de voorafgaande studie die plaatsvond bij de oude GIF van het CERN, waarbij een cesiumbron van 494 GBq beschikbaar was. Een eerste opstelling met een enkel reserve-exemplaar van een CMS RPC vormde een gelegenheid om eerste versies van datacollectie-, datakwaliteitmonitorings- en data-analysetools te ontwikkelen. De plaatkamer werd geïnstalleerd in een bunker voor de radioactieve bron om vervolgens bestraald te worden met een achtergrondruis van \SIrate{600}. De performantie van de detector werd getest met een scintillator-gebaseerde kosmische muontelescoop als trigger. De resultaten suggereerden dat de performantie van de detector gedaald was tot 80\% efficiëntie met een werkzame spanning verhoogd tot \SI{1000}{V} bij \SIrate{600}.\\
	Een grootschaliger experiment werd vervolgens ontworpen aan de GIF++. Deze nieuwe opstelling maakte gebruikt van twee reserve-RPCs van het RE2/2 type, geproduceerd in 2007 en twee van het RE4/2 type uit 2013. Eén plaatkamer van elk type diende als referentie terwijl de tweede diende voor studies naar langdurigheid onder straling. Op moment van schrijven is respectievelijk 74 en 40\% van het geplande programma uitgevoerd sinds 2016. De opvolging van de stroomdichtheid in de plaatkamer en van de achtergrondstraling vertoont een sterke correlatie met de temperatuur aan de GIF++. Wanneer dit in rekening wordt gebracht, tonen de stroomdichtheid en de achtergrondruis gemeten in de detector een daling ten opzichte van de referentiedetectors. Dit effect lijkt gecorreleerd te zijn met een toename aan resistiviteit in de bestraalde detectors met een factor 2 ten opzichte van de referentiedetectors. Echter, de schommelingen in de stroomdichtheid en de achtergrondruis tonen een correlatie met de relatieve luchtvochtigheid in het gasmengsel. Het effect lijkt dus reversibel door de plaatkamers aan een hogere relatieve luchtvochtigheid te gebruiken.\\
	Geen verlies aan performantie was merkbaar bij een achtergrondruis van \SIrate{600}. Alle detectors geven blijk van een efficiëntie boven 94\% tot op de vereiste ruisniveaus. De opgemeten verschuiving in werkzame spanning van 100 V voor de RE2 RPCs en van 300 tot \SI{500}{V} voor de RE4 RPCs is consistent met het verschil in de elektroderesistiviteit van beide detectortypes. De RE2 detectors hebben een resistiviteit in het interval tussen 1 en \Sci{3}{10}\,\si{\ohm\cdot cm} terwijl die van de nieuwere RE4 ligt tussen 0.7 en \Sci{2}{11}\,\si{\ohm\cdot cm}. De gammastraling veroorzaakt een spanningsdaling over de elektrodes evenredig met hun weerstand. Aangezien de elektrodes zich gedragen als condensators, zorgt deze daling in spanning (gewoonlijk verwaarloosbaar bij afwezigheid van straling) ervoor dat het effectief elektrisch veld in het gasvolume bij een gegeven aangelegde spanning daalt, wanneer het aantal inkomende gammadeeltjes toeneemt. Wanneer we hier rekening mee houden verdwijnt deze verschuiving in de werkzame spanning. De overblijvende gemonitorde parameters zoals efficiëntie, de gemiddeld muonclustergrootte of de ladingsdepositie per ‘avalanche’, tonen geen enkel verschil tussen referentie- en bestraalde detector.\\
	Op dit moment is de CMS RPC groep nog steeds bezig om het huidige RPC-systeem te certifiëren voor de toekomstige HL-LHC periode. Deze langdurigheidsstudie zou afgerond moeten zijn eind 2021. Met de upgrade van het Linksysteem zouden de huidige detectors de hoge-luminositeitsfase van de LHC moet kunnen doorstaan zonder grote veranderingen in hun performantie.

\clearpage{\pagestyle{empty}\cleardoublepage}


