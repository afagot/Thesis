% Header
\renewcommand\evenpagerightmark{{\scshape\small Chapter 6}}
\renewcommand\oddpageleftmark{{\scshape\small Improved RPC investigation and preliminary electronics studies}}

\renewcommand{\bibname}{References}

\hyphenation{}

\chapter[Improved RPC investigation and preliminary electronics studies]%
{Improved RPC investigation and preliminary electronics studies}
\label{chapt6}

The extension in the endcap of the RPC sub-system towards higher pseudo-rapidity will bring the new detectors to be exposed to much more intense background radiations due to the proximity of the detectors with the beam line. The challenge will be to produce high counting rate detectors with limited ageing rate to ensure a stable operation of the detector over a period longer than 10 years. In the previous chapter was discussed the influence of the detector design (number and thickness of gas volumes, OR system, etc...) on the charge deposition and rate capability. Nevertheless, this question can also be adressed from the electronics point of view as a better signal to noise ratio would also mean the possibility to greatly lower the charge threshold on the signals to be detected, allowing to use the detector at lower gain, hence lowering the charge deposition per avalanche in the gas volume. Cardarelli showed that the production of low-noise fast FEE could help decreasing the charge deposition per avalanche at working voltage by an order of magnitude virtually increasing the life expectancy of such a detector in the same way~\cite{CARDARELLI2012}.

\section{FEE candidate for the production of iRPCs}

	\subsection{INFN \acl{FEE}}
	
	\subsection{PETIROC \acl{ToF} \acl{FEE}}

\section{Preliminary tests at CERN}

\section{Test }

\clearpage{\pagestyle{empty}\cleardoublepage}