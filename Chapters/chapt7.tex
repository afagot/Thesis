% Header
\renewcommand\evenpagerightmark{{\scshape\small Chapter 7}}
\renewcommand\oddpageleftmark{{\scshape\small Conclusions and outlooks}}

\hyphenation{}

\chapter[Conclusions and outlooks]{Conclusions and outlooks}
\label{chapt7}
	
	The CMS RPC upgrade has been and will keep on being an exciting scientific research. The collaboration converged towards the solutions that will be adopted in the perspective of HL-LHC. Eventhough the consolidation of the present CMS RPC infrastructure and the certification of the new technologies that will complete the redundancy of the muon system are still ongoing, the future of the experiment from the RPC point of view is now clear.
	
	To reach this point, the contribution of Ghent during the preliminary phase of tests between 2012 and 2015 has been decisive in both the consolidation of the present detectors and in the selection of the \acl{FEE} that will equip the iRPCs. At every step, Ghent University played a leading role in setting up the experiments but also in gathering and analysing the data. First of all, two potential FEE technologies were selected and it was showed that both of them could be used for new CMS detectors. On one side, the INFN amplifier provided a very interesting sensitivity to low charge depositions. On the other hand, the FEEs developed by OMEGA (HARDROC, PETIROC) showed that they could provide a reduction of the working voltage at similar charge deposition ranges. Moreover, this technology had already been certified through multiple experiments using detectors such as scintillators and RPCs. Finally, it had the advantage of proposing a 2D read-out that would greatly improve the spatial resolution of the detectors in the radial direction. The expertise of the Intrumentation group was demonstrated in this campaign.
	
	As a natural continuation, a door was opened to join the GIF++ effort at key positions. A major contribution to the development of the \acl{DAQ}, \acl{DQM} and data analysis tools was provided and will keep helping the collaboration in conducting robust R\&D research in the future. Indeed, new young experts are emerging and taking over the tools to improve them with fresh ideas. So far, the CMS RPC group is on the way of certifying the current RPC system for the HL-LHC period. The Link-system will be upgraded and the present detectors should live through the high-luminosity phase of the LHC without important change in their performance. The RPC that will complete the redundancy of the muon system are being certified as well and show very good performance under high-irradiation that is so far foreseen to stay stable throughout the whole Phase-2.
	
	Nevertheless, the present thesis document only focusses on the R\&D produced by the CMS RPC on the present and new detection technologies that are and will be used at CMS. Few information about the very important research being conducted in order to find a replacement to the standard RPC gas mixture is provided. The outcome of this search for new gases will be of major interest as the restriction for the standard mixture will get harder.
	
	Once the R\&D will be complete, the next phase will consist in the upgrade of the RPC sub-system. Ghent will mainly take part in manufacturing the detectors for the expension of the endcaps as was already the case for the production of the RE4 detectors for the fourth endcap disk of CMS between 2012 and 2013. After LS3, the LHC will finally enter its high-luminosity phase and new breakthrough will be foreseen. The good performance of the RPCs and of all of CMS sub-systems will be important in this reguard and the skills developed during the present R\&D will become an important asset in maintaining the performance of the detectors at their best level.

%\renewcommand*{\thesection}{\thechapter.\arabic{section}}       % reset again to chaptnum.sectnum

\clearpage{\pagestyle{empty}\cleardoublepage}