% Header
\renewcommand\evenpagerightmark{{\scshape\small Chapter 5}}
\renewcommand\oddpageleftmark{{\scshape\small Longevity studies and Consolidation of the present CMS RPC subsystem}}

\renewcommand{\bibname}{References}

\hyphenation{}

\chapter[Longevity studies and Consolidation of the present CMS RPC subsystem]{Longevity studies and Consolidation of the present CMS RPC subsystem}
\label{chapt5}
    
	The RPC system, located in both barrel and endcap regions, provides a fast, independent muon trigger with a looser \pT threshold over a large portion of the pseudo-rapidity range (\psrapl{1.6}). During HL-LHC operations the expected conditions in terms of background and pile-up will make the identification and correct \pT assignment a challenge for the muon system. The goal of RPC upgrade is to provide additional hits to the Muon System with more precise timing. All this information will be elaborated by the Trigger System in a global way enhancing the performance of the trigger in terms of efficiency and rate control. The RPC Upgrade is based on two projects: an improved Link Board System and the extension of the RPC coverage up to \psrape{2.4}.

	The Link Board System is responsible for the processing, the synchronization and the zero-suppression the signals coming from the RPC FEBs. The Link Board components have been produced between 2006 and 2007 and will be subjected to ageing and failure on a long term scale. An upgraded Link Board System will overcome the ageing problems and will allow for a more precise timing information to the RPC hits from 25 to \SI{1.5}{ns}.\\
	In order to develop an improved RPC that fulfills CMS requirements, an extensive R\&D program is being conducted. The benefits of adding two new RPC layers in the innermost ring of stations 3 and 4 will be mainly observed in the neutron-induced background reduction and efficiency improvement for both trigger and offline reconstruction.

	The coverage of the RPC System up to higher pseudo-rapidity \psrape{2.1} was part of the original CMS TDR. Nevertheless, the expected background rates being higher than the certified rate capability of the present CMS RPCs in that region and the budget being limited, RPCs were restricted to a shorter range. Even though the iRPC technology that will equip the extension of the Muon System will be different than the current CMS RPC technology, it is necessary to certify the rate capability and longevity of the existing detectors as the radiation level will increase together with the increase of instantaneous luminosity of the LHC. For this purpose, spare RPC detectors built but not installed in CMS have been installed in different irradiation facilities, first of all, to certify the detectors to the new levels of irradiation they will be subjected to and, finally, to study their ageing and certify their good operation throughout the HL-LHC program.

\section{Testing detectors under extreme conditions}
\label{chapt5:sec:extreme}

	The upgrade from LHC to HL-LHC will increase the peak luminosity from \Ord{34} \siflux to reach \Sci{5}{34} \siflux, increasing in the same way the total expected background to which the RPC System will be subjected to. Mainly composed of low energy gammas, neutrons, and electrons and positrons from $p$-$p$ collisions, but also of low momentum primary and secondary muons, punch-through hadrons from calorimeters, and particles produced in the interaction of the beams with collimators, the background will mostly affect the regions of CMS that are the closest to the beam line, i.e. the RPC detectors located in the endcaps.
    
	\begin{figure}[H]
		\begin{subfigure}{0.5\linewidth}
			\centering
			\includegraphics[height=5cm]{fig/chapt5/Rate-vs-Lumi-Barrel.png}
			\caption{\label{fig:Rate-I-vs-Lumi:A}}
		\end{subfigure}
		\begin{subfigure}{0.5\linewidth}
			\centering
			\includegraphics[height=5cm]{fig/chapt5/Rate-vs-Lumi-Endcap.png}
			\caption{\label{fig:Rate-I-vs-Lumi:B}}
		\end{subfigure}
		\begin{subfigure}{0.5\linewidth}
			\centering
			\includegraphics[height=5cm]{fig/chapt5/Current-vs-Lumi-Barrel.png}
			\caption{\label{fig:Rate-I-vs-Lumi:C}}
		\end{subfigure}
		\begin{subfigure}{0.5\linewidth}
			\centering
			\includegraphics[height=5cm]{fig/chapt5/Current-vs-Lumi-Endcap.png}
			\caption{\label{fig:Rate-I-vs-Lumi:D}}
		\end{subfigure}
		\caption{\label{fig:Rate-I-vs-Lumi} Mean RPC Barrel (left column) and Endcap (right column) rate (top row) and current (bottom row) as a function of the instantaneous luminosity as measured in 2017 $p$-$p$ collision data.}
	\end{figure}

	Data collected over 2017, presented through Figure~\ref{fig:Rate-I-vs-Lumi}, allows to study the values of the background rate in all the RPC System. This was achieved thanks to a monitoring of the rates in each RPC rolls, where rolls correspond to the pseudo-rapidity partitioning of the readout electronics, and of the current in each HV channel. A linear dependence in between the mean rate or current with instantaneous luminosity is showed in selected runs with identical LHC running parameters. In Figure~\ref{fig:RPC-HL-LHC}, a linear extrapolation of the distribution of the background hit rate per unit area as well as the integrated charge is showed at a HL-LHC condition. The maximum rate per unit area in the endcap detectors at HL-LHC conditions is expected to be of the order of \SIrate{600} while the charge deposition should reach approximately \SI{800}{mC/cm^2}. These extrapolations are provided with a required safety factor 3 for the certification study.
    
	\begin{figure}[H]
		\begin{subfigure}{0.5\linewidth}
			\centering
			\includegraphics[height=5cm]{fig/chapt5/RPC-Rate-HL-LHC_2017.png}
			\caption{\label{fig:RPC-HL-LHC:A}}
		\end{subfigure}
		\begin{subfigure}{0.5\linewidth}
			\centering
			\includegraphics[height=5cm]{fig/chapt5/RPC-IC-HL-LHC_2016.png}
			\caption{\label{fig:RPC-HL-LHC:B}}
		\end{subfigure}
		\caption{\label{fig:RPC-HL-LHC} Figure~\ref{fig:RPC-HL-LHC:A}: The hit rate per region (Barrel, Endcap) is linearly extrapolated to HL-LHC highest instantaneous luminosity (\Sci{5}{34} \siflux) using the rate as a function of instantaneous luminosity recorded by RPCs in 2017 showing a linear dependence. Figure~\ref{fig:RPC-HL-LHC:B}: The integrated charge per region (Barrel, Endcap) is extrapolated to HL-LHC integrated luminosity (\SI{3000}{fb^{-1}}) using the data accumulated in 2016 in every HV channels.}
	\end{figure}
    
	\begin{figure}[H]
		\begin{subfigure}{0.5\linewidth}
			\centering
			\includegraphics[height=5cm]{fig/chapt5/Mean-Int-charge-Barrel.png}
			\caption{\label{fig:Mean-Int-Charge:A}}
		\end{subfigure}
		\begin{subfigure}{0.5\linewidth}
			\centering
			\includegraphics[height=5cm]{fig/chapt5/Mean-Int-charge-Endcap.png}
			\caption{\label{fig:Mean-Int-Charge:B}}
		\end{subfigure}
		\caption{\label{fig:Mean-Int-Charge} CMS RPC mean integrated charge in the Barrel region (Figure~\ref{fig:Mean-Int-Charge:A}) and the Endcap region (Figure~\ref{fig:Mean-Int-Charge:B}). The integrated charge per year is shown in blue. The red curve shows the evolution of the accumulated integrated charge through time. The blank period in 2013 and 2014 corresponds to LS1. The total integrated charge for the entire operation period (Oct.2009 - Dec.2017) is estimated to be about \SI{1.66}{mC/cm^2} in the Barrel and \SI{4.58}{mC/cm^2} in the Endcap.}
	\end{figure}

	In the past, extensive long-term tests were carried out at several gamma and neutron facilities certifying the detector performance. Both full size and small prototype RPCs have been irradiated with photons up to an integrated charge of $\sim$\SI{0.05}{C/cm^2} and $\sim$\SI{0.4}{C/cm^2} respectively and were certified for rates reaching \SIrate{200}~\cite{GIF2004,AGING2009}. Since the beginning of Run-I until December 2017, the RPC system provided stable operation and excellent performance and did not show any ageing effects for a maximum integrated charge in a detector of the order of \SI{0.01}{C/cm^2} - the average being of the order of \SI{2}{mC/cm^2} in the Barrel and \SI{5}{mC/cm^2} in the Endcap, closer to the beam line, as can be seen from Figure~\ref{fig:Mean-Int-Charge} - and a peak luminosity reaching \Sci{1.4}{34} \siflux during 2017 data taking period.
	
	To perform the necessary studies on the present CMS RPC detectors, facilities offering the possibility to irradiate the chambers are necessary in order to recreate HL-LHC conditions or stronger and study their performance through time. Such facilities exist at CERN and were exploited to conduct this study. A first series of preliminary studies were conducted in the former gamma facility of CERN (GIF) before its dismantlement. This preliminary study was used as a stepping stone towards the building of a more powerful irradiation fully dedicated to longevity studies of CMS and ATLAS subsystems in the perspective of HL-LHC. The period of preliminary work has also been a key moment in the elaboration and improvement of data acquisition, offline analysis and online monitoring tools that are extensively used in the new gamma irradiation facility.
	
		\subsection{GIF}
		\label{chapt5:ssec:GIF}
	
	\begin{figure}[H]
		\centering
		\includegraphics[width = \plotwidth]{fig/chapt5/GIF-Layout.pdf}\\
		\caption{\label{fig:GIFLayout} Layout of the test beam zone called X5c GIF at CERN. Photons from the radioactive source produce a sustained high rate of random hits over the whole area. The zone is surrounded by \SI{8}{m} high and \SI{80}{cm} thick concrete walls. Access is possible through three entry points. Two access doors for personnel and one large gate for material. A crane allows installation of heavy equipment in the area.}
	\end{figure}
		
	Located in the SPS West Area at the downstream end of the X5 test beam, the \acf{GIF} was a test area in which particle detectors were exposed to a particle beam in presence of an adjustable gamma background~\cite{AGOSTEO1999}. Its goal was to reproduce background conditions these detectors would suffer in their operating environment at LHC. GIF layout is showed in Figure ~\ref{fig:GIFLayout}. Gamma photons are produced by a strong $^{137}$Cs source installed in the upstream part of the zone inside a lead container. The source container includes a collimator, designed to irradiate a \SIsurface{6}{6}{m} area at \SI{5}{m} maximum to the source. A thin lens-shaped lead filter helps providing with a uniform out-coming flux in a vertical plane, orthogonal to the beam direction. The photon rate is controlled by further lead filters allowing the maximum rate to be limited and to vary within a range of four orders of magnitude. Particle detectors under test are then placed within the pyramidal volume in front of the source, perpendicularly to the beam line in order to profit from the homogeneous photon flux. Adjusting the background flux of photons can then be done by using the filters and choosing the position of the detectors with respect to the source.
			
	As described on Figure~\ref{fig:CsSource}, the $^{137}$Cs source emits a \SI{662}{keV} photon in 85\% of the decays. An activity of \SI{740}{GBq} was measured on the \Th{5} of March 1997. To estimate the strength of the flux in 2014, was considered the nuclear decay through time associated to the Cesium source whose half-life is well known ($t_{1/2}=$ \SIerror{30.05}{0.08}{y}). The GIF tests where done in between the \Th{20} and the \St{31} of August 2014, i.e. at a time $t=$ \SIerror{17.47}{0.02}{y} resulting in an attenuation of the activity from \SI{740}{GBq} in 1997 to \SI{494}{GBq} in 2014.

	\begin{figure}[H]
		\centering
		\includegraphics[width = \plotwidth]{fig/chapt5/Cs137.pdf}\\
		\caption{\label{fig:CsSource} $^{137}$Cs decays by $\beta^-$ emission to the ground state of $^{137}$Ba (BR = 5.64\%) and via the \SI{662}{keV} isomeric level of $^{137}$Ba (BR = 94.36\%) whose half-life is 2.55min.}
	\end{figure}
		
		\subsection{GIF++}
		\label{chapt5:ssec:GIF++}
		
	The \acf{GIF++}, located in the SPS North Area at the downstream end of the H4 test beam, has replaced its predecessor during LS1 and has been operational since spring 2015~\cite{JAKEL2014}. Like GIF, GIF++ features a $^{137}$Cs source of \SI{662}{keV} gamma photons, their fluence being controlled with a set of filters of various attenuation factors. The source provides two separated large irradiation areas for testing several full-size muon detectors with continuous homogeneous irradiation, as presented in Figure~\ref{fig:GIFpp-Layout}.
	
	 The source activity was measured to be about \SI{13.5}{TBq} in March 2016. The photon flux being far greater than HL-LHC expectations, GIF++ provides an excellent facility for accelerated ageing tests of muon detectors. The source is situated in a bunker designed to perform irradiation test along a muon beam line, the muon beam being available during selected periods throughout the year. The H4 beam, providing the area with muons with a maximum momentum of about \SI{150}{GeV/c}, passes through the GIF++ zone and is used to periodically study the performance of the detectors placed under continuous irradiation. Its flux is of \SI{104}{particles/s/\square\cm} focused in an area similar to \SIsurface{10}{10}{cm}. Therefore, with properly adjusted filters, one can simulate the background expected at HL-LHC and study the performance and ageing of muon detectors in HL-LHC environment.\\
	
	\begin{figure}[H]
		\centering
		\includegraphics[width = 0.8\plotwidth]{fig/chapt5/GIFpp-Layout.png}\\
		\caption{\label{fig:GIFpp-Layout} Floor plan of the GIF++ facility. When the facility downstream of the GIF++ takes electron beam, a beam pipe is installed along the beam line (z-axis). The irradiator can be displaced laterally (its center moves from $x=$ \SI{0.65}{m} to \SI{2.15}{m}), to increase the distance to the beam pipe.}
	\end{figure}
	 
	\begin{figure}[H]
		\begin{subfigure}{0.5\linewidth}
		    \centering
			\includegraphics[width = 0.6\plotwidth]{fig/chapt5/GIFpp-gCurrent-XZ.png}\\
			\caption{\label{fig:GIFpp-gCurrent:XZ}}
		\end{subfigure}
		\begin{subfigure}{0.5\linewidth}
		    \centering
			\includegraphics[width = 0.6\plotwidth]{fig/chapt5/GIFpp-gCurrent-YZ.png}
			\caption{\label{fig:GIFpp-gCurrent:YZ}}
		\end{subfigure}
		\caption{\label{fig:GIFpp-gCurrent} Simulated unattenuated current of photons in the xz plane (Figure~\ref{fig:GIFpp-gCurrent:XZ}) and yz plane (Figure~\ref{fig:GIFpp-gCurrent:YZ}) through the source at $x=$ \SI{0.65}{m} and $y=$ \SI{0}{m}. With angular correction filters, the current of \SI{662}{keV} photons is made uniform in xy planes.}
    \end{figure}
	 
\section{Preliminary studies at GIF}
\label{chapt5:sec:GIFtests}

	\subsection{\acl{RPC} test setup}
	\label{chapt5:ssec:RPCSetup}
	
	During summer 2014, preliminary tests have been conducted in the GIF area on an endcap chamber of type RE4/2 and labeled \texttt{RE-4-2-BARC-161} produced for the extension of the endcap with a fouth disk in 2013. This chamber has been placed into a trolley covered with a tent. The position of the RPC inside the tent and of the tent with respect to the source in the bunker are described in Figure~\ref{fig:GIFSetup}. The goal of the study were to have a preliminary understanding of the rate capability of the present technology used in CMS. It was decided to measure the efficiency of the RPC under irradiation at detecting cosmic muons as, at the time of the tests, the beam not operational anymore. Three different absorber settings were used and compared to the case where the detector was not irradiated in order to study the evolution of the performance of the detector with increasing exposition to gamma radiation. First of all, measurements were done with fully opened source. To complete this preliminary study, the gamma flux has been attenuated by a factor 2, a factor 5 and finally the source was shut down. Was measured the efficiency of the RPC at detecting the cosmic muons in coincidence with a cosmic trigger as well as the background rate as seen by the detectors.

	\begin{figure}[H]
		\begin{subfigure}{0.5\linewidth}
		    \centering
			\includegraphics[width = 0.5\plotwidth]{fig/chapt5/GIF-Setup-A.pdf}
			\caption{\label{fig:GIFSetup:A}}
		\end{subfigure}
		\begin{subfigure}{0.5\linewidth}
		    \centering
			\includegraphics[width = 0.5\plotwidth]{fig/chapt5/GIF-Setup-B.pdf}
			\caption{\label{fig:GIFSetup:B}}
		\end{subfigure}
		\caption{\label{fig:GIFSetup} Description of the RPC setup. Dimensions are given in \si{mm}. A tent containing RPCs is placed at \SI{1720}{mm} from the source container. The source is situated in the center of the container. RE-4-2-BARC-161 chamber is \SI{160}{mm} inside the tent. This way, the distance between the source and the chambers plan is \SI{2060}{mm}. Figure~\ref{fig:GIFSetup:A} provides a side view of the setup in the $xz$ plane while Figure~\ref{fig:GIFSetup:B} shows a top view in the $yz$ plane.}
	\end{figure}
	
	The trigger system was composed of 2 plastic scintillators and was placed in front of the setup with an inclination of \SI{10}{\degree} with respect to the detector plane in order to look at cosmic muons. Using this particular trigger layout, showed in Figure~\ref{fig:GIF-RPCSetup}, leads to a cosmic muon hit distribution into the chamber similar to the one of Figure~\ref{fig:HitProf}. Measured without gamma irradiation, two peaks can be seen on the profile of readout partition B, centered on strips 52 and 59. Section~\ref{chapt5:ssec:GeoAcc} will help us understand that these two peaks are due respectively to forward and backward coming cosmic particles where forward coming particles are first detected by the scintillators and then the RPC while the backward coming muons are first detected in the RPC.

	\begin{figure}[H]
		\centering
		\includegraphics[width = 0.8\plotwidth]{fig/chapt5/GIF-RPCSetup.jpg}\\
		\caption{\label{fig:GIF-RPCSetup} \texttt{RE-4-2-BARC-161} chamber is inside the tent as described in Figure~\ref{fig:GIFSetup}. In the top right, the two scintillators used as trigger can be seen. This trigger system has an inclination of \SI{10}{\degree} relative to horizontal and is placed above half-partition B2 of the RPCs. PMT electronics are shielded thanks to lead blocks placed in order to protect them without stopping photons from going through the scintillators and the chamber.}
	\end{figure}

		\begin{figure}[H]
		\centering
		\includegraphics[width = 0.7\plotwidth]{fig/chapt5/Data-21-profile.pdf}\\
		\caption{\label{fig:HitProf} Hit distributions over all 3 partitions of \texttt{RE-4-2-BARC-161} chamber is showed. Top, middle and bottom figures respectively correspond to partitions A, B, and C. The profiles show that some events still occur in other half-partitions than B2, which corresponds to strips 49 to 64, in front of which the trigger is placed, contributing to the inefficiency of detection of cosmic muons. In the case of partitions A and C, the very low amount of data can be interpreted as noise. On the other hand, it is clear that a little portion of muons reach the half-partition B1, corresponding to strips 33 to 48.}
	\end{figure}
	
	The data taking is then performed thanks to a \texttt{CEAN} TDC module of type \texttt{V1190A}~\cite{V1190AMUT} to which is connected the digitized output of the RPC \acl{FEB}, as described in Figure~\ref{fig:DAQ:A} and the trigger signal from the telescope. The communication with the computer is performed thanks to a \texttt{CAEN} communication module of type \texttt{V1718}~\cite{V1718MUT}. In order to control the rates recorded by the detector, the digitized RPC signals are also sent to scalers as described in Figure~\ref{fig:DAQ:B}. The \texttt{C++} DAQ software used in GIF was developed as an early attempt towards the understanding of the \texttt{CAEN} libraries and the data collected by the TDCs was saved into \texttt{.dat} files is analysed with an algorithm computing parameters such as efficiency, hit profile, cluster size, or gamma and noise rates which was developed with \texttt{C++} as well. Finally, histograms and curves are produced using \texttt{ROOT}.

	\begin{figure}[H]
		\begin{subfigure}{\linewidth}
			\centering
			\includegraphics[width = \plotwidth]{fig/chapt5/pulse-processing.pdf}\\
			\caption{\label{fig:DAQ:A}}
		\end{subfigure}
		\begin{subfigure}{\linewidth}
			\centering
			\includegraphics[width = \plotwidth]{fig/chapt5/pulse-processing-2.pdf}
			\caption{\label{fig:DAQ:B}}
		\end{subfigure}
		\caption{\label{fig:DAQ} Signals from the RPC strips are shaped by the FEE described on Figure ~\ref{fig:DAQ:A}. Output LVDS signals are then read-out by a TDC module connected to a computer or converted into NIM and sent to scalers. Figure~\ref{fig:DAQ:B} describes how these converted signals are put in coincidence with the trigger.}
	\end{figure}
	
	\subsection{Geometrical acceptance of the setup layout to cosmic muons}
	\label{chapt5:ssec:GeoAcc}
				
	In order to profit from a constant gamma irradiation, the detectors inside of the GIF bunker need to be placed in a plane orthogonal to the beam line. The muon beam that used to be available was meant to test the performance of detectors under test. This beam not being active anymore, an other solution to test detector performance had to be used. Thus, it has been decided to use cosmic muons detected through a telescope composed of two scintillators. Lead blocks were used as shielding to protect the photomultipliers from gammas as can be seen from Figure~\ref{fig:GIF-RPCSetup}.
				
	An inclination of $\sim$\SI{10}{\degree} has been given to the cosmic telescope to maximize the muon flux. A good compromise had to be found between good enough muon flux and narrow enough hit distribution to be sure to contain all the events into only one half partitions as required from the limited available readout hardware. It was then foreseen to detect muons and read them out only from half-partition B2, the last 16 channels of readout partition B (i.e. strips 49 to 64). Nevertheless, a consequence of the misplaced trigger, that can be seen as a loss of events in half-partition B1 (strips 33 to 48) in Figure~\ref{fig:HitProf}, is an inefficiency. The observed inefficiency of approximately 20\% highlighted in Figure~\ref{fig:EffCompar} by comparing the performance of chamber \texttt{RE-4-2-BARC-161} as measured prior to the study at GIF and at GIF without irradiation seems too important, compared to the 12.7\% of data contained into the first 16 strips observed on Figure~\ref{fig:HitProf}, to only be explained by the geometrical acceptance of the setup itself. Simulations have been conducted to show how the setup brings inefficiency.

	\begin{figure}[H]
            \centering
		\includegraphics[width = \plotwidth]{fig/chapt5/Compared-Efficiency.pdf}
		\caption{\label{fig:EffCompar} Results are derived from data taken on half-partition B2 only. On the \Th{18} of June 2014, data has been taken on chamber \texttt{RE-4-2-BARC-161} at CERN building 904 (Prevessin Site) with cosmic muons providing us a reference efficiency plateau of \numerror{97.54}{0.15}\% represented by a black curve. A similar measurement has been done at GIF on the \St{21} of July with the same chamber giving a plateau of \numerror{78.52}{0.94}\% represented by a red curve.}
	\end{figure}
	
		\subsubsection{Description of the simulation layout}
		\label{chapt5:sssec:SimLayout}
		
	The layout of GIF setup has been reproduced, only roughly using Figure~\ref{fig:GIF-RPCSetup} due to the lack of measures, and incorporated into a \texttt{C++} \acf{MC} simulation to study the geometrical acceptance of the telescope projected onto the readout strips~\cite{GEOACCEPT}. A 3D view of the simulated layout is given into Figure~\ref{fig:SimGIFLay}. Muons are generated randomly in a horizontal plane located at a height corresponding to the lowest point of the PMTs. This way, the needed size of the plane in order to simulate events happening at very large azimuthal angles (i.e. $\theta\approx\pi$) can be kept relatively small while the total number of muon tracks to propagate is kept relatively small. The muon flux is designed to follow the usual $cos^2\theta$ distribution for cosmic particles. The goal of the simulation is to look at muons that pass through the telescope composed of the two scintillators and define their distribution onto the RPC read-out plane. During the reconstruction, the read-out plane is then divided into read-out strips and each muon track is assigned to a strip.

	\begin{figure}[H]
		\begin{subfigure}{\linewidth}
			\centering
			\includegraphics[width = 0.8\plotwidth]{fig/chapt5/GIFSetup-SimA.png}\\
			\caption{\label{fig:SimGIFLay:A}}
		\end{subfigure}
		\begin{subfigure}{\linewidth}
			\centering
			\includegraphics[width = 0.8\plotwidth]{fig/chapt5/GIFSetup-SimB.png}
			\caption{\label{fig:SimGIFLay:B}}
		\end{subfigure}
		\caption{\label{fig:SimGIFLay} Representation of the layout used for the simulations of the test setup. The RPC read-out plane is represented as a yellow trapezoid while the two scintillators as blue cuboids looking at the sky. The green plane corresponds to the muon generation plane within the simulation. Figure~\ref{fig:GIFSetup:A} shows a global view of the simulated setup. Figure~\ref{fig:GIFSetup:B} shows a zoomed view that allows to see the 2 scintillators as well as the full RPC plane.}
	\end{figure}
		
		\subsubsection{Simulation procedure}
		\label{chapt5:sssec:SimProc}
		
	$N_{\mu}=$ \Ord{8} muons are randomly generated inside the muon plane with an azimutal angle $\theta$ chosen to follow a $cos^2\theta$ distribution. Infinite planes are associated to each surface of the scintillators. Knowing the muon position into the muon generation plane and its direction allows, by assuming that muons travel in a straight line, to compute the intersection of the muon track with these planes. Applying conditions to the limits on the contours of the scintillators' faces then gives an answer to weither or not the muon passed through the scintillators. In the case the muon was not \textit{detected} into both scintillators, the simulation discards the muon and generates a new one.
	
	On the contrary, if the muon is labeled as good, its position within the RPC read-out plane is computed and the corresponding strip, determined through geometrical tests, gets a \textit{hit}. Muon hits fill different histograms weither they are associated to forward or backward coming muons. A discrimination is performed according to their direction components. An $(x,y,z)$ position  into the generation plane as well as a ($\theta$;$\phi$) pair are associated to each generated muon providing with information on the direction the track follows. This way, muons satisfying the condition $0\leq\phi<\pi$ are labeled as \textit{backward} coming muons while muons satisfying $\pi\leq\phi<2\pi$ as \textit{forward} coming muons.
		
		\subsubsection{Results and limitations}
		\label{chapt5:sssec:SimRes}
	
	The output from the simulation is given in Figure~\ref{fig:SimResult} in which the distribution is showed for all muons but also for the separate contributions of forward and backward coming muons. The strip number is here given in a range of 1 to 32 corresponding to the 32 strips contained in each RPC read-out partition, without taking into account the fact that partition B of an RPC correponds, by convention, to strips 33 to 64. Comparing the number of muons recoreded respectively in the first 16 strips and the in all of the 32 strips of the RPC read-out panel, it can be established than, out of the total amount of muons that have passed through the telescope and reached the RPC, 16.8\% where to be detected in the 16 first strip of the read-out plane corresponding to half partition B1. This brings a geometrical inefficiency of the same amount that can then be used to correct the data by scaling up by a factor $c_{geo} = 1/(1-0.168)$ the maximum efficiency measured during data taking.

	\begin{figure}[H]
		\begin{subfigure}{\linewidth}
			\centering
			\includegraphics[width = 0.6\plotwidth]{fig/chapt5/Geometrical-acceptance.pdf}\\
			\caption{\label{fig:SimResult:A} Full acceptance distribution}
		\end{subfigure}
		\begin{subfigure}{0.5\linewidth}
			\centering
			\includegraphics[width = 0.6\plotwidth]{fig/chapt5/Geometrical-acceptance-forward.pdf}
			\caption{\label{fig:SimResult:B} Forward acceptance}
		\end{subfigure}
		\begin{subfigure}{0.5\linewidth}
			\centering
			\includegraphics[width = 0.6\plotwidth]{fig/chapt5/Geometrical-acceptance-backward.pdf}
			\caption{\label{fig:SimResult:C} Backward acceptance}
		\end{subfigure}
		\caption{\label{fig:SimResult} Geometrical acceptance distribution as provided by the \acl{MC} simulation.}
	\end{figure}
	
	Nevertheless, it is difficult to evaluate a systematical uncertainty on this geometrical correction for different reasons. First of all, eventhough the dimensions of the scintillators and of the RPC are well known, the position of each element of the setup with respect to one another was not measured. It was then necessary, using known dimensions, to extract the positions of each element from Figure~\ref{fig:GIF-RPCSetup} with unknown uncertainty. The inclination is also roughly measured to be \SI{10}{\degree} and even if the position of each peak, distant in the simulation of 7 strips, tends to confirm this assumption, the geometrical inefficiency would be affected by a variation of the inclination angle. Introducing in the simulation an error of $\pm$\SI{2}{\degree} would lead to a correction factor $c_{geo} = 1.20^{+0.04}_{-0.03}$ that allows for a good improvement of the efficiency measured in GIF, as can be seen from Figure~\ref{fig:EffCorrection}. GIF measurement is in agreement 

	\begin{figure}[H]
            \centering
		\includegraphics[width = \plotwidth]{fig/chapt5/Compared-Efficiency-Correction.pdf}
		\caption{\label{fig:EffCorrection} Correction of the efficiency without source. The efficiency after correction gets much closer to the Reference measurement performed before the study in GIF by reaching a plateau of \numerror{93.52}{2.64}\%.}
	\end{figure}
	
	Further corrections could be also be brought as it can easily be understood that the distribution showed through Figure~\ref{fig:SimResult:A} differs from the measured hit profile showed in Figure~\ref{fig:HitProf}. The contributions of forward and backward muon indicate that 28.1\% of the total geometrical acceptance should contribute to detecting backward muons whereas it is measured that the hit profile contains 22.0\% of backward data only. This estimation of the backward versus forward content in the data was done through a fit using a sum of two skew distribution, one acting on the forward muon peak while the other acts on the backward muon fit, as showed in Figure~\ref{fig:Fit-data}. Although a skew distribution lacks physical interpretation, it allows to easily fit such kind of data. A description of a skew distribution, as the product of a gaussian and a sigmoid (Formula~\ref{eq:gaus-sig}), is given through Formula~\ref{eq:skew}.
	
	\begin{equation}
	\label{eq:gaus-sig}
	g(x) = A_g e^{\frac{-(x-\bar{x})^2}{2\sigma^2}}\; , \quad s(x) = \frac{A_s}{1+e^{-\lambda(x-x_i)}}
	\end{equation}
	
	\begin{equation}
	\label{eq:skew}
	sk(x) = g(x)\times s(x) = A_{sk}\frac{e^{\frac{-(x-\bar{x})^2}{2\sigma^2}}}{1+e^{-\lambda(x-x_i)}}
	\end{equation}

	\begin{figure}[H]
            \centering
		\includegraphics[width = 0.7\plotwidth]{fig/chapt5/Cosmic-data-21-skew-fit.pdf}
		\caption{\label{fig:Fit-data} Hit distributions over read-out partition B of \texttt{RE-4-2-BARC-161} chamber together with skew distribution fits corresponding to forward and backward coming muons.}
	\end{figure}
	
	From the obvious difference in between geometrical simulation and data, it is necessary to realize that the geometrical acceptance and the hit profile are two dinstinct information. When the geometrical acceptance only provides with the information about what the detector can expect to see in a perfect world where all muons are detected in the exact same way independantly from their energy, angle of incidence, fluctuation of the detector gain due to complexe avalanche development, thresholds applied on the scintillators and on the RPC FEEs to reduce the noise, the cross-talk and corresponding spread of the induced charge observed on the read-out strips, the hit profile provides the final product of all the previously mentioned contributions and can greatly differ from purely geometrical considerations. A full physics analysis involving softwares such as \texttt{GEANT} would be required to further refine the correction on the measured efficiency at GIF.
	
	\subsection{Photon flux at \acs{GIF}}
	\label{chapt5:ssec:gFlux}
		
	In order to understand and evaluate the $\gamma$ flux in the GIF area, simulations had been conducted at the time GIF was opened for research purposes~\cite{AGOSTEO1999}. Table~\ref{tab:Sim1997} presented in this article gives the $\gamma$ flux for different distances $D$ to the source. The simulation was done using GEANT and a \acf{MCNP} transport code, and the flux $F$ is given with the estimated error from these packages expressed in \%.
	
	\begin{table}[H]
		\centering
		\begin{tabular}{|*{5}{c|}}
			\hline
			Nominal & \multicolumn{4}{c|}{Photon flux $F$ [\siflux]} \\
			\cline{2-5}
			ABS & at $D=$ \SI{50}{cm} & at $D=$ \SI{155}{cm} & at $D=$ \SI{300}{cm} & at $D=$ \SI{400}{cm} \\
			\hline
			1 & \Sci{0.12}{8} $\pm$ 0.2\% & \Sci{0.14}{7} $\pm$ 0.5\% & \Sci{0.45}{6} $\pm$ 0.5\% & \Sci{0.28}{6} $\pm$ 0.5\% \\
			\hline
			2 & \Sci{0.68}{7} $\pm$ 0.3\% & \Sci{0.80}{6} $\pm$ 0.8\% & \Sci{0.25}{6} $\pm$ 0.8\% & \Sci{0.16}{6} $\pm$ 0.6\% \\
			\hline
			5 & \Sci{0.31}{7} $\pm$ 0.4\% & \Sci{0.36}{6} $\pm$ 1.2\% & \Sci{0.11}{6} $\pm$ 1.2\% & \Sci{0.70}{5} $\pm$ 0.9\% \\
			\hline
		\end{tabular}
		\caption{\label{tab:Sim1997} Total photon flux ($E\gamma \leq$ \SI{662}{keV}) with statistical error predicted considering a $^{137}$Cs activity of \SI{740}{GBq} at different values of the distance $D$ to the source along the x-axis of irradiation field~\cite{AGOSTEO1999}.}
	\end{table}
	
	The simulation does not provide with an estimated flux at the level of the RPC under test. First of all, it is necessary to extract the value of the flux from the available data contained in the original paper and then to estimate the flux in 2014 at the time the experimentation took place. In the case of a pointlike source emiting isotrope and homogeneous gamma radiations, the gamma flux $F$ at a distance $D$ from the source with respect to a reference point situated at $D_0$ where a known flux $F_0$ is measured will be expressed like in Formula~\ref{eq:Flux}, assuming that the flux decreases as $1/D^2$, where $c$ is a fitting factor that can be written from Formula~\ref{eq:Flux} as Formula~\ref{eq:Factor}. Finally, using Equation~\ref{eq:Factor} and the data of Table~\ref{tab:Sim1997}, with $D_0=$ \SI{50}{cm} as reference point, Table~\ref{tab:CorrFactor} can be built. It is interesting to note that $c$ for each value of $D$ doesn't depend on the absorption factor.
	
	\begin{equation}
	\label{eq:Flux}
	F^{ABS} = F_0^{ABS} \times \left( \frac{c D_0}{D} \right)^2
	\end{equation}
	
	\begin{equation}
	\label{eq:Factor}
	c = \frac{D}{D_0}\sqrt{\frac{F^{ABS}}{F_0^{ABS}}} \; , \quad \Delta c = \frac{c}{2}\left(\frac{\Delta F^{ABS}}{F^{ABS}}+\frac{\Delta F^{ABS}_0}{F^{ABS}_0}\right)
	\end{equation}
	
	\begin{table}[H]
		\centering
		\begin{tabular}{|*{4}{c|}}
			\hline
			Nominal & \multicolumn{3}{c|}{Correction factor $c$} \\
			ABS & at $D=$ \SI{155}{cm} & at $D=$ \SI{300}{cm} & at $D=$ \SI{400}{cm} \\
			\hline
			1 & $1.059 \pm 0.70\%$ & $1.162 \pm 0.70\%$ & $1.222 \pm 0.70\%$ \\
			\hline
			2 & $1.063 \pm 1.10\%$ & $1.150 \pm 1.10\%$ & $1.227 \pm 0.90\%$ \\
			\hline
			5 & $1.056 \pm 1.60\%$ & $1.130 \pm 1.60\%$ & $1.202 \pm 1.30\%$ \\
			\hline
		\end{tabular}
		\caption{\label{tab:CorrFactor} Correction factor c is computed thanks to Formula~\ref{eq:Factor} taking as reference $D_0 =$ \SI{50}{cm} and the associated flux $F_0^{ABS}$ for each absorption factor available in table~\ref{tab:Sim1997}.}
	\end{table}
	
	For the range of $D/D_0$ values available, it is possible to use a simple linear fit to get the evolution of $c$ that can be expressed as $c(D/D_0)=aD/D_0+b$. Using Formula~\ref{eq:FluxLinearAp}, but neglecting the incertainty on $D$ that will only be used when extrapolating the values for the position of the RPC under test whose position is not perfectly known, the results showed in Figure~\ref{fig:CorrFactor} is obtained. Figure~\ref{fig:CorrFactor:B} confirms that using only a linear fit to extract $c$ is enough as the evolution of the rate that can be obtained superimposes well on the simulation points.
	
	\begin{equation}
	\label{eq:FluxLinearAp}
	F^{ABS} = F^{ABS}_0 \left( a + \frac{bD_0}{D} \right)^2 \; , \quad \Delta F^{ABS} = F^{ABS} \left[\frac{\Delta F^{ABS}_0}{F^{ABS}_0} + 2\frac{\Delta a + \Delta b\frac{D_0}{D} + \Delta D\frac{bD_0}{D^2}}{a + \frac{bD_0}{D}}\right]
	\end{equation}
	
	In the context of the 2014 GIF tests, the RPC read-out plane is located at a distance $D=$ \SI{206}{cm} from the source. Moreover, to estimate the strength of the flux in 2014 it is necessary to consider the nuclear decay through time assiciated to the Cesium source whose half-life is well known ($t_{1/2}=$ \SIerror{30.05}{0.08}{y}). The very first source activity measurement has been done on the \Th{5} of March 1997 while the GIF tests where done in between the \Th{20} and the \Th{31} of August 2014, i.e. at a time $t=$ \SIerror{17.47}{0.02}{y} resulting in an attenuation of the activity from \SI{740}{GBq} in 1997 to \SI{494}{GBq} in 2014. All the needed information to extrapolate the expected flux through the detector at the moment of GIF preliminary tests has now been assembled, leading to Table~\ref{tab:extra2014}. By assuming a sensitivity of the RPC to $\gamma$ of \Sci{2}{-3}, the order of magnitude of the expected hit rate per unit area would be of the order of the \si{kHz} for the fully opened source, as reported in the last column of the table.
	
	\begin{figure}[H]
		\begin{subfigure}{\linewidth}
			\centering
			\includegraphics[width = 0.8\plotwidth]{fig/chapt5/flux_correction.pdf}\\
			\caption{\label{fig:CorrFactor:A}}
		\end{subfigure}
		\begin{subfigure}{\linewidth}
			\centering
			\includegraphics[width = 0.8\plotwidth]{fig/chapt5/correction_model.pdf}
			\caption{\label{fig:CorrFactor:B}}
		\end{subfigure}
		\caption{\label{fig:CorrFactor} Figure~\ref{fig:CorrFactor:A} shows the linear approximation fit performed on data extracted from table~\ref{tab:CorrFactor}. Figure~\ref{fig:CorrFactor:B} shows a comparison of Formula~\ref{eq:FluxLinearAp} with the simulated flux using $a$ and $b$ given in figure~\ref{fig:CorrFactor:A} in formulae ~\ref{eq:Flux} and the reference value $D_0 =$ \SI{50}{cm} and the associated flux for each absorption factor $F_0^{ABS}$ from table~\ref{tab:Sim1997}}
	\end{figure}
	
	\begin{table}[H]
		\begin{tabular}{|*{5}{c|}}
			\hline
			Nominal & \multicolumn{3}{c|}{Photon flux $F$ [\siflux]} & Rate [\sirate] \\
			ABS & at $D_0^{97}=$ \SI{50}{cm} & at $D^{97}=$ \SI{206}{cm} & at $D^{2014}=$ \SI{206}{cm} & at $D^{2014}=$ \SI{206}{cm} \\
			\hline
			1 & \Sci{0.12}{8} $\pm$ 0.2\% & \Sci{0.84}{6} $\pm$ 1.2\% & \Sci{0.56}{6} $\pm$ 1.2\% & $1129 \pm 14$ \\
			\hline
			2 & \Sci{0.68}{7} $\pm$ 0.3\% & \Sci{0.48}{6} $\pm$ 1.2\% & \Sci{0.32}{6} $\pm$ 1.2\% & $640 \pm 8$ \\
			\hline
			5 & \Sci{0.31}{7} $\pm$ 0.4\% & \Sci{0.22}{6} $\pm$ 1.2\% & \Sci{0.15}{6} $\pm$ 1.2\% & $292 \pm 4$ \\
			\hline
		\end{tabular}
		\caption{\label{tab:extra2014} The data at $D_0$ in 1997 is taken from~\cite{AGOSTEO1999}. Using Formula~\ref{eq:FluxLinearAp}, the flux at $D$, including an error of \SI{1}{cm}, can be estimated in 1997. Then, taking into account the attenuation of the source activity, the flux at $D$ can be estimated at the time of the tests in GIF in 2014. Assuming a sensitivity of the RPC to $\gamma$ $s =$ \Sci{2}{-3}, an estimation of the hit rate per unit area is obtained.}
	\end{table}
	
	The goal of the study will be to have a good measurement of the intrinsic performance without source irradiation. Then, taking profit of the two working absorbers, at absorbtion factors 5 (\SI{300}{Hz}) and 2 ($\sim$\SI{600}{Hz}) the goal will be to show that the detectors fulfill the performance certification of CMS RPCs. Finally, a first idea of the performance of the detectors at higher background will be provided with absorbtion factor 1 (no absorbtion and $>$\SI{1}{kHz})).
	
	\subsection{Results and discussions}
	\label{chapt5:ssec:results6}
	
	The data taking at GIF has been conducted in between the \St{21} and the \St{31} of August, 2014. Data has been collected with both source OFF and ON using three different absorber settings (ABS 5, 2 and 1) in order to vary the irradiation on the RPC. For each source setting, two HV scans have been performed with two different trigger settings. During a first scan the trigger sent to the TDC module was the coincidence of the two scintillators composing the telescope while during a second scan the trigger was a pulse coming from a pulse generator in order to measure the noise or gamma rate seen by the chamber. Indeed, using a pulse allows to trigger at moments not linked to any physical event and, hence, to obtain a \textit{RANDOM} trigger on noise and gamma events to measure the associated rates, the probability to have a pulse in coincidence with a cosmic muon being negligible.
	
	From the cosmic trigger scans, a summary of the effiencies and corresponding cluster sizes is showed in Figure~\ref{fig:GIFEffCS}. The efficiency curves with Source ON show a shift with respect to the case without irradiation. With ABS 5, the general shape of the efficiency curve stays unchanged whereas a clear alteration of the performance is observed at ABS 2 and ABS 1. From the cluster size results, a reduction of the cluster size under irradiation can be oberved at equivalent efficiency. This effect can be due to the perturbation of the electric field by the strong rate of gamma particles starting avalanches in the gas volume of the detector.
	
	\begin{figure}[H]
    	\begin{subfigure}{0.5\linewidth}
			\centering
			\includegraphics[width = 0.7\plotwidth]{fig/chapt5/Efficiency.pdf}
        	\caption{\label{fig:GIFEffCS:A}}
    	\end{subfigure}
    	\begin{subfigure}{0.5\linewidth}
			\centering
			\includegraphics[width = 0.7\plotwidth]{fig/chapt5/Cluster-Size.pdf}
        	\caption{\label{fig:GIFEffCS:B}}
    	\end{subfigure}
		\caption{\label{fig:GIFEffCS} Efficiency (Figure~\ref{fig:GIFEffCS:A}) and cluster size (Figure~\ref{fig:GIFEffCS:B}) of chamber \texttt{RE-4-2-BARC-161} measured at GIF with Source OFF (red) and Source ON using different absorber settings: ABS 5 (green), ABS 2 (blue) and ABS 1 (yellow). The results are compared to the Reference values obtained with cosmics.}
	\end{figure}
	
	It is necessary to study the evolution of the performance of the chamber with the increasing rate. In Figure~\ref{fig:GIFRate:A}, the noise rate when the source is OFF stays low but increases at voltages above \SI{9500}{V}. The rise of the noise rate in the detector can be related to the increased streamer probability observed with such a large electric field. The rates with source ON measured at GIF all show a similar behaviour until a high voltage of approximately \SI{9400}{V} at which the rate of ABS 5 saturates, corresponding to the chamber reaching full efficiency. It is important to note that, eventhough the rates look similar independently from the gamma flux, relatively to the efficiency of the chamber, the rate actually increases with increasing flux at equivalent efficiency. A rough way to measure the rate effectively observed by the detector for each source setting would be to normalize the measured rates to the efficiency of the detector. This exercise was done with Figure~\ref{fig:GIFRate:B} from which constant fits where done on Source ON data in order to extract the rate the chamber was subjected to.
	
	\begin{figure}[H]
    	\begin{subfigure}{0.5\linewidth}
			\centering
			\includegraphics[width = 0.7\plotwidth]{fig/chapt5/Gamma-Rate.pdf}
        	\caption{\label{fig:GIFRate:A}}
    	\end{subfigure}
    	\begin{subfigure}{0.5\linewidth}
			\centering
			\includegraphics[width = 0.7\plotwidth]{fig/chapt5/Unconvoluted-Gamma-Rate.pdf}
        	\caption{\label{fig:GIFRate:B}}
    	\end{subfigure}
		\caption{\label{fig:GIFRate} Rates in chamber \texttt{RE-4-2-BARC-161} measured at GIF with Source OFF (red) and Source ON using different absorber settings: ABS 5 (green), ABS 2 (blue) and ABS 1 (yellow). On Figure~\ref{fig:GIFRate:B}, the rates of Figure~\ref{fig:GIFRate:A} were normalized to the measured efficiency and constant fits are performed on Source ON data showing the gamma rate in the chamber.}
	\end{figure}
	
	\begin{figure}[H]
    	\begin{subfigure}{0.5\linewidth}
			\centering
    		\includegraphics[width = 0.7\plotwidth]{fig/chapt5/HV-Knee-ABS.pdf}
        	\caption{\label{fig:Evolution:A}}
    	\end{subfigure}
    	\begin{subfigure}{0.5\linewidth}
			\centering
    		\includegraphics[width = 0.7\plotwidth]{fig/chapt5/Eff-ABS.pdf}\\
        	\caption{\label{fig:Evolution:B}}
    	\end{subfigure}
		\caption{\label{fig:Evolution} Evolution of the voltages at half maximum and at 95\% of the maximum efficiency (Figure~\ref{fig:Evolution:A}), and of the maximum efficiency (Figure~\ref{fig:Evolution:B}) as a function of the rate in chamber \texttt{RE-4-2-BARC-161}. The data is extracted from the fits in Figures~\ref{fig:GIFEffCS:A} and~\ref{fig:GIFRate:B}.}
	\end{figure}
	
	The results need to be taken with care as a better estimation of the rate would have been to push the detector towards higher voltages to reach the efficiency plateau for each absorber configuration and only then extract the measured rate at working voltage, defined as in Formula~\ref{eq:KneeWP}. Nevertheless, using this method to estimate the rate the chamber is subjected to, it is possible to look at the evolution of the $HV_{50}$ and $HV_{knee}$ (the working voltage being defined to be \SI{150}{V} above the knee in the endcap) as a function of the increasing rate as showed in Figure~\ref{fig:Evolution}. The results from GIF suggest that at a rate of \SIrate{600} the working voltage of the chamber is increased by a thousand \si{V} while the efficiency is reduced to approximately 80\%, although the result still is consistent with an efficiency better than 90\% due to the large error on the measurement. Moreover, it is likely that the rates obtained through fitting on normalized values is underestimated. Indeed, monitoring the current in the 3 gaps composing a CMS endcap RPC (Figure~\ref{fig:EndcapRPC}) while knowing the rate, the charge deposition per avalanche $q_\gamma$ can be computed.
	
	\begin{figure}[H]
    	\begin{subfigure}{0.69\linewidth}
			\centering
    		\includegraphics[height = 5cm]{fig/chapt5/Endcap-3D.pdf}
        	\caption{\label{fig:EndcapRPC:A}}
    	\end{subfigure}
    	\begin{subfigure}{0.29\linewidth}
			\centering
    		\includegraphics[height = 5cm]{fig/chapt5/Endcap-side.pdf}\\
        	\caption{\label{fig:EndcapRPC:B}}
    	\end{subfigure}
		\caption{\label{fig:EndcapRPC} Presentation of a double-gap endcap RPC with its 3 RPC gaps. Due to the partitioniing of the read-out strips into 3 rolls, the TOP layer of gap is divided into 2 gaps: TOP NARROW (TN) and TOP WIDE (TW). The BOTTOM (B) only consists in 1 gap.}
	\end{figure}
	
	A charge is expressed in \si{C} which is consistent with a current density, expressed in \si{A/cm^2}, divided by a rate per unit area, expressed in \si{Hz/cm^2}. The current driven by the RPC is assumed to be due to the irradiation, hence, to the avalanches developing in the gas volume due to the photons of the Cesium source. On the other hand, the rate is supposed to be a measure of the number of photons interacting with the detector. This way, it comes that the charge deposition per avalanche is expressed like $q_\gamma = J_{mon}/R_\gamma$, $J_{mon}$ being the monitored current density and $R_\gamma$ the measured $\gamma$ rate. The current density is computed as the sum of the current density measured on the top gap layer and of which measured in the bottom gap layer, $J_{mon} = (I_{mon}^{TW}+I_{mon}^{TW})/(A_{TW}+A_{TN}) + I_{mon}^B/A_B$, $A_{B,TN,TW}$ being the active area and $I_{mon}^{B,TN,TW}$ the monitored currents of the gaps. According to Figure~\ref{fig:Charge}, the charge deposition per avalanche consistently converges to a value of the order of \SI{50}{pC} for each absorber setting which corresponds to a value more than twice greater than what reported in litterature for CMS detectors~\cite{PUGLIESE2002,PUGLIESE2003} indicating that the rates could have been wrongly evaluated during the short study performed at GIF. An increase of the $\gamma$ rate by a factor 2 would be consistent with the expected rates calculated in Table~\ref{tab:extra2014}, assuming the sensitivity to $\gamma$ to be of the order of \Sci{2}{-3}.
	
	\begin{figure}[H]
    	\begin{subfigure}{0.5\linewidth}
			\centering
    		\includegraphics[width = 0.7\plotwidth]{fig/chapt5/Current-Density.pdf}
        	\caption{\label{fig:Charge:A}}
    	\end{subfigure}
    	\begin{subfigure}{0.5\linewidth}
			\centering
    		\includegraphics[width = 0.7\plotwidth]{fig/chapt5/Charge-per-gamma.pdf}
        	\caption{\label{fig:Charge:B}}
    	\end{subfigure}
		\caption{\label{fig:Charge} Current density and charge deposition per gamma avalanche, defined as the current density normalized to the measured rate taken from Figure~\ref{fig:GIFRate:A} as a function of the effective high voltage in chamber \texttt{RE-4-2-BARC-161} measured at GIF with Source ON using different absorber settings: ABS 5 (green), ABS 2 (blue) and ABS 1 (yellow).}
	\end{figure}
	
	Overall, working at GIF has been a rewarding experience as it offered CMS RPC R\&D team the possibility to start developping the necessary skills and tools that would become the core of GIF++ experiment. The quality of the results can be argued both due to the little robustness of the experimental setup and the lack of available statistics to draw conclusions from, bringing large errors on the final result. The confrontation of the data to known results pointed to a failure in correctly measuring the $\gamma$ rate at working voltage and, hence, to an overestimation of the charge per avalanche and of the drift of working voltage with increasing rate. Nevertheless, the prototypes of DAQ and offline analysis tools proved to be reliable.

\section{Longevity tests at \acs{GIF++}}
\label{chapt5:sec:GIFpptests}

    Longevity studies imply a monitoring of the performance of the detectors probed using a high intensity muon beam in a irradiated environment by periodically measuring their rate capability, the dark current running through them and the bulk resistivity of the Bakelite composing their electrodes. GIF++, with its very intense $^{137}$Cs source, provides the perfect environment to perform such kind of tests. Assuming a maximum acceleration factor of 3, it is expected to accumulate the equivalent charge in 1.7 years.\\
    As the maximum background is found in the endcap, the choice naturally was made to focus the GIF++ longevity studies on endcap chambers. Most of the RPC system was installed in 2007. Nevertheless, the large chambers in the fourth endcap (RE4/2 and RE4/3) have been installed during LS1 in 2014. The Bakelite of these two different productions having different properties, four spare chambers of the present system were selected, two RE2,3/2 spares and two RE4/2 spares. Having two chambers of each type allows to always keep one of them non irradiated as reference, the performance evolution of the irradiated chamber being then compared through time to the performance of the non irradiated one.\\
    The performance of the detectors under different level of irradiation is measured periodically during dedicated test beam periods using the H4 muon beam. In between these test beam periods, the two RE2,3/2 and RE4/2 chambers selected for this study are irradiated by the $^{137}$Cs source in order to accumulate charge and the gamma background is monitored, as well as the currents. The two remaining chambers are kept non-irradiated as reference detectors. Due to the limited gas flow in GIF++, the RE4 chamber remained non-irradiated until end of November 2016 where a new mass flow controller has been installed allowing for bigger volumes of gas to flow in the system.\\
     Figures~\ref{Fig:Eff-vs-Rate} and \ref{Fig:WP-vs-Rate} give us for different test beam periods, and thus for increasing integrated charge through time, a comparison of the maximum efficiency, obtained using a sigmoid-like function, and of the working point of both irradiated and non irradiated chambers~\cite{ABBRESCIA2005}. No aging is yet to see from this data, the shifts in $\gamma$ rate per unit area  in between irradiated and non irradiated detectors and RE2 and RE4 types being easily explained by a difference of sensitivity due to the various Bakelite resistivities of the HPL electrodes used for the electrode production.\\
     Collecting performance data at each test beam period allows us to extrapolate the maximum efficiency for a background hit rate of \SIflux{300} corresponding to the expected HL-LHC conditions. Aging effects could emerge from a loss of efficiency with increasing integrated charge over time, thus Figure~\ref{Fig:Eff-vs-Qint} helps us understand such degradation of the performance of irradiated detectors in comparison with non irradiated ones. The final answer for an eventual loss of efficiency is given in Figure~\ref{Fig:Eff-Bef-Aft} by comparing for both irradiated and non irradiated detectors the efficiency sigmoids before and after the longevity study. Moreover, to complete the performance information, the Bakelite resistivity is regularly measured thanks to $Ag$ scans (Figure~\ref{Fig:Res-vs-Qint}) and the noise rate is monitored weekly during irradiation periods (Figure~\ref{Fig:Noise-vs-Qint}). At the end of 2016, no signs of aging were observed and further investigation is needed to get closer to the final integrated charge requirements proposed for the longevity study of the present CMS RPC sub-system.\\
    
    \begin{figure}[H]
    	\begin{subfigure}{0.5\linewidth}
			\centering
    		\includegraphics[width=\linewidth]{fig/chapt5/Eff-vs-Rate-RE2.png}\\
        	\caption{\label{Fig:Eff-vs-Rate:RE2}}
    	\end{subfigure}
    	\begin{subfigure}{0.5\linewidth}
			\centering
    		\includegraphics[width=\linewidth]{fig/chapt5/Eff-vs-Rate-RE4.png}\\
        	\caption{\label{Fig:Eff-vs-Rate:RE4}}
    	\end{subfigure}
        \caption{\label{Fig:Eff-vs-Rate} Evolution of the maximum efficiency for RE2 (\ref{Fig:Eff-vs-Rate:RE2}) and RE4 (\ref{Fig:Eff-vs-Rate:RE4}) chambers with increasing extrapolated $\gamma$ rate per unit area at working point. Both irradiated (blue) and non irradiated (red) chambers are shown.}
    \end{figure}
    
    \begin{figure}[H]
    	\begin{subfigure}{0.5\linewidth}
			\centering
    		\includegraphics[width=\linewidth]{fig/chapt5/WP-vs-Rate-RE2.png}\\
        	\caption{\label{Fig:WP-vs-Rate:RE2}}
    	\end{subfigure}
    	\begin{subfigure}{0.5\linewidth}
			\centering
    		\includegraphics[width=\linewidth]{fig/chapt5/WP-vs-Rate-RE4.png}\\
        	\caption{\label{Fig:WP-vs-Rate:RE4}}
    	\end{subfigure}
        \caption{\label{Fig:WP-vs-Rate} Evolution of the working point for RE2 (\ref{Fig:WP-vs-Rate:RE2}) and RE4 (\ref{Fig:WP-vs-Rate:RE4}) with increasing extrapolated $\gamma$ rate per unit area at working point. Both irradiated (blue) and non irradiated (red) chambers are shown.}
    \end{figure}
    
    \begin{figure}[H]
    	\begin{subfigure}{0.5\linewidth}
			\centering
    		\includegraphics[width=\linewidth]{fig/chapt5/Eff-vs-Qint-RE2.png}\\
        	\caption{\label{Fig:Eff-vs-Qint:RE2}}
    	\end{subfigure}
    	\begin{subfigure}{0.5\linewidth}
			\centering
    		\includegraphics[width=\linewidth]{fig/chapt5/Eff-vs-Qint-RE4.png}\\
        	\caption{\label{Fig:Eff-vs-Qint:RE4}}
    	\end{subfigure}
        \caption{\label{Fig:Eff-vs-Qint} Evolution of the maximum efficiency at HL-LHC conditions, i.e. a background hit rate per unit area of \SI{300}{Hz/cm^2}, with increasing integrated charge for RE2 (\ref{Fig:Eff-vs-Qint:RE2}) and RE4 (\ref{Fig:Eff-vs-Qint:RE4}) detectors. Both irradiated (blue) and non irradiated (red) chambers are shown. The integrated charge for non irradiated detectors is recorded during test beam periods and stays small with respect to the charge accumulated in irradiated chambers.}
    \end{figure}
    
    \begin{figure}[H]
    	\begin{subfigure}{0.5\linewidth}
			\centering
    		\includegraphics[width=\linewidth]{fig/CMSlogo.png}\\
        	\caption{\label{Fig:Eff-Bef-Af:RE2}}
    	\end{subfigure}
    	\begin{subfigure}{0.5\linewidth}
			\centering
    		\includegraphics[width=\linewidth]{fig/CMSlogo.png}\\
        	\caption{\label{Fig:Eff-Bef-Af:RE4}}
    	\end{subfigure}
        \caption{\label{Fig:Eff-Bef-Aft} Comparison of the efficiency sigmoid before (triangles) and after (circles) irradiation for RE2 (\ref{Fig:Eff-Bef-Af:RE2}) and RE4 (\ref{Fig:Eff-Bef-Af:RE4}) detectors. Both irradiated (blue) and non irradiated (red) chambers are shown.}
    \end{figure}
    
    \begin{figure}[H]
    	\begin{subfigure}{0.5\linewidth}
			\centering
    		\includegraphics[width=\linewidth]{fig/chapt5/Res-vs-Qint-RE2.png}\\
        	\caption{\label{Fig:Res-vs-Qint:RE2}}
    	\end{subfigure}
    	\begin{subfigure}{0.5\linewidth}
			\centering
    		\includegraphics[width=\linewidth]{fig/chapt5/Res-vs-Qint-RE4.png}\\
        	\caption{\label{Fig:Res-vs-Qint:RE4}}
    	\end{subfigure}
        \caption{\label{Fig:Res-vs-Qint} Evolution of the Bakelite resistivity for RE2 (\ref{Fig:Res-vs-Qint:RE2}) and RE4 (\ref{Fig:Res-vs-Qint:RE4}) detectors. Both irradiated (blue) and non irradiated (red) chambers are shown.}
    \end{figure}
    
    \begin{figure}[H]
		\centering
    	\includegraphics[width=0.5\linewidth]{fig/chapt5/Noise-vs-Qint-RE2.png}\\
        \caption{\label{Fig:Noise-vs-Qint} Evolution of the noise rate per unit area for the irradiated chamber RE2-2-BARC-9 only.}
    \end{figure}

	\subsection{Description of the \acl{DAQ}}
	\label{chapt5:ssec:GIF++DAQ}

	For the longevity studies, four spare chambers of the present system are used. Two spare RPCs of the RE2,3 stations as well as two spare RPCs from the new RE4 stations have been mounted in a Trolley. Six RE4 gaps are also placed in the trolley. The trolley is placed inside the GIF++ in the upstream region of the bunker, taking the cesium source as a reference. The trolley is oriented for the detection surface of the chambers to be orthogonal to the beam line. The system can be moved along the orthogonal plane in order to have the beam in all $\eta$-partitions. For the aging the trolley is moved outside the beam line and is placed in a distance of \SI{5.2}{m} to the source, which irradiates the bunker using an attenuation filter of 2.2 which corresponds to a fluence of \Ord{7}\si{gamma/cm\squared}.
	
	During GIF++ operation, the data collected can be divided into different categories as several parameters are monitored in addition to the usual RPC performance data. On one hand, to know the performance of a chamber, it is need to measure its efficiency and to know the background conditions in which it is operated. To do this, the hit signals from the chamber are recorded and stored in a ROOT file via a DAQ software. On the other hand, it is also very important to monitor parameters such as environmental pressure and temperature, gas temperature and humidity, RPC HV, LV, and currents, or even source and beam status. This is done through the GIF++ web DCS that stores this information in a database.
	            
	Two different types of tests are conducted on RPCs via the DAQ. Indeed, the performance of the detectors is measured periodically during dedicated test beam periods using the H4 muon beam. In between these test beam periods, when the beam is not available, the chambers are irradiated by the $^{137}$Cs in order to accumulate deposited charge and the gamma background is measured.
	
	RPCs under test are connected through LVDS cables to V1190A TDC modules manufactured by CAEN. These modules, located in the rack area outside of the bunker, get the logic signals sent by the chambers and save them into their buffers. Due to the limited size of the buffers, the collected data is regularly erased and replaced. A trigger signal is needed for the TDC modules to send the useful data to the DAQ computer via a V1718 CAEN USB communication module.
	
	In the case of performance test, the trigger signal used for data acquisition is generated by the coincidence of three scintillators. A first one is placed upstream outside of the bunker, a second one is placed downstream outside of the bunker, while a third one is placed in front of the trolley, close by the chambers. Every time a trigger is sent to the TDCs, i.e. every time a muon is detected, knowing the time delay in between the trigger and the RPC signals, signals located in the right time window are extracted from the buffers and saved for later analysis. Signals are taken in a time window of \SI{400}{ns} centered on the muon peak (here we could show a time spectrum). On the other hand, in the case of background rate measurement, the trigger signal needs to be "random" not to measure muons but to look at gamma background. A trigger pulse is continuously generated at a rate of \SI{300}{Hz} using a dual timer. To integrate an as great as possible time, all signals contained within a time window of 10us prior to the random trigger signal are extracted form the buffers and saved for further analysis (here another time spectrum to illustrate could be useful, maybe even place both spectrum together as a single Figure).
	
	The signals sent to the TDCs correspond to hit collections in the RPCs. When a particle hits a RPC, it induce a signal in the pickup strips of the RPC readout. If this signal is higher than the detection threshold, a LVDS signal is sent to the TDCs. The data is then organised into 4 branches keeping track of the event number, the hit multiplicity for the whole setup, and the time and channel profile of the hits in the TDCs.
	
	\subsection{RPC current, environmental and operation parameter monitoring}
	\label{chapt5:ssec:GIF++DCS}
            
	In order to take into account the variation of pressure and temperature between different data taking periods the applied voltage is corrected following the relationship :

	\begin{equation}
		HVeff = HVapp\times\left(0.2 + 0.8\cdot\frac{P_0}{P}\times\frac{T}{T_0}\right)
	\end{equation}
	
	where $T_0$ (=\SI{293}{K}) and $P_0$ (=\SI{990}{mbar}) are the reference values.

	\subsection{Measurement procedure}
	\label{chapt5:ssec:GIF++Proc}

	Insert a short description of the online tools (DAQ, DCS, DQM).\\
	Insert a short description of the offline tools : tracking and efficiency algorithm.\\
	Identify long term aging effects we are monitoring the rates per strip.
	
	\subsection{Longevity studies results}
	\label{chapt5:ssec:GIF++Results}

\clearpage{\pagestyle{empty}\cleardoublepage}