% Header
\renewcommand\evenpagerightmark{{\scshape\small Chapter 2}}
\renewcommand\oddpageleftmark{{\scshape\small Investigating the \si{TeV} scale}}

\renewcommand{\bibname}{References}

\hyphenation{}

\chapter[Investigating the \si{TeV} scale]%
{Investigating the \si{TeV} scale}
\label{chapt:2}

	\vfill
	
	{\Large\textit{,,We may regard the present state of the universe as the effect of the past and the cause of the future. An intellect which at any given moment knew all of the forces that animate nature and the mutual positions of the beings that compose it, if this intellect were vast enough to submit the data to analysis, could condense into a single formula the movement of the greatest bodies of the universe and that of the lightest atom; for such an intellect nothing could be uncertain and the future just like the past would be present before its eyes.''}}\\
	
	{\normalsize\raggedleft - Pierre Simon de Laplace, \textit{A Philosophical Essay on Probabilities}, 1814}
	
	\vfill
	
\newpage

	Throughout history, physics experiment became more and more powerful in order to investigate finer details of nature and helped understanding the elementary blocks of matter and the fondamental interactions that bond them in the microscopic world. Nowadays, the \acf{SM} of particle physics is the most accurate theory designed to explain the behaviour of particles and was able to make very precise predictions that are constantly verified, although some hints of new physics are visible as bricks are still missing to have a global comprehension of the Universe.\\
	To highlight the limits of the SM and test the different alternative theories, ever more powerful machines are needed. This is in this context that the \acf{LHC} has been thought and built to accelerate and collide particles at energies exceeding anything that had been done before. Higher collision energies and high pile-up imply the use of enormous detectors to measure the properties of the interaction products. The \acf{CMS} is a multipurpose experiment that have been designed to study the proton-proton collisions of the LHC and give answers on various high energy physics scenari. Nevertheless, the luminosity delivered by the collider will in the future be increased to levels beyond the original plans to improve its discovery potential giving no choice to experiments such as CMS to upgrade their technologies to cope with the increased radiation levels and detection rates.

\section{The Standard Model of Particle Physics}
\label{chapt2:sec:SM}

	In this early \St{21} century it is now widely accepted that matter is made of elementary blocks referred to as \textit{elementary particles}. The physics theory that classifies and describes the best the behaviour and interaction of such elementary particles is the so called\acl{SM} that formalizes 3 of the 4 fondamental interactions (electromagnetic, weak and strong interactions). It's development took place during the \Th{20} century thanks to a strong collaboration in between the theoretical and experimental physicists.\\

	\subsection{A brief history of particle physics}
	\label{chapt2:ssec:history}
	
	The idea that nature is composed of elementary bricks, called \textit{atomism}, is not contemporary as it was already discussed by Indian or Greek philosophers during antiquity. In Greece, atomism has been rejected by Aristotelianism as the existance of \textit{atoms} would imply the existance of a void that would violate the physical principles of Aristotle philosophy. Aristotelianism has been considered as a reference in the european area until the \Th{15} century and the italian \textit{Rinascimento} where antic text and history started to be more deeply studied. The re-discovery of Platon's philosophy would allow to open the door to alternative theories and give a new approach to natural sciences where experimentation would become central. A new era of knowledge was starting. By the begining of the \Th{17} century, atomism was re-discovered by philosophers and the very first attempt to estimate an \textit{atom} size would be provided by Magnetus in 1646. Although his \textit{atoms} correspond to what would nowadays be called \textit{molecules}, Magnetus achieved feats by calculating that the number of molecules in a grain of incense would be of the order of $10^{18}$ simply by considering the time necessary to smell it everywhere in a large church after the stick was lit on. It is now known that this number only falls short by 1 order of magnitude.\\
	
	An alternative philosophy to atomism popularized by Descartes was corpuscularianism. Built on ever divible corpuscles, contrary to atoms, it's principles would be mainly used by alchemists like Newton who would later develop a corpuscular theory of light. Boyle would combine together ideas of both atomism or corpuscularianism leading to mechanical philosophy. The \Th{18} century have seen the development of engineering providing philosophical thought experiments with repeatable demonstration and a new point of view to explain the composition of matter and Lavoisier would greatly contribute to chemistry and atomism by publishing in 1789 a list of 33 chemical elements corresponding to what is now called \textit{atoms}. In the early \Th{19} century Dalton would summarize the knowledge on composition of matter and Fraunhofer would invent the spectrometer and discover the spectral lines. The rise of atomic physics, chemistry and mathematical formalism would unravel the different atomic elements and ultimately, the \Th{20} century would see the very first sub-atomic particles.\\
	
	The negative \textit{electron} would be the first to be discovered in 1897 by Thompson after 3 decades of research on cathode rays and in 1900, Becquerel would show the \textit{beta rays} emitted by radium had the same properties pointing to eletrons as a constituant of atoms. In 1909, Rutherford and Royds showed that \textit{alpha} particles could combine with 2 electrons to form a $^4$He after they already constrained the atom structure in 1907 through the gold foil experiment that highlighted atoms where mainly empty with nearly all its mass contained into a tiny positively charged \textit{nucleus}. The link in between atomic number and number of positive chages contained into the nuclei would fast be understood and the different kind of element transmutation appeared to be purely nuclear processes making clear that the electromagnetic nature of chemical transformation could not possibly change nuclei. Thus a new branch in physics appeared to study nuclei exclusively: the nuclear physics which would in turn give birth to quantum physics.\\
	
	By studying alpha emission, Rutherford reported in 1919 the very first nuclear reaction leading to the discovery that the hydrogen nucleus was composed of a single positively charged particle that was then baptised \textit{proton}. This idea came from 1815 Prout's hypothesis proposing that all atoms are composed of \textit{"protolytes"} (i.e. hydrogen atoms). By using scintillation detectors, Rutherford could highlight typical hydrogen nuclei signature and understand that the impact of alpha particles with nitrogen would knock out an hydrogen nucleus and produce an oxygen 17, as explicited in Formula~\ref{eq:nuclear} and would then postulate that protons are building bricks of all elements.\\
	
	\begin{equation}
		\label{eq:nuclear}
		^{14}N + \alpha \rightarrow ^{17}O + p
	\end{equation}
	
	With this assumption and the discovery of isotopes together with Aston, elements with identical atomic number but different masses, Rutherford would propose that all elements' nuclei but hydrogen's are composed of both charged particles, protons, and of chargeless particles, which he called \textit{neutrons}, and that these neutral particles would help maintaining nuclei as one, as charged protons were likely to electrostatically repulse each other, and introduced the idea of a new force, a \textit{nuclear} force. Though the first idea concerning neutrons was a bond state of protons and electrons as it was known that the beta decay, emitting electrons, was taking place in the nucleus, it was then showed that such a model would hardly be possible due to Heisenberg's uncertainty principle and by the recently measured \textit{spin} of both protons and electrons. The spin, discovered through the study of the emission spectrum of alkali metals, would be understood as a "two-valued quantum degree of freedom" and formalized by Pauli and extended by Dirac, to take the relativist case into account. Measured to be $\frac{1}{2} \hbar$ for both, it was impossible to arrange an odd number of half integer spins and obtain a global nucleus spin that would be integer. Finally, in 1932, following the discovery of a new neutral radiation, Chadwick could discover the neutron as an uncharged particle with a mass similar to that of the proton.\\
	
	\subsection{Construction and test of the model}
	\label{chapt2:ssec:model}
	
	\subsection{Investigating the TeV scale}
	\label{chapt2:ssec:TeV}

\section{The \acl{LHC} \& the \acl{CMS}}
\label{chapt2:sec:LHC-CMS}

	Throughout its history, CERN has played a leading role in high energy particle physics. Large regional facilities such as CERN were thought after the second world war in an attempt to increase international scientific collaboration and allows scientists to share the forever increasing costs of experiment facilties required due to the need for increasing the energy in the center of mass to deeper probe matter. The construction of the first accelerators at the end of the 50s, the \acf{SC} and the \acf{PS}, was directly followed by the first observation of antinuclei in 1965~\cite{MASSAM1965}. Strong from the experience of the \acf{ISR}, the very first proton-proton collider that showed hints that protons are not elementary particles, the \acf{SPS} was built in the 70s to investigate the structure of protons, the preference for matter over antimatter, the state of matter in the early universe or exotic particles, and lead to the discovery in 1983 of the W and Z bosons~\cite{UA1W1983,UA2W1983,UA1Z1983,UA2Z1983}. These newly discovered particles and the electroweak intereaction would then be studied in details by the \acf{LEP} collider that will help to prove in 1989 that there only are three generations of elementary particles~\cite{ALEPH1989}. The LEP would then be dismantled in 2000 to allow for the LHC to be constructed in the existing tunnel.

	\subsection{LHC, the most powerful particle accelerator}
	\label{chapt2:ssec:LHC}
	
	The LHC has always been considered as an option to the future of CERN. At the moment of the construction of the LEP beneath the border between France and Switzerland, the tunnel was built in order to accomodate what would be a \acl{LHC} with a dipole field of \SI{10}{T} and a beam energy in between 8 and \SI{9}{TeV}~\cite{ANNUALREPORT1984} directly followed in 1985 with the creation of a 'Working Group on the Scientific and Technological Future of CERN' to investigate such a collider~\cite{ANNUALREPORT1985}. The decision was finally taken almost 10 years later, in 1994, to construct the LHC in the LEP tunnel~\cite{ANNUALREPORT1994} and the approval of the 4 main experiments that would take place at the 4 interaction points would come in 1997~\cite{ANNUALREPORT1997} and 1998~\cite{ANNUALREPORT1998}:
	
	\begin{itemize}
		\item[•] ALICE~\cite{ALICELOI} has been designed in the purpose of studying quark-gluon plasma that is believed to have been a state of matter that existed in the very first moment of the universe.
		\item[•] ATLAS~\cite{ATLASLOI} and CMS~\cite{CMSLOI} are general purpose experiements that have been designed with the goal of continuing the exploration of the Standard Model and investigate new physics.
		\item[•] LHCb~\cite{LHCBLOI} has been designed to investigate the preference of matter over antimatter in the universe through the CP violation.
	\end{itemize}
	
	These large scale experiments, as well as the full CERN accelerator complex, are displayed on Figure~\ref{fig:CERNComplex}.

	\begin{figure}[H]
		\centering
		\hspace*{-0.1\linewidth}
		\includegraphics[width=1.2\linewidth]{fig/chapt2/CERN_Accelerator_Complex.png}
		\caption{\label{fig:CERNComplex} CERN accelerator complex.}
	\end{figure}
	
	The LHC is a \SI{27}{km} long hadron collider and the most powerful accelerator used for particle physics since 2008~\cite{LHC2008}. The LHC was originally designed to collide protons at a center-of-mass energy of \SI{14}{TeV} and luminosity of $10^{34}$ \si{cm^{-2}s^{-1}}, as well as $Pb$ ions at a center-of-mass energy of \SI{2.8}{TeV/A} with a peak luminosity of $10^{27}$ \si{cm^{-2}s^{-1}}. Run 1 of LHC, when the center-of-mass energy only was half of the nominal LHC energy, was enough for both CMS and ATLAS to discover the Higgs boson~\cite{HIGGS2015} and for LHCb to discover pentaquarks~\cite{PENTAQUARK2015} and confirm the existance of tetraquarks~\cite{TETRAQUARK2017}. Nevertheless, after the \acf{LS3} (2024-2026), the accelerator will be in the so called \acf{HL-LHC} configuration~\cite{HLLHC2017}, increasing its instantaneous luminosity to $10^{35}$ \si{cm^{-2}s^{-1}} for $pp$ collisions and to $4.5\times 10^{27}$ \si{cm^{-2}s^{-1}}, boosting the discovery potential of the LHC.
	
		\subsubsection{Particle acceleration}
		\label{chapt2:sssec:acceleration}
	
	The LHC is the last of a long series of accelerating devices. Before being accelerated by the LHC, the particles need to pass through different acceleration stages. All these acceleration stages are visible on Figure~\ref{fig:CERNComplex} and pictures of the accelerators are showed in Figure~\ref{fig:CERNAccelerators}.\\
	
	The story of accelerated protons at CERN starts with a bottle of hydrogen gas injected into the source chamber of the linear particle accelerator \textit{LINAC 2} 2 in which a strong electric field strips the electron off the hygroden molecules only to keep their nuclei, the protons. The cylindrical conductors, alternatively positively or negatively charged by radiofrequency cavities, accelerate protons by pushing them from behing and pulling them from the front and ultimately give them an energy of \SI{50}{MeV}, increasing their mass by 5\% in the process.\\
	
	When exiting the LINAC 2, the protons are divided into 4 bunches and injected into the 4 superimposed synchrotron rings of the \textit{Booster} where they are then accelerated to reach an energy of \SI{1.4}{GeV} before being injected into the \textit{PS}. Before the Booster was operational in 1972, the protons were directly injected into the PS from the LINAC 2 but the low injection energy limited the amount of protons that could be accelerated at once by the PS. With the Booster, the PS accepts approximately 100 times more particles.\\
	
	The 4 proton bunches are thus sent as one to the PS where their energy eventually reaches \SI{26}{GeV}. Since the 70s, the main goal of this \SI{628}{m} circumference synchrotron has been to supply other machines with accelerated particles. Nowadays, not only the PS accelerates protons, it also accelerates heavy ions from the \textit{\acf{LEIR}}. Indeed, the LHC experiments are not only designed to study $pp$-collinsions but also $Pb$-collisions. Lead is first injected into the dedicated linear collider \textit{LINAC 3}, that accelerate the ions using the same principle than LINAC 2. Electrons are striped off the lead ions all along the acceleration process and eventually, only bare nuclei are injected in the LEIR whose goal is to transform the long ion pulses received into short dense bunches for LHC. Ions injected and stored in the PS were aceelerated by the LEIR from \SI{4.2}{MeV} to \SI{72}{MeV}.\\
	
	Directly following the PS, is finally the last acceleration stage before the LHC, the \SI{7}{km} long \textit{SPS}. The SPS accelerates the protons to \SI{450}{GeV} and inject proton in both LHC accelerator rings that will increase their energy up to \SI{7}{TeV}. When the LHC runs with heavy lead ions for ALICE and LHCb, ions are injected and accelerated to reach the energy of \SI{2.8}{TeV/A}.

	\begin{figure}[H]
		\centering
		\includegraphics[width=0.5\linewidth]{fig/chapt2/CERN-accelerators.jpg}
		\caption{\label{fig:CERNAccelerators} Pictures of the different accelerators. From top to bottom: first the LINAC 2 and the $Pb$ source of LINAC 3. Then the Booster and the LEIR. Finally, the PS, the SPS and the LHC.}
	\end{figure}
	
	The LHC beams are not continuous and are rather organised in bunch of paticles. When in $pp$-collision mode, the beams are composed of 2808 bunches of $1.15 \times 10^{11}$ protons separated by \SI{25}{ns}. When in $Pb$ collision mode, the 592 $Pb$ bunches are on the contrary composed of $2.2 \times 10^8$ ions separated by \SI{100}{ns}. The two parrallel proton beams of the LHC are contained in a single twin-bore magnet due to the space restriction in the LEP tunnel. Indeed, building 2 completely separate accelerator rings next to each other was impossible. The dipoles of the 1232 twin-bore magnets are showed in Figure~\ref{fig:LHCDipole} alongside the magnetic field generated along the dipole section to accelerate the particles. The dipoles generate a nominal field of \SI{8.33}{T}, needed to give protons and lead nucleons their nominal energy. Some 392 quadrupoles, presented in Figure~\ref{fig:LHCQuadrupole}, are also used to focus to the beams, as well as other multipoles to correct smaller imperfections.
	
	\begin{figure}[H]
		\begin{subfigure}{0.5\linewidth}
			\centering
			\includegraphics[height = 4cm]{fig/chapt2/LHC-dipole.png}
			\caption{\label{fig:LHCDipole:A}}
		\end{subfigure}
		\begin{subfigure}{0.5\linewidth}
			\centering
			\includegraphics[height = 4cm]{fig/chapt2/LHC-dipole-field.jpg}
			\caption{\label{fig:LHCDipole:B}}
		\end{subfigure}
		\caption{\label{fig:LHCDipole} Figure~\ref{fig:LHCDipole:A}: schematics of the LHC cryodipoles. 1: Superconducting Coils, 2: Beam pipe, 3: Heat exchanger Pipe, 4: Helium-II Vessel, 5: Superconducting Bus-bar, 6: Iron Yoke, 7: Non-Magnetic Collars, 8: Vacuum Vessel, 9: Radiation Screen, 10: Thermal Shield, 11: Auxiliary Bus-bar Tube, 12: Instrumentation Feed Throughs, 13: Protection Diode, 14: Quadrupole Bus-bars, 15: Spool Piece Bus-bars. Figure~\ref{fig:LHCDipole:B}: magnetic field and resulting motion force applied on the beam particles.}
	\end{figure}
	
	\begin{figure}[H]
		\begin{subfigure}{0.5\linewidth}
			\centering
			\includegraphics[height = 4cm]{fig/chapt2/LHC-quadrupole.jpg}
			\caption{\label{fig:LHCQuadrupole:A}}
		\end{subfigure}
		\begin{subfigure}{0.5\linewidth}
			\centering
			\includegraphics[height = 4cm]{fig/chapt2/LHC-quadrupole-field.png}
			\caption{\label{fig:LHCQuadrupole:B}}
		\end{subfigure}
		\caption{\label{fig:LHCQuadrupole} Figure~\ref{fig:LHCQuadrupole:A}: picture of the LHC quadrupoles. Figure~\ref{fig:LHCQuadrupole:B}: magnetic fields and resulting focussing force applied on the beam by 2 consecutive quadrupoles.}
	\end{figure}
	
		\subsubsection{LHC discoveries and LHC physics program}
		\label{chapt2:sssec:discovery}
		
	The very first proton beam successfully circulated in the LHC in September 2008 directly followed by an incident leading to mechanical damage that would delay the LHC program for a year until November 2009.
	
		\subsubsection{\acl{HL-LHC}}
		\label{chapt2:sssec:HL-LHC}

	\subsection{CMS, a multipurpose experiment}
	\label{chapt2:ssec:CMS}

\section{Muon Phase-II Upgrade}
\label{chapt2:sec:phase-2}

After the more than two years lasting \acf{LS1}, the \acf{LHC} delivered its very first Run-II proton-proton collisions early 2015. LS1 gave the opportunity to the LHC and to the its experiments to undergo upgrades. The accelerator is now providing collisions at center-of-mass energy of \SI{13}{TeV} and bunch crossing rate of \SI{40}{MHz}, with a peak luminosity exceeding its design value. During the first and upcoming second LHC Long Shutdown, the \acf{CMS} detector is also undergoing a number of upgrades to maintain a high system performance~\cite{MUONTDR}.

From the LHC Phase-2 or \acf{HL-LHC} period onwards, i.e. past the \acf{LS3}, the performance degradation due to integrated radiation as well as the average number of inelastic collisions per bunch crossing, or pileup, will rise substantially and become a major challenge for the LHC experiments, like CMS that are forced to address an upgrade program for Phase-II~\cite{PHASEIITP}. Simulations of the expected distribution of absorbed dose in the CMS detector under HL-LHC conditions, show in figure~\ref{fig:Dose} that detectors placed close to the beamline will have to withstand high irradiation, the radiation dose being of the order of a few tens of\si{Gy}.

\begin{figure}[H]
	\centering
	\includegraphics[width=0.7\textwidth]{fig/chapt2/HL-LHC-Dose.png}
	\caption{\label{fig:Dose} Absorbed dose in the CMS cavern after an integrated luminosity of \SI{3000}{\femto\per\barn}. R is the transverse distance from the beamline and Z is the distance along the beamline from the Interaction Point at Z=0.}
\end{figure}

The measurement of small production cross-section and/or decay branching ratio processes, such as the Higgs boson coupling to charge leptons or the $B_s \longrightarrow \mu^+\mu^-$ decay, is of major interest and specific upgrades in the forward regions of the detector will be required to maximize the physics acceptance on the largest possible solid angle. To ensure proper trigger performance within the present coverage, the muon system will be completed with new chambers. In figure~\ref{fig:Quadrant} one can see that the existing \acfp{CSC} will be completed by \acfp{GEM} and \acfp{RPC} in the pseudo-rapidity region $1.6<\vert\eta\vert<2.4$ to complete its redundancy as originally scheduled in the CMS Technical Proposal~\cite{CMSTP}.

\begin{figure}[H]
	\centering
	\includegraphics[width=0.7\textwidth]{fig/chapt2/MuonUpgrade-Plans.jpg}
	\caption{\label{fig:Quadrant} A quadrant of the muon system, showing DTs (yellow), RPCs (light blue), and CSCs (green). The locations of new forward muon detectors for Phase-II are contained within the dashed box and indicated in red for GEM stations (ME0, GE1/1, and GE2/1) and dark blue for improved RPC (iRPC) stations (RE3/1 and RE4/1).}
\end{figure}

RPCs are used by the CMS first level trigger for their good timing performances. Indeed, a very good bunch crossing identification can be obtained with the present CMS RPC system, given their fast response of the order of \SI{1}{ns}. In order to contribute to the precision of muon momentum measurements, muon chambers should have a spatial resolution less or comparable to the contribution of multiple scattering~\cite{MUONTDR}. Most of the plausible physics is covered only considering muons with $p_T<$\SI{100}{GeV} thus, in order to match CMS requirements, a spatial resolution of $\mathcal{O}$(few $\mathrm{mm}$) the proposed new RPC stations, as shown by the simulation in figure~\ref{fig:MultiScat}. According to preliminary designs, RE3/1 and RE4/1 readout pitch will be comprised between 3 and \SI{6}{mm} and 5 $\eta$-partitions could be considered.

\begin{figure}[H]
	\centering
	\includegraphics[width=0.6\textwidth]{fig/chapt2/MS_allstations.pdf}
	\caption{\label{fig:MultiScat}  RMS of the multiple scattering displacement as a function of muon $p_T$ for the  proposed forward muon stations. All of the electromagnetic processes such as bremsstrahlung and magnetic field effect are included in the simulation.}
\end{figure}

\clearpage{\pagestyle{empty}\cleardoublepage}
