% Header
\renewcommand\evenpagerightmark{{\scshape\small Chapter 1}}
\renewcommand\oddpageleftmark{{\scshape\small Introduction}}

\renewcommand{\bibname}{References}

\hyphenation{}

\chapter[Introduction]%
{Introduction}
\label{chap:intro}

Grasping an understanding of the world in which they are leaving in has always been part of humans lives. Beyond the metaphysical questioning of our origins and purpose, curiosity has brought mankind to question its surrondings. Following the philosophy of ancient greeks and indians came the development of sciences as the systematic experimentation aimed at testing hypothesis and reproducing results obtained by fellow natural philosophers. With the industrial revolution and the organisation of science, it became possible to go always further in the understanding of the universe and of the matter in particular. Investigation on the constituant of matter proved to require more and more powerful machines in order to break the bricks of our world apart into ever smaller pieces and study their behaviour and extract new knowledge to help the development of humanity. So far, the largest and most power machine that was built to study the particles composing matter and test the models thought by physicists to explain their behaviours is the \acf{LHC}, a circular particle accelerator used to collide protons and heavy ions. After only a few years of investigations conducted thanks to the LHC, several discoveries, prodicted by the existing models, have been made and in the future, in order to boost the discovery potential on the LHC and be able to test hypothesis lying beyond the already aknoledged models, the instantaneous luminosity, i.e. the rate of particle interactions, will be slightly increased into a so called High Luminosity phase.

As the \acl{LHC} will see its instantaneous luminosity be increased, the detectors on the different experimental sites will have to suffer an increased background irradiation due to the byproducts of the interaction of the beams with the infrastructure. This will cause for the detectors a stress which implications need to be understood throughout the \acf{HL-LHC} phase of operations. In the case of the \acf{CMS} experiment, it is important to understand if the detectors that will be subjected to the higher levels of radiation will be able to sustain higher detection rates while displaying the same performance they have so far been operated at and if this level of performance of the detectors will stay stable for a period longer than 10 years. More specifically, the detectors placed very close to the beam line will be the most subjected to the change of luminosity. In CMS, the endcap detectors that will then be particularly affected by the stronger background radiation. The endcap detectors compose a part of the muon system of CMS and among them, the \acf{RPC}s play a key role in providing the experiment a reliable trigger on potentially interesting data.

Looking at the successive evolution of the theoretical models that gave birth to the \acf{SM} of particle physics, the need for very intense particle beams in high energy physics experiment will be explained. The implications for LHC experiments and in particular for the CMS detector will help replacing the need for longevity and rate capability studies conducted on the \acl{RPC}s that are part of its Muon System. RPCs are gaseous detectors which physics principles are non trivial and are still being investigated. Nevertheless, key processes driving the electron multiplication in the gas volume, rate capability and ageing have been successfully identified and will define the parameters that will have to be monitored during the longevity and rate capability certification campaign that is reported in this document.

\clearpage{\pagestyle{empty}\cleardoublepage}