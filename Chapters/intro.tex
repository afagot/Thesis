% Header
\renewcommand\evenpagerightmark{{\scshape\small Chapter 1}}
\renewcommand\oddpageleftmark{{\scshape\small Introduction}}

\renewcommand{\bibname}{References}

\hyphenation{}

\chapter[Introduction]%
{Introduction}
\label{chap:intro}

Grasping an understanding of the world in which they are leaving in has always been part of human life. Beyond the metaphysical questioning of our origins and purpose, curiosity has brought mankind to question its surroundings. Following the philosophy of the ancient Greeks and Indians came the development of the sciences as the systematic experimentation aimed at testing hypothesis and reproducing results obtained by fellow natural philosophers. With the industrial revolution and the organisation of science, it became possible to go always further in the understanding of the universe and of the matter in particular. Investigation on the constituent of matter proved to require more and more powerful machines in order to break apart the bricks of the world into ever smaller pieces, study their behaviour and extract new knowledge to help the development of humanity. So far, the largest and most powerful machine that was built to study the particles composing matter and test the models thought by physicists to explain their behaviour is the \acf{LHC}, a circular particle accelerator used to collide protons and heavy ions. After only a few years of investigations conducted thanks to the LHC, several discoveries, predicted by the existing models, have been made. In the future, in order to boost the discovery potential on the LHC and be able to test hypotheses lying beyond the already acknowledged models, the instantaneous luminosity, i.e. the rate of particle interactions, will be slightly increased into a so-called High Luminosity phase to boost its discovery potential.

As the \acl{LHC} will see its instantaneous luminosity be increased, the detectors on the different experimental sites will have to suffer an increased background irradiation due to the byproducts of the interaction of the beams with the infrastructure. This will cause for the detectors a stress which implications need to be understood throughout the \acf{HL-LHC} phase of operations. In the case of the \acf{CMS} experiment, it is important to understand if the detectors that will be subjected to the higher levels of radiation will be able to sustain higher detection rates while displaying the same performance they have so far been operated at and if this level of performance of the detectors will stay stable for a period longer than ten years. More specifically, the detectors placed very close to the beam line will be the most subjected to the change of luminosity. In CMS, the endcap detectors will then be particularly affected by the stronger background radiation. The endcap detectors compose a part of the muon system of CMS and among them, the \acf{RPC}s play a key role in providing the experiment a reliable trigger on potentially interesting data.

CMS experiment main focus revolves around testing the \acf{SM} of particle physics using a multipurpose detector design to detect the interaction products of the protons and ions colliding along the LHC. Looking at the successive evolution of the theoretical models that gave birth to the SM, the need for very intense particle beams in high energy physics experiment becomes clear in that the higher the center-of-mass energy for each interaction, the greater the probe on very small cross-section processes predicted by the theory, justifying the successive increase in beam energy and intensity at LHC.\\
The implications for LHC experiments and in particular for the CMS detector explain the need for longevity and rate capability studies conducted on the \acl{RPC}s which are an important part of its Muon System as it is needed to certify the quality of operation of the trigger detectors throughout the lifetime of HL-LHC.\\
RPCs are gaseous detectors which physics principles are non trivial and are still being investigated. Nevertheless, key processes driving the electron multiplication in the gas volume, rate capability and ageing have been successfully identified and will define the parameters that will have to be taken into consideration while producing the detectors extending the pseudo-rapidity coverage of RPCs toward the beam line as well as the ones to be monitored during the on-going longevity and rate capability certification campaign.\\
On the one hand, the present RPC sub-system consists in two different RPC productions. Indeed, most of the RPC detectors were produced in view of the start of LHC activities in 2010. These detectors were build in between 2007 and 2008 to equip the barrel and the three disks of each endcaps of the CMS apparatus. But during the \acf{LS1} of LHC in between end of 2012 and 2015, a fourth disk of endcap was added in order to reinforce the redundancy of the endcap trigger. Hence, new RPC detectors using the same technology were built in between 2012 and 2013. These two sets of detector productions only differ in the properties of the \acf{HPL} used for their electrodes that could lead to a different ageing rate. This is why spare detectors of both production periods have been tested over the past years to certify their good operation through HL-LHC.\\
On the other hand, producing detectors to equip a highly irradiated region such as the extension of CMS RPC endcaps requires to substantially improve the ageing properties of the chosen technology by reducing the charge deposition per ionizing particle. This can be achieved both by modifying the design of the detector volume or by improving the signal to noise ratio of the \acf{FEE} used to process the charge collected by the read-out strips making them more sensitive to weaker signals. Two \acf{iRPC} designs were selected and tested in order to extend of CMS endcap coverage.\\
Thanks to the study presented in this document, preliminary conclusions will be brought on the production of iRPCs and on the longevity of the present RPC system, providing with a better understand of the future performance of the RPC sub-system within the CMS experiment.

\clearpage{\pagestyle{empty}\cleardoublepage}