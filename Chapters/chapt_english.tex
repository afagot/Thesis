\renewcommand{\thesection}{\arabic{section}}    % chapter without number, so don't use chapterno.sectionno

\renewcommand{\bibname}{References}
% Header
\renewcommand\evenpagerightmark{{\scshape\small English summary}}
\renewcommand\oddpageleftmark{{\scshape\small English summary}}

\chapter[English summary]%
{English summary}

\hyphenation{}
\def\hyph{-\penalty0\hskip0pt\relax}

The upgrade of the \acf{LHC} toward the \acf{HL-LHC} have started in 2018. It aims at increasing the luminosity of the accelerator to boost its discovery potential. Many extensions to the \acf{SM} feature \acf{HSCPs} which, in the context of the \acf{CMS}, could be identified thanks to its muon system. An increase in instantaneous luminosity will mechanically result into an increase of the background noise and of the irradiation levels the machines will be subjected to. The current muon system needs to be certified for the HL-LHC period. On the other hand, additional Resistive Plate Chambers (RPCs) will be installed in the region closest to the beam line. The goal is to ensure the best quality possible of the muon trigger by mitigating the background effects and increasing the redundancy of the trigger system.

After an introduction on the historical context of the present work and the presentation of the physics of RPCs, this thesis will extensively discuss the longevity studies conducted at the \acf{GIF++} on the existing RPCs of the CMS muon system. Spare chambers from the 2007 and 2012 RPC production are being certified for a background rate of \SIrate{600} and for an integrated charge of \SI{840}{mC/cm^2}, corresponding to three times the worst conditions expected during HL-LHC. Thanks to comparison with reference RPCs, no clear ageing effects have been showed for the detectors which have reached 40\% to 74\% of the total irradiation program. The increase in resistivity of the irradiated RPCs could be better mitigated with a better control of the relative humidity of the gas mixture and is not believed to be an irreversible consequence of the irradiation. Finally, the performance of the irradiated detectors is comparable to the performance of the reference RPCs.\\
The thesis will also provide an overview of the ongoing R\&D of the \acf{iRPC} that will equip the high pseudo-rapidity region of CMS. Two solutions have been identified and the iRPC prototypes are reaching their final design. The prototypes are being certified at a background rate of \SIkrate{2} at which they keep an efficiency above 96\%. Demonstrators will be installed in CMS during the technical stop of winter 2021/2022 and the installation of the remaining detectors should take place in January 2023.



\clearpage{\pagestyle{empty}\cleardoublepage}

\renewcommand*{\thesection}{\thechapter.\arabic{section}}       % reset again to chaptnum.sectnum


