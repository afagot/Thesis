% options used for 'Hoofdstuk' or 'Chapter'
\documentclass[10pt,a4paper,twoside,dutch,english,openright,leqno]{book}                

%%%%%%%%%%%%%%%%%%%%%%%%%%%%%%%%%%%%%%%%%%%%%%%%%%%%%%%%%%%%%%%%%%%%%%%%%%%%%%%%%%%%%%%%%
%% PACKAGE LOADING      %%
%%%%%%%%%%%%%%%%%%%%%%%%%%

\usepackage{pseudocode}
\usepackage{times}      % use Times New Roman Type 1 fonts  (redefines sfdefault,rmdefault,ttdefault)
\usepackage[times]{quotchap}   % fancy chapter beginning
\usepackage{fancyhdr}
\usepackage[dutch,english]{babel}
\usepackage{cmap}
\usepackage[utf8]{inputenc}
\usepackage[a4paper,verbose, asymmetric, centering]{geometry}  % better control over margins
\usepackage{caption}    % better control over captions (sideways, font, ...)  
\usepackage[subrefformat=parens,justification=centering]{subcaption}
\usepackage{hyperref}
\usepackage{enumerate}  % to make it possible to define the numbers (A,a, ...)
\usepackage{verbatim}   % extra support for verbatim environments
\usepackage{float}      % you can define 'H' so that floats are forced to be putted 'here'
\usepackage{multirow}   % multirow{nrows}[bigstruts]{width}[fixup]{text} multirow cells
\usepackage{wrapfig}    % to wrap text around an image
\usepackage{array}
\ifx\pdftexversion\undefined
\usepackage[dvips]{graphicx}
\else
\usepackage[pdftex]{graphicx}
\fi
\usepackage{pdflscape}

\usepackage{lineno}		%to get the line number in the document
%\linenumbers
\usepackage{chappg}     % page numbering (chapno-pageno), for ToC
\usepackage{url}        % for better url typesetting
\usepackage{expdlist}   % Expanded description (e.g. better alignement) -> needed for acronym_expdlist package
\usepackage{acronym_expdlist}   % for list of acronyms
\usepackage{hhline}     % generates nicer table lines (without missing pixels) + more flexible
\usepackage{afterpage}  % adds \afterpage command, which makes it possible to issue \afterpage{\clearpage} which flushes all floats after this page
\usepackage{amsmath,amsfonts,amsthm,amssymb}
\usepackage{marvosym}   % for Euro symbol
\usepackage{ifthen}     % ifthenelse command
\usepackage{color}
\definecolor{bg}{rgb}{0.95,0.95,0.95}
\definecolor{dirtree}{rgb}{0.40,0.40,0.40}

\usepackage[newfloat]{minted}
\newminted{bash}{bgcolor=bg,fontsize=\footnotesize}
\newminted{cpp}{bgcolor=bg,fontsize=\footnotesize, breaklines=true, tabsize=4}
\newmintinline{cpp}{fontsize=\footnotesize, breaklines=true}
\newminted{ini}{bgcolor=bg,fontsize=\footnotesize}
\newmintinline{ini}{fontsize=\footnotesize, breaklines=true}
\newminted{text}{bgcolor=bg, fontsize=\footnotesize, breaklines=true}
\newmintinline{text}{fontsize=\footnotesize, breaklines=true}

\newenvironment{code}{\captionsetup{type=listing}}{}
\SetupFloatingEnvironment{listing}{name=Source Code}

\usepackage{dirtree}
\renewcommand*\DTstylecomment{\footnotesize\rmfamily\color{black}\textsc}
\renewcommand*\DTstyle{\footnotesize\ttfamily\textcolor{dirtree}}

\usepackage{multicol} % To write documents using multiple columns
\usepackage{longtable}
\usepackage{siunitx}
\usepackage[ % For citing references; more options than standard BibTeX or natbib
   backend=bibtex
  ,style=numeric-comp
  ,sorting=none
  ,firstinits=false
  ,maxcitenames=2
  ,sortcites=true
  ,maxbibnames=2
  ,url=false
  ,doi=true
  ,isbn=false
]{biblatex}
\bibliography{bib/PhD}

%%%%%%%%%%%%%%%%%%%%%%%%%%%%%%%%%%%%%%%%%%%%%%%%%%%%%%%%%%%%%%%%%%%%%%%%%%%%%%%%%%%%%%%%%
%% CUSTOM TOOLS    %%
%%%%%%%%%%%%%%%%%%%%%

% My own implementation of errors to provide compatibility between
% siunitx v1 and v2
\newcommand{\numerror}[2]{$(#1 \pm #2)$} % e.g. 1000 ± 10
\newcommand{\SIerror}[3]{\numerror{#1}{#2}\,\si{#3}} % e.g. (100 ± 5) nA
\newcommand{\SIsurface}[3]{\mbox{\num{#1 x #2}\,\si{#3^2}}}
\newcommand{\SIflux}[1]{\num{#1}\,\si{cm^{-2}.s^{-1}}}
\newcommand{\SIrate}[1]{\num{#1}\,\si{Hz/cm^2}}
\newcommand{\SIkrate}[1]{\num{#1}\,\si{kHz/cm^2}}
\newcommand{\siflux}{\si{cm^{-2}.s^{-1}}}
\newcommand{\sirate}{\si{Hz/cm^2}}
\newcommand{\sikrate}{\si{kHz/cm^2}}

% Functions to write century number or date numbers
\newcommand{\St}[1]{#1$^{st}$}
\newcommand{\Nd}[1]{#1$^{nd}$}
\newcommand{\Rd}[1]{#1$^{rd}$}
\newcommand{\Th}[1]{#1$^{th}$}

% Function to write scientific numbers
\newcommand{\Sci}[2]{$#1 \times 10^{#2}$}
\newcommand{\SciErr}[3]{$(#1\pm#2) \times 10^{#3}$}
\newcommand{\SciErrP}[3]{\Sci{#1}{#2} $\pm #3$\%}
\newcommand{\Ord}[1]{$10^{#1}$}

% Function to refer to sigma in statistics
\newcommand{\Sig}[1]{$#1\,\sigma$}

% Function to easily write pseudorapidity norm and norm range
\newcommand{\psrap}{$\vert\eta\vert$}
\newcommand{\psrapr}[2]{$#1<\vert\eta\vert<#2$}
\newcommand{\psrape}[1]{$\vert\eta\vert=#1$}
\newcommand{\psrapl}[1]{$\vert\eta\vert<#1$}
\newcommand{\psrapg}[1]{$\vert\eta\vert>#1$}

% Command to simplify writing straight derivatives
\newcommand{\deriv}{\mathrm{d}}

% Command to write pT without using to many characters
\newcommand{\pT}{p$_T$}

% Command to call the L of luminosity
\newcommand{\lum}{\mathcal{L}}

% Command to write GWP
\newcommand{\GWP}[1]{GWP $=$ #1}
\newcommand{\GWPsim}[1]{GWP $\sim$ #1}
\newcommand{\GWPlt}[1]{GWP $<$ #1}
\newcommand{\GWPgt}[1]{GWP $>$ #1}
\newcommand{\GWPleq}[1]{GWP $\leq$ #1}
\newcommand{\GWPgeq}[1]{GWP $\geq$ #1}

%%%%%%%%%%%%%%%%%%%%%%%%%%%%%%%%%%%%%%%%%%%%%%%%%%%%%%%%%%%%%%%%%%%%%%%%%%%%%%%%%%%%%%%%%
%% PAGE LAYOUT %%
%%%%%%%%%%%%%%%%%

%%%%%%%%%%%%%%%%%%%%%%%%%%%%%%%%%%%%%%%%%%%%%%%%%%%%%%%%%%%%%%%%%%%%%%
% document: page_layout_definition.tex
%
% last modified: $Id: page_layout_definition.tex,v 1.1 2005/11/18 11:49:23 bvolckae Exp $
%UPDATED ON 05/02/2014 BY SEVENOIS RUBEN TO KEEP COMPATIBILITY WITH NEWER PACKAGE VERSIONS
%
% author: Filip De Turck, Stefaan Vanhastel, Bart Duysburgh, Brecht Vermeulen, Bruno Volckaert, Steven Van den Berghe
%%%%%%%%%%%%%%%%%%%%%%%%%%%%%%%%%%%%%%%%%%%%%%%%%%%%%%%%%%%%%%%%%%%%%%

%%%%%%%%%%%%%%%%%%%%%%%%%%%%%%%%%%%%%%%%%%%%%%%%%%%%%%%%%%%%%%%%%%%%%%

%
% basic dimensions when printing the small page %
% and by using the geometry package             %

% settings Filip en Stefaan
%\geometry{bottom=4.0cm,rmargin=4.25cm,body={12.5cm,19.5cm}} % 10pt op a4
%\geometry{marginpar=0.0cm,marginparsep=0.0cm,twosideshift=0.0cm}

% new settings (according book pim which was approved by the promotors) by Bart Duysburgh
%\geometry{bottom=5.34cm,rmargin=4.5cm,body={11.5cm,18.92cm}} % 10pt op a4
%\geometry{marginpar=0.0cm,marginparsep=0.0cm,twosideshift=0.0cm}

\geometry{body={14cm,21cm}} % 10pt op a4
%\geometry{bottom=5.34cm,rmargin=4.5cm,body={11.5cm,18.92cm}} % 10pt op a4
\geometry{twoside,marginpar=0.0cm,marginparsep=0.0cm}%,twosideshift=0.0cm}

%\geometry{bottom=2.15cm,rmargin=2.5cm,body={14.14125cm,23.6cm}} % 12pt op a4

%%%%%%%%%%%%%%%%%%%%%%%%%%%%%%%%%%%%%%%

\setlength{\textwidth}{14cm}
%\setlength{\textheight}{19.5cm}
%\setlength{\topmargin}{0.0cm}
%\setlength{\oddsidemargin}{0.7cm}
%\setlength{\evensidemargin}{0.7cm}
%\setlength{\marginparwidth}{0pt}
%\setlength{\marginparsep}{0pt}

%%%%%%%%%%%%%%%%%%%%%%%%%%%%%%%%%%%%%%%

\renewcommand{\topfraction}{0.8}

%%%%%%%%%%%%%%%%%%%%%%%%%%%%%%%%%%%%%%%
% headings %

\fancypagestyle{plain}{
\fancyhf{}
\renewcommand{\headrulewidth}{0pt}
\renewcommand{\footrulewidth}{0pt}}

\pagestyle{fancy}
\fancyhf{} %clear all header and footer fields
\addtolength{\headwidth}{\marginparsep}
\addtolength{\headwidth}{\marginparwidth}

\renewcommand{\chaptermark}[1]{\markboth{#1}{}}
%\renewcommand{\sectionmark}[1]{\markright{\thesection\ #1}}

%\newcommand\fdtsvrightmarktmp{{\scshape\small Chapter }}
%\newcommand\fdtsvrightmark{{\scshape\small{Acknowledgment}}}
%\newcommand\fdtsvleftmark{{\scshape\small{Dankwoord}}}

\newcommand\oddpageleftmark{}
\newcommand\evenpagerightmark{}

%\fancyhead[LE,RO]{\itshape\bfseries\small\thepage}
%\fancyhead[LO]{\itshape\bfseries\small\leftmark}
%\fancyhead[RE]{\itshape\bfseries\small\rightmark}
\fancyhead[LE,RO]{\small\thepage}
\fancyhead[LO]{\oddpageleftmark}
\fancyhead[RE]{\evenpagerightmark}
%\fancyfoot[C]{\itshape\bfseries\footnotesize \chaptername\ \thechapter}

%%%%%%%%%%%%%%%%%%%%%%%%%%%%%%%%%%%%%%%%%%%%%%%%%%%%


%%%%%%%%%%%%%%%%%%%%%%%%%%%%%%%%%%%%%%%%%%%%%%%%%%%%
% depth of numbering and depth of table of contents %

\setcounter{tocdepth}{3} % titels tot en met niveau subsubsection worden in table of contents opgenomen
\setcounter{secnumdepth}{3} % tot en met niveau subsubsection wordt er genummerd
%%%%%%%%%%%%%%%%%%%%%%%%%%%%%%%%%%%%%%%%%%%%%%%%%%%



%%%%%%%%%%%%%%%%%%%%%%%%%%%%%%%%%%%%%%%%%%%%%%%%%%%
%%%%% Definition for Big letter at the beginning of a paragraph %%
\def\PARstart#1#2{\begingroup\def\par{\endgraf\endgroup\lineskiplimit=0pt}
    \setbox2=\hbox{\uppercase{#2} }\newdimen\tmpht \tmpht \ht2
    \advance\tmpht by \baselineskip\font\hhuge=cmr10 at \tmpht
    \setbox1=\hbox{{\hhuge #1}}
    \count7=\tmpht \count8=\ht1\divide\count8 by 1000 \divide\count7 by\count8
    \tmpht=.001\tmpht\multiply\tmpht by \count7\font\hhuge=cmr10 at \tmpht
    \setbox1=\hbox{{\hhuge #1}} \noindent \hangindent1.05\wd1
    \hangafter=-2 {\hskip-\hangindent \lower1\ht1\hbox{\raise1.0\ht2\copy1}%
    \kern-0\wd1}\copy2\lineskiplimit=-1000pt}
%%%%%%%%%%%%%%%%%%%%%%%%%%%%%%%%%%%%


%%%%%%%%%%%%%%%%%%%%%%%%%%%%%%%%%%%%%%%%%%%%%%%%%%%%
%%% Nog een paar andere zaken  %%%%
%% om een cross-ref naar een voetnoot te kunnen maken definier ik \usefn %%
\newcommand{\usefn}[1]{\mbox{\textsuperscript{\normalfont#1}}}

%\setlength{\captionindent}{1cm}
\renewcommand{\captionfont}{\small \itshape \mdseries \rmfamily}

\AtBeginDocument{%
%   \renewcommand{\figurename}{Fig.}%
%   \renewcommand{\tablename}{TABLE}%
   \renewcommand{\tablename}{Table}
   \renewcommand{\bibname}{References}%
}

%%%%%%%%%%%%%%%%%%%%%%%%%%%%%%%%%%%%%%%%%%%%%%%%%%%



%%%%%%%%%%%%%%%%%%%%%%%%%%%%%%%%%%%%%%%%%%%%%%%%%%%%%%%%%%%%%%%%%%%%%%%%%%%%%%%%%%%%%%%%%
%% HYPHENATION %%
%%%%%%%%%%%%%%%%%

\input{hyphenation.tex} % to have a separate file with hyphenations

%%%%%%%%%%%%%%%%%%%%%%%%%%%%%%%%%%%%%%%%%%%%%%%%%%%%%%%%%%%%%%%%%%%%%%%%%%%%%%%%%%%%%%%%%
%%  START BOOK    %%
%%%%%%%%%%%%%%%%%%%%

\begin{document}
% definition of some useful variables
\latintext
\graphicspath{{fig/}}
\restylefloat{figure}
\restylefloat{table}
\newfloat{algorithm}{ht}{alg}

% definition of equal column spaning under a multicolumn
\makeatletter


\def\x@multispan#1{%
  \global\let\@tempa\@empty
  \@multicnt#1\relax
  \loop\ifnum\@multicnt>\@ne
  \xdef\@tempa{\@tempa\kern\dimen@i\hfill&\omit}%
   \advance\@multicnt\m@ne
  \repeat
  \@tempa\kern\dimen@i\hfill}


\long\def\xmulticolumn#1#2#3{%
 \omit
 \begingroup
   \def\@addamp{\if@firstamp \@firstampfalse \else
                \@preamerr 5\fi}%
  \@mkpream{#2}\@addtopreamble\@empty
  \endgroup
  \def\@sharp{#3}%
  \setbox\z@\hbox{{\@preamble}}%
\global\dimen@i\wd\z@
\global\divide\dimen@i#1\relax
 \ignorespaces
\x@multispan{#1}}
\makeatother

%%%%%%%%%%%%%%%%%%%%%%%%%%%%%%%%%%%%%%%%%%%%%%%%%%%%%%%%%%%%%%%%%%%%%%%%%%%%%%%%%%%%%%%%%
%%   FRONT PAGE       %%
%%%%%%%%%%%%%%%%%%%%%%%%
% 
 \thispagestyle{empty}   % no headings for this page
% 
% % Header
 \noindent
 \begin{minipage}{3cm}%
   \href{https://www.ugent.be/}{\includegraphics*[width=2cm]{UGent.pdf}}
 \end{minipage}\hfill
 \begin{minipage}{8cm}
 \raggedleft
 \textsf{Ghent University\\
 Faculty of Sciences\\
 Department of Physics and Astronomy}
 \end{minipage}
\vspace{2cm}
% 
% % Title
\bigskip
 \hrule
 \vspace{5mm}
   \begin{centering}
     \LARGE \textbf{\textsf{Consolidation and extension of the CMS Resistive Plate Chamber system in view of the High-Luminosity LHC Upgrade}}\\
   \end{centering}
 \vspace{5mm}
 \hrule
% 
% % Footer
 \vfill
 \begin{minipage}{2.0cm}%
     \href{https://cms.cern/}{\includegraphics*[width=2.0cm]{CMS.pdf}}
 \end{minipage}\hfill
 \begin{minipage}{7cm}
 \centering
 \LARGE\textsf{Alexis Fagot\\}
 \vspace{5mm}
 \normalsize\textsf{Thesis to obtain the degree of\\
 Doctor of Philosophy in Physics\\
 Academic year 2019-2020}
 \end{minipage}\hfill
 \begin{minipage}{2.0cm}%
     \href{https://home.cern/}{\includegraphics*[width=2.0cm]{CERN.pdf}}
 \end{minipage}\hfill


%%%%%%%%%%%%%%%%%%%%%%%%%%%%%%%%%%%%%%%%%%%%%%%%%%%%%%%%%%%%%%%%%%%%%%%%%%%%%%%%%%%%%%%%%
%% INFORMATION PAGE     %%
%%%%%%%%%%%%%%%%%%%%%%%%%%

\clearpage{\pagestyle{empty}\cleardoublepage}
\thispagestyle{empty}

\normalsize

% Header
\noindent
\begin{minipage}{3cm}%
   \href{https://www.ugent.be/}{\includegraphics*[width=2cm]{UGent.pdf}}
 \end{minipage}\hfill
 \begin{minipage}{8cm}
 \raggedleft
 \textsf{Ghent University\\
 Faculty of Sciences\\
 Department of Physics and Astronomy}
 \end{minipage}
\\[2cm]

\vfill
\noindent \begin{tabular}{ @{} l l}
Promotors: & Dr.\ Michael Tytgat\\
 & Prof.\ Dr.\ Dirk Ryckbosch\\
\end{tabular}
\\[2cm]

\noindent Ghent University \\
\noindent Faculty of Sciences\\
[0.3cm]
\noindent Department of Physics and Astronomy \\
\noindent Proeftuinstraat 86, B-9000 Ghent, Belgium\\
[0.3cm]
\noindent Tel.: +32 9 264.65.28\\
\noindent Fax.: +32 9 264.66.97

\vfill

 \begin{minipage}{2.0cm}%
     \href{https://cms.cern/}{\includegraphics*[width=2.0cm]{CMS.pdf}}
 \end{minipage}\hfill
 \begin{minipage}{7cm}
 \centering
 \LARGE\textsf{Alexis Fagot\\}
 \vspace{5mm}
 \normalsize\textsf{Thesis to obtain the degree of\\
 Doctor of Philosophy in Physics\\
 Academic year 2019-2020}
 \end{minipage}\hfill
 \begin{minipage}{2.0cm}%
     \href{https://home.cern/}{\includegraphics*[width=2.0cm]{CERN.pdf}}
 \end{minipage}\hfill
\clearpage{\pagestyle{empty}\cleardoublepage}

%%%%%%%%%%%%%%%%%%%%%%%%%%%%%%%%%%%%%%%%%%%%%%%%%%%%%%%%%%%%%%%%%%%%%%%%%%%%%%%%%%%%%%%%%
%%   TABLE OF CONTENT   %%
%%%%%%%%%%%%%%%%%%%%%%%%%%

\hyphenation{bu-reau-ge-no-ten}
\frontmatter
\renewcommand{\contentsname}{Table of Contents}
\tableofcontents

\renewcommand{\bibname}{References}

%%%%%%%%%%%%%%%%%%%%%%%%%%%%%%%%%%%%%%%%%%%%%%%%%%%%%%%%%%%%%%%%%%%%%%%%%%%%%%%%%%%%%%%%%
%% SUMMARY IN ENGLISH   %%
%%%%%%%%%%%%%%%%%%%%%%%%%%
%\renewcommand{\thesection}{\arabic{section}}    % chapter without number, so don't use chapterno.sectionno

\renewcommand{\bibname}{References}
% Header
\renewcommand\evenpagerightmark{{\scshape\small English summary}}
\renewcommand\oddpageleftmark{{\scshape\small English summary}}

\chapter[English summary]%
{English summary}

\hyphenation{}
\def\hyph{-\penalty0\hskip0pt\relax}

The upgrade of the \acf{LHC} toward the \acf{HL-LHC} have started in 2018. It aims at increasing the luminosity of the accelerator to boost its discovery potential. Many extensions to the \acf{SM} feature \acf{HSCPs} which, in the context of the \acf{CMS}, could be identified thanks to its muon system. An increase in instantaneous luminosity will mechanically result into an increase of the background noise and of the irradiation levels the machines will be subjected to. The current muon system needs to be certified for the HL-LHC period. On the other hand, additional Resistive Plate Chambers (RPCs) will be installed in the region closest to the beam line. The goal is to ensure the best quality possible of the muon trigger by mitigating the background effects and increasing the redundancy of the trigger system.

After an introduction on the historical context of the present work and the presentation of the physics of RPCs, this thesis will extensively discuss the longevity studies conducted at the \acf{GIF++} on the existing RPCs of the CMS muon system. Spare chambers from the 2007 and 2012 RPC production are being certified for a background rate of \SIrate{600} and for an integrated charge of \SI{840}{mC/cm^2}, corresponding to three times the worst conditions expected during HL-LHC. Thanks to comparison with reference RPCs, no clear ageing effects have been showed for the detectors which have reached 40\% to 74\% of the total irradiation program. The increase in resistivity of the irradiated RPCs could be better mitigated with a better control of the relative humidity of the gas mixture and is not believed to be an irreversible consequence of the irradiation. Finally, the performance of the irradiated detectors is comparable to the performance of the reference RPCs.\\
The thesis will also provide an overview of the ongoing R\&D of the \acf{iRPC} that will equip the high pseudo-rapidity region of CMS. Two solutions have been identified and the iRPC prototypes are reaching their final design. The prototypes are being certified at a background rate of \SIkrate{2} at which they keep an efficiency above 96\%. Demonstrators will be installed in CMS during the technical stop of winter 2021/2022 and the installation of the remaining detectors should take place in January 2023.



\clearpage{\pagestyle{empty}\cleardoublepage}

\renewcommand*{\thesection}{\thechapter.\arabic{section}}       % reset again to chaptnum.sectnum




%%%%%%%%%%%%%%%%%%%%%%%%%%%%%%%%%%%%%%%%%%%%%%%%%%%%%%%%%%%%%%%%%%%%%%%%%%%%%%%%%%%%%%%%%
%% THE BOOK ITSELF   %%
%%%%%%%%%%%%%%%%%%%%%%%
\mainmatter     % related to chappg numbering
\renewcommand*{\thesection}{\thechapter.\arabic{section}}

\newcommand\fdtsvrightmarktmp{{\scshape\small Chapter }}
\renewcommand\evenpagerightmark{{\scshape\small\chaptername\ \thechapter}}
\renewcommand\oddpageleftmark{{\scshape\small\leftmark}}

% To get my plot widths consistent
\newlength{\plotwidth}
\setlength{\plotwidth}{.8\textwidth}

\baselineskip 13.0pt

% Header
\renewcommand\evenpagerightmark{{\scshape\small Chapter 1}}
\renewcommand\oddpageleftmark{{\scshape\small Introduction}}

\renewcommand{\bibname}{References}

\hyphenation{}

\chapter[Introduction]%
{Introduction}
\label{chap:intro}



\clearpage{\pagestyle{empty}\cleardoublepage}

% Header
\renewcommand\evenpagerightmark{{\scshape\small Chapter 2}}
\renewcommand\oddpageleftmark{{\scshape\small Investigating the \si{TeV} scale}}

\renewcommand{\bibname}{References}

\hyphenation{}

\chapter[Investigating the \si{TeV} scale]%
{Investigating the \si{TeV} scale}
\label{chapt:2}

	\vfill
	
	{\Large\textit{,,We may regard the present state of the universe as the effect of the past and the cause of the future. An intellect which at any given moment knew all of the forces that animate nature and the mutual positions of the beings that compose it, if this intellect were vast enough to submit the data to analysis, could condense into a single formula the movement of the greatest bodies of the universe and that of the lightest atom; for such an intellect nothing could be uncertain and the future just like the past would be present before its eyes.''}}\\
	
	{\normalsize\raggedleft - Pierre Simon de Laplace, \textit{A Philosophical Essay on Probabilities}, 1814}
	
	\vfill
	
\newpage

	Throughout history, physics experiment became more and more powerful in order to investigate finer details of nature and helped understanding the elementary blocks of matter and the fondamental interactions that bond them in the microscopic world. Nowadays, the \acf{SM} of particle physics is the most accurate theory designed to explain the behaviour of particles and was able to make very precise predictions that are constantly verified, although some hints of new physics are visible as bricks are still missing to have a global comprehension of the Universe.\\
	To highlight the limits of the SM and test the different alternative theories, ever more powerful machines are needed. This is in this context that the \acf{LHC} has been thought and built to accelerate and collide particles at energies exceeding anything that had been done before. Higher collision energies and high pile-up imply the use of enormous detectors to measure the properties of the interaction products. The \acf{CMS} is a multipurpose experiment that have been designed to study the proton-proton collisions of the LHC and give answers on various high energy physics scenari. Nevertheless, the luminosity delivered by the collider will in the future be increased to levels beyond the original plans to improve its discovery potential giving no choice to experiments such as CMS to upgrade their technologies to cope with the increased radiation levels and detection rates.

\section{The Standard Model of Particle Physics}
\label{chapt2:sec:SM}

	In this early \St{21} century it is now widely accepted that matter is made of elementary blocks referred to as \textit{elementary particles}. The physics theory that classifies and describes the best the behaviour and interaction of such elementary particles is the so called\acl{SM} that formalizes 3 of the 4 fondamental interactions (electromagnetic, weak and strong interactions). It's development took place during the \Th{20} century thanks to a strong collaboration in between the theoretical and experimental physicists.\\

	\subsection{A brief history of particle physics}
	\label{chapt2:ssec:history}
	
	The idea that nature is composed of elementary bricks, called \textit{atomism}, is not contemporary as it was already discussed by Indian or Greek philosophers during antiquity. In Greece, atomism has been rejected by Aristotelianism as the existance of \textit{atoms} would imply the existance of a void that would violate the physical principles of Aristotle philosophy. Aristotelianism has been considered as a reference in the european area until the \Th{15} century and the italian \textit{Rinascimento} where antic text and history started to be more deeply studied. The re-discovery of Platon's philosophy would allow to open the door to alternative theories and give a new approach to natural sciences where experimentation would become central. A new era of knowledge was starting. By the begining of the \Th{17} century, atomism was re-discovered by philosophers and the very first attempt to estimate an \textit{atom} size would be provided by Magnetus in 1646. Although his \textit{atoms} correspond to what would nowadays be called \textit{molecules}, Magnetus achieved feats by calculating that the number of molecules in a grain of incense would be of the order of $10^{18}$ simply by considering the time necessary to smell it everywhere in a large church after the stick was lit on. It is now known that this number only falls short by 1 order of magnitude.\\
	
	An alternative philosophy to atomism popularized by Descartes was corpuscularianism. Built on ever divible corpuscles, contrary to atoms, it's principles would be mainly used by alchemists like Newton who would later develop a corpuscular theory of light. Boyle would combine together ideas of both atomism or corpuscularianism leading to mechanical philosophy. The \Th{18} century have seen the development of engineering providing philosophical thought experiments with repeatable demonstration and a new point of view to explain the composition of matter and Lavoisier would greatly contribute to chemistry and atomism by publishing in 1789 a list of 33 chemical elements corresponding to what is now called \textit{atoms}. In the early \Th{19} century Dalton would summarize the knowledge on composition of matter and Fraunhofer would invent the spectrometer and discover the spectral lines. The rise of atomic physics, chemistry and mathematical formalism would unravel the different atomic elements and ultimately, the \Th{20} century would see the very first sub-atomic particles.\\
	
	The negative \textit{electron} would be the first to be discovered in 1897 by Thompson after 3 decades of research on cathode rays and in 1900, Becquerel would show the \textit{beta rays} emitted by radium had the same properties pointing to eletrons as a constituant of atoms. In 1909, Rutherford and Royds showed that \textit{alpha} particles could combine with 2 electrons to form a $^4$He after they already constrained the atom structure in 1907 through the gold foil experiment that highlighted atoms where mainly empty with nearly all its mass contained into a tiny positively charged \textit{nucleus}. The link in between atomic number and number of positive chages contained into the nuclei would fast be understood and the different kind of element transmutation appeared to be purely nuclear processes making clear that the electromagnetic nature of chemical transformation could not possibly change nuclei. Thus a new branch in physics appeared to study nuclei exclusively: the nuclear physics which would in turn give birth to quantum physics.\\
	
	By studying alpha emission, Rutherford reported in 1919 the very first nuclear reaction leading to the discovery that the hydrogen nucleus was composed of a single positively charged particle that was then baptised \textit{proton}. This idea came from 1815 Prout's hypothesis proposing that all atoms are composed of \textit{"protolytes"} (i.e. hydrogen atoms). By using scintillation detectors, Rutherford could highlight typical hydrogen nuclei signature and understand that the impact of alpha particles with nitrogen would knock out an hydrogen nucleus and produce an oxygen 17, as explicited in Formula~\ref{eq:nuclear} and would then postulate that protons are building bricks of all elements.\\
	
	\begin{equation}
		\label{eq:nuclear}
		^{14}N + \alpha \rightarrow ^{17}O + p
	\end{equation}
	
	With this assumption and the discovery of isotopes together with Aston, elements with identical atomic number but different masses, Rutherford would propose that all elements' nuclei but hydrogen's are composed of both charged particles, protons, and of chargeless particles, which he called \textit{neutrons}, and that these neutral particles would help maintaining nuclei as one, as charged protons were likely to electrostatically repulse each other, and introduced the idea of a new force, a \textit{nuclear} force. Though the first idea concerning neutrons was a bond state of protons and electrons as it was known that the beta decay, emitting electrons, was taking place in the nucleus, it was then showed that such a model would hardly be possible due to Heisenberg's uncertainty principle and by the recently measured \textit{spin} of both protons and electrons. The spin, discovered through the study of the emission spectrum of alkali metals, would be understood as a "two-valued quantum degree of freedom" and formalized by Pauli and extended by Dirac, to take the relativist case into account. Measured to be $\frac{1}{2} \hbar$ for both, it was impossible to arrange an odd number of half integer spins and obtain a global nucleus spin that would be integer. Finally, in 1932, following the discovery of a new neutral radiation, Chadwick could discover the neutron as an uncharged particle with a mass similar to that of the proton.\\
	
	\subsection{Construction and test of the model}
	\label{chapt2:ssec:model}
	
	\subsection{Investigating the TeV scale}
	\label{chapt2:ssec:TeV}

\section{The \acl{LHC} \& the \acl{CMS}}
\label{chapt2:sec:LHC-CMS}

	Throughout its history, CERN has played a leading role in high energy particle physics. Large regional facilities such as CERN were thought after the second world war in an attempt to increase international scientific collaboration and allows scientists to share the forever increasing costs of experiment facilties required due to the need for increasing the energy in the center of mass to deeper probe matter. The construction of the first accelerators at the end of the 50s, the \acf{SC} and the \acf{PS}, was directly followed by the first observation of antinuclei in 1965~\cite{MASSAM1965}. Strong from the experience of the \acf{ISR}, the very first proton-proton collider that showed hints that protons are not elementary particles, the \acf{SPS} was built in the 70s to investigate the structure of protons, the preference for matter over antimatter, the state of matter in the early universe or exotic particles, and lead to the discovery in 1983 of the W and Z bosons~\cite{UA1W1983,UA2W1983,UA1Z1983,UA2Z1983}. These newly discovered particles and the electroweak intereaction would then be studied in details by the \acf{LEP} collider that will help to prove in 1989 that there only are three generations of elementary particles~\cite{ALEPH1989}. The LEP would then be dismantled in 2000 to allow for the LHC to be constructed in the existing tunnel.

	\subsection{LHC, the most powerful particle accelerator}
	\label{chapt2:ssec:LHC}
	
	The LHC has always been considered as an option to the future of CERN. At the moment of the construction of the LEP beneath the border between France and Switzerland, the tunnel was built in order to accomodate what would be a \acl{LHC} with a dipole field of \SI{10}{T} and a beam energy in between 8 and \SI{9}{TeV}~\cite{ANNUALREPORT1984} directly followed in 1985 with the creation of a 'Working Group on the Scientific and Technological Future of CERN' to investigate such a collider~\cite{ANNUALREPORT1985}. The decision was finally taken almost 10 years later, in 1994, to construct the LHC in the LEP tunnel~\cite{ANNUALREPORT1994} and the approval of the 4 main experiments that would take place at the 4 interaction points would come in 1997~\cite{ANNUALREPORT1997} and 1998~\cite{ANNUALREPORT1998}:
	
	\begin{itemize}
		\item[•] ALICE~\cite{ALICELOI} has been designed in the purpose of studying quark-gluon plasma that is believed to have been a state of matter that existed in the very first moment of the universe.
		\item[•] ATLAS~\cite{ATLASLOI} and CMS~\cite{CMSLOI} are general purpose experiements that have been designed with the goal of continuing the exploration of the Standard Model and investigate new physics.
		\item[•] LHCb~\cite{LHCBLOI} has been designed to investigate the preference of matter over antimatter in the universe through the CP violation.
	\end{itemize}
	
	These large scale experiments, as well as the full CERN accelerator complex, are displayed on Figure~\ref{fig:CERNComplex}.

	\begin{figure}[H]
		\centering
		\hspace*{-0.1\linewidth}
		\includegraphics[width=1.2\linewidth]{fig/chapt2/CERN_Accelerator_Complex.png}
		\caption{\label{fig:CERNComplex} CERN accelerator complex.}
	\end{figure}
	
	The LHC is a \SI{27}{km} long hadron collider and the most powerful accelerator used for particle physics since 2008~\cite{LHC2008}. The LHC was originally designed to collide protons at a center-of-mass energy of \SI{14}{TeV} and luminosity of $10^{34}$ \si{cm^{-2}s^{-1}}, as well as $Pb$ ions at a center-of-mass energy of \SI{2.8}{TeV/A} with a peak luminosity of $10^{27}$ \si{cm^{-2}s^{-1}}. Run 1 of LHC, when the center-of-mass energy only was half of the nominal LHC energy, was enough for both CMS and ATLAS to discover the Higgs boson~\cite{HIGGS2015} and for LHCb to discover pentaquarks~\cite{PENTAQUARK2015} and confirm the existance of tetraquarks~\cite{TETRAQUARK2017}. Nevertheless, after the \acf{LS3} (2024-2026), the accelerator will be in the so called \acf{HL-LHC} configuration~\cite{HLLHC2017}, increasing its instantaneous luminosity to $10^{35}$ \si{cm^{-2}s^{-1}} for $pp$ collisions and to $4.5\times 10^{27}$ \si{cm^{-2}s^{-1}}, boosting the discovery potential of the LHC.
	
		\subsubsection{Particle acceleration}
		\label{chapt2:sssec:acceleration}
	
	The LHC is the last of a long series of accelerating devices. Before being accelerated by the LHC, the particles need to pass through different acceleration stages. All these acceleration stages are visible on Figure~\ref{fig:CERNComplex} and pictures of the accelerators are showed in Figure~\ref{fig:CERNAccelerators}.\\
	
	The story of accelerated protons at CERN starts with a bottle of hydrogen gas injected into the source chamber of the linear particle accelerator \textit{LINAC 2} 2 in which a strong electric field strips the electron off the hygroden molecules only to keep their nuclei, the protons. The cylindrical conductors, alternatively positively or negatively charged by radiofrequency cavities, accelerate protons by pushing them from behing and pulling them from the front and ultimately give them an energy of \SI{50}{MeV}, increasing their mass by 5\% in the process.\\
	
	When exiting the LINAC 2, the protons are divided into 4 bunches and injected into the 4 superimposed synchrotron rings of the \textit{Booster} where they are then accelerated to reach an energy of \SI{1.4}{GeV} before being injected into the \textit{PS}. Before the Booster was operational in 1972, the protons were directly injected into the PS from the LINAC 2 but the low injection energy limited the amount of protons that could be accelerated at once by the PS. With the Booster, the PS accepts approximately 100 times more particles.\\
	
	The 4 proton bunches are thus sent as one to the PS where their energy eventually reaches \SI{26}{GeV}. Since the 70s, the main goal of this \SI{628}{m} circumference synchrotron has been to supply other machines with accelerated particles. Nowadays, not only the PS accelerates protons, it also accelerates heavy ions from the \textit{\acf{LEIR}}. Indeed, the LHC experiments are not only designed to study $pp$-collinsions but also $Pb$-collisions. Lead is first injected into the dedicated linear collider \textit{LINAC 3}, that accelerate the ions using the same principle than LINAC 2. Electrons are striped off the lead ions all along the acceleration process and eventually, only bare nuclei are injected in the LEIR whose goal is to transform the long ion pulses received into short dense bunches for LHC. Ions injected and stored in the PS were aceelerated by the LEIR from \SI{4.2}{MeV} to \SI{72}{MeV}.\\
	
	Directly following the PS, is finally the last acceleration stage before the LHC, the \SI{7}{km} long \textit{SPS}. The SPS accelerates the protons to \SI{450}{GeV} and inject proton in both LHC accelerator rings that will increase their energy up to \SI{7}{TeV}. When the LHC runs with heavy lead ions for ALICE and LHCb, ions are injected and accelerated to reach the energy of \SI{2.8}{TeV/A}.

	\begin{figure}[H]
		\centering
		\includegraphics[width=0.5\linewidth]{fig/chapt2/CERN-accelerators.jpg}
		\caption{\label{fig:CERNAccelerators} Pictures of the different accelerators. From top to bottom: first the LINAC 2 and the $Pb$ source of LINAC 3. Then the Booster and the LEIR. Finally, the PS, the SPS and the LHC.}
	\end{figure}
	
	The LHC beams are not continuous and are rather organised in bunch of paticles. When in $pp$-collision mode, the beams are composed of 2808 bunches of $1.15 \times 10^{11}$ protons separated by \SI{25}{ns}. When in $Pb$ collision mode, the 592 $Pb$ bunches are on the contrary composed of $2.2 \times 10^8$ ions separated by \SI{100}{ns}. The two parrallel proton beams of the LHC are contained in a single twin-bore magnet due to the space restriction in the LEP tunnel. Indeed, building 2 completely separate accelerator rings next to each other was impossible. The dipoles of the 1232 twin-bore magnets are showed in Figure~\ref{fig:LHCDipole} alongside the magnetic field generated along the dipole section to accelerate the particles. The dipoles generate a nominal field of \SI{8.33}{T}, needed to give protons and lead nucleons their nominal energy. Some 392 quadrupoles, presented in Figure~\ref{fig:LHCQuadrupole}, are also used to focus to the beams, as well as other multipoles to correct smaller imperfections.
	
	\begin{figure}[H]
		\begin{subfigure}{0.5\linewidth}
			\centering
			\includegraphics[height = 4cm]{fig/chapt2/LHC-dipole.png}
			\caption{\label{fig:LHCDipole:A}}
		\end{subfigure}
		\begin{subfigure}{0.5\linewidth}
			\centering
			\includegraphics[height = 4cm]{fig/chapt2/LHC-dipole-field.jpg}
			\caption{\label{fig:LHCDipole:B}}
		\end{subfigure}
		\caption{\label{fig:LHCDipole} Figure~\ref{fig:LHCDipole:A}: schematics of the LHC cryodipoles. 1: Superconducting Coils, 2: Beam pipe, 3: Heat exchanger Pipe, 4: Helium-II Vessel, 5: Superconducting Bus-bar, 6: Iron Yoke, 7: Non-Magnetic Collars, 8: Vacuum Vessel, 9: Radiation Screen, 10: Thermal Shield, 11: Auxiliary Bus-bar Tube, 12: Instrumentation Feed Throughs, 13: Protection Diode, 14: Quadrupole Bus-bars, 15: Spool Piece Bus-bars. Figure~\ref{fig:LHCDipole:B}: magnetic field and resulting motion force applied on the beam particles.}
	\end{figure}
	
	\begin{figure}[H]
		\begin{subfigure}{0.5\linewidth}
			\centering
			\includegraphics[height = 4cm]{fig/chapt2/LHC-quadrupole.jpg}
			\caption{\label{fig:LHCQuadrupole:A}}
		\end{subfigure}
		\begin{subfigure}{0.5\linewidth}
			\centering
			\includegraphics[height = 4cm]{fig/chapt2/LHC-quadrupole-field.png}
			\caption{\label{fig:LHCQuadrupole:B}}
		\end{subfigure}
		\caption{\label{fig:LHCQuadrupole} Figure~\ref{fig:LHCQuadrupole:A}: picture of the LHC quadrupoles. Figure~\ref{fig:LHCQuadrupole:B}: magnetic fields and resulting focussing force applied on the beam by 2 consecutive quadrupoles.}
	\end{figure}
	
		\subsubsection{LHC discoveries and LHC physics program}
		\label{chapt2:sssec:discovery}
		
	The very first proton beam successfully circulated in the LHC in September 2008 directly followed by an incident leading to mechanical damage that would delay the LHC program for a year until November 2009.
	
		\subsubsection{\acl{HL-LHC}}
		\label{chapt2:sssec:HL-LHC}

	\subsection{CMS, a multipurpose experiment}
	\label{chapt2:ssec:CMS}

\section{Muon Phase-II Upgrade}
\label{chapt2:sec:phase-2}

After the more than two years lasting \acf{LS1}, the \acf{LHC} delivered its very first Run-II proton-proton collisions early 2015. LS1 gave the opportunity to the LHC and to the its experiments to undergo upgrades. The accelerator is now providing collisions at center-of-mass energy of \SI{13}{TeV} and bunch crossing rate of \SI{40}{MHz}, with a peak luminosity exceeding its design value. During the first and upcoming second LHC Long Shutdown, the \acf{CMS} detector is also undergoing a number of upgrades to maintain a high system performance~\cite{MUONTDR}.

From the LHC Phase-2 or \acf{HL-LHC} period onwards, i.e. past the \acf{LS3}, the performance degradation due to integrated radiation as well as the average number of inelastic collisions per bunch crossing, or pileup, will rise substantially and become a major challenge for the LHC experiments, like CMS that are forced to address an upgrade program for Phase-II~\cite{PHASEIITP}. Simulations of the expected distribution of absorbed dose in the CMS detector under HL-LHC conditions, show in figure~\ref{fig:Dose} that detectors placed close to the beamline will have to withstand high irradiation, the radiation dose being of the order of a few tens of\si{Gy}.

\begin{figure}[H]
	\centering
	\includegraphics[width=0.7\textwidth]{fig/chapt2/HL-LHC-Dose.png}
	\caption{\label{fig:Dose} Absorbed dose in the CMS cavern after an integrated luminosity of \SI{3000}{\femto\per\barn}. R is the transverse distance from the beamline and Z is the distance along the beamline from the Interaction Point at Z=0.}
\end{figure}

The measurement of small production cross-section and/or decay branching ratio processes, such as the Higgs boson coupling to charge leptons or the $B_s \longrightarrow \mu^+\mu^-$ decay, is of major interest and specific upgrades in the forward regions of the detector will be required to maximize the physics acceptance on the largest possible solid angle. To ensure proper trigger performance within the present coverage, the muon system will be completed with new chambers. In figure~\ref{fig:Quadrant} one can see that the existing \acfp{CSC} will be completed by \acfp{GEM} and \acfp{RPC} in the pseudo-rapidity region $1.6<\vert\eta\vert<2.4$ to complete its redundancy as originally scheduled in the CMS Technical Proposal~\cite{CMSTP}.

\begin{figure}[H]
	\centering
	\includegraphics[width=0.7\textwidth]{fig/chapt2/MuonUpgrade-Plans.jpg}
	\caption{\label{fig:Quadrant} A quadrant of the muon system, showing DTs (yellow), RPCs (light blue), and CSCs (green). The locations of new forward muon detectors for Phase-II are contained within the dashed box and indicated in red for GEM stations (ME0, GE1/1, and GE2/1) and dark blue for improved RPC (iRPC) stations (RE3/1 and RE4/1).}
\end{figure}

RPCs are used by the CMS first level trigger for their good timing performances. Indeed, a very good bunch crossing identification can be obtained with the present CMS RPC system, given their fast response of the order of \SI{1}{ns}. In order to contribute to the precision of muon momentum measurements, muon chambers should have a spatial resolution less or comparable to the contribution of multiple scattering~\cite{MUONTDR}. Most of the plausible physics is covered only considering muons with $p_T<$\SI{100}{GeV} thus, in order to match CMS requirements, a spatial resolution of $\mathcal{O}$(few $\mathrm{mm}$) the proposed new RPC stations, as shown by the simulation in figure~\ref{fig:MultiScat}. According to preliminary designs, RE3/1 and RE4/1 readout pitch will be comprised between 3 and \SI{6}{mm} and 5 $\eta$-partitions could be considered.

\begin{figure}[H]
	\centering
	\includegraphics[width=0.6\textwidth]{fig/chapt2/MS_allstations.pdf}
	\caption{\label{fig:MultiScat}  RMS of the multiple scattering displacement as a function of muon $p_T$ for the  proposed forward muon stations. All of the electromagnetic processes such as bremsstrahlung and magnetic field effect are included in the simulation.}
\end{figure}

\clearpage{\pagestyle{empty}\cleardoublepage}

\graphicspath{{chapt_dutch/}{intro/}{chapt2/}{chapt3/}{chapt4/}{chapt5/}{chapt6/}{chapt7/}{chapt8/}}

% Header
\renewcommand\evenpagerightmark{{\scshape\small Chapter 3}}
\renewcommand\oddpageleftmark{{\scshape\small Amplification processes in gaseous detectors}}

\renewcommand{\bibname}{References}

\hyphenation{}

\chapter[Amplification processes in gaseous detectors]%
{Amplification processes in gaseous detectors}
\label{chapt:3}

\section{Signal formation}
\label{sec:signal}

\section{Gas transport parameters}
\label{sec:transport}



\clearpage{\pagestyle{empty}\cleardoublepage}

\graphicspath{{chapt_dutch/}{intro/}{chapt2/}{chapt3/}{chapt4/}{chapt5/}{chapt6/}{chapt7/}}

% Header
\renewcommand\evenpagerightmark{{\scshape\small Chapter 4}}
\renewcommand\oddpageleftmark{{\scshape\small Resistive Plate Chambers}}

\renewcommand{\bibname}{References}

\hyphenation{}

\chapter[Resistive Plate Chambers]%
{Resistive Plate Chambers}
\label{chapt:4}

\section{Principle}
\label{sec:principle}

\section{Rate capability of Resistive Plate Chambers}
\label{sec:RateCapa}

\section{High time resolution}
\label{sec:TimeRes}

\section{Resistive Plate Chambers at CMS}
\label{sec:CMS-RPC}

    \subsection{Pulse processing of CMS RPCs}
    \label{ssec:PulseProc}
	
		Signals induced by cosmic particle in the RPC strips are shaped by standard CMS RPC \acf{FEE} following the scheme of Figure~\ref{fig:DAQ}. On a first stage, analogic signals are amplified and then sent to the \acf{CFD} described in Figure~\ref{fig:CFD}. At the end of the chain, \SI{100}{ns} long pulses are sent in the LVDS output. These output signal are sent on one side to a V1190A \acf{TDC} module from CAEN and on the other to an OR module to count the number of detected signals. Trigger and hit coïncidences are monitored using scalers. The TDC is used to store the data into ROOT files. These files are thus analysed to understand the detectors performance.

			\begin{figure}[!h]
			\begin{subfigure}{\linewidth}
				\centering
				\includegraphics[width = \plotwidth]{fig/chapt5/pulse-processing.pdf}\\
				\caption{\label{fig:DAQ:A}}
			\end{subfigure}
			\begin{subfigure}{\linewidth}
				\centering
				\includegraphics[width = \plotwidth]{fig/chapt5/pulse-processing-2.pdf}
				\caption{\label{fig:DAQ:B}}
			\end{subfigure}
			\caption{\label{fig:DAQ} Signals from the RPC strips are shaped by the FEE described on Figure ~\ref{fig:DAQ:A}. Output LVDS signals are then read-out by a TDC module connected to a computer or converted into NIM and sent to scalers. Figure~\ref{fig:DAQ:B} describes how these converted signals are put in coincidence with the trigger.}
		\end{figure}
		
		\begin{figure}[!h]
			\begin{subfigure}{\linewidth}
				\centering
				\includegraphics[width = \plotwidth]{fig/chapt5/CFD_1.pdf}\\
				\caption{\label{fig:CFD:A}}
			\end{subfigure}
			\begin{subfigure}{\linewidth}
			    \centering
				\includegraphics[width = 1.2\plotwidth]{fig/chapt5/CFD_2.png}
				\caption{\label{fig:CFD:B}}
			\end{subfigure}
			\caption{\label{fig:CFD} Description of the principle of a CFD. A comparison of threshold triggering (left) and constant franction triggering (right) is shown in Figure~\ref{fig:CFD:A}. Constant franction triggering is obtained thanks to zero-crossing technique as explained in Figure~\ref{fig:CFD:B}. The signal arriving at the input of the CFD is split into three components. A first one is delayed and connected to the inverting input of a first comparator. A second component is connected to the noninverting input of this first comparator. A third component is connected to the noninverting input of another comparator along with a threshold value connected to the inverting input. Finally, the output of both comparators is fed through an AND gate.}
		\end{figure}

\clearpage{\pagestyle{empty}\cleardoublepage}

\graphicspath{{chapt_dutch/}{intro/}{chapt2/}{chapt3/}{chapt4/}{chapt5/}{chapt6/}{chapt7/}}

% Header
\renewcommand\evenpagerightmark{{\scshape\small Chapter 5}}
\renewcommand\oddpageleftmark{{\scshape\small Consolidation and Research and Development approval}}

\hyphenation{}

\chapter[Longevity studies and Consolidation of the present \acs{CMS} \acs{RPC} subsystem]%
{Longevity studies and Consolidation of the present \acs{CMS} \acs{RPC} subsystem}
\label{chapt:5}

%****************************************** TESTING DETECTORS UNDER EXTREME CONDITIONS *****************************************************
\section{Testing detectors under extreme conditions}
\label{sec:extreme}

	%****************************************** HL LHC ******************************************************************************
	\subsection{\acl{HL-LHC}}
	\label{ssec:HL-LHC}
	
		The upgrade from LHC to HL-LHC will increase the peak luminosity from 10$^{34}$ \si{cm^{-2}.s^{-1}} to reach 7.5$\times$10$^{34}$ \si{cm^{-2}.s^{-1}}, increasing in the same way the total expected background to which the RPC system will be subjected to. Composed of low energy gammas and neutrons from $p$-$p$ collisions, low momentum primary and secondary muons, puch-through hadrons from calorimeters, and particles produced in the interaction of the beams with collimators, the background will mostly affect the regions of CMS that are the closest to the beam line, i.e. the RPC detectors located in the endcaps.\\
	    The information collected with 2016 data allowed us to understand that the hottest RPC regions are located in the fourth endcap stations. Extrapolating from the data shown in Figure~\ref{Fig:Data-2016}, the maximum rate per unit area under HL-LHC conditions is therefore foreseen to increase to values of the order of \SI{400}{Hz/cm^2} in the chambers of the present muon system. To the \SI{4000}{fb^{-1}} of integrated luminosity, over the 10 years of HL-LHC lifetime, will correspond $\sim$\SI{0.4}{C/cm^2} of integrated charge inside the hottest regions of the detectors, considering the current total delivered luminosity from $p$-$p$ collisions of about \SI{75}{fb^{-1}} and the total integrated charge estimated to be about \SI{5.8}{mC/cm^2} in the endcap.\\
	    During Run-I, the RPC system provided stable operation and excellent performance and did not show any aging effects. In the past, extensive long-term tests were carried out at several gamma and neutron facilities certifying the detector performance up to values of dose, charge and fluence close to those expected after ten years of HL-LHC operation. Both full size and small prototype RPCs have been irradiated with photons up to an integrated charge of $\sim$\SI{0.05}{C/cm^2} and $\sim$\SI{0.4}{C/cm^2}, respectively~\cite{GIF2004,AGING2009}.\\
	    
	    \begin{figure}[!h]
		    \begin{subfigure}{\plotwidth}
	    		\begin{center}
		    		\includegraphics[height=6cm]{fig/RE4-rates-2016-data.png}\\
		        	\caption{\label{Fig:Rates-2016}}
	    		\end{center}
		    \end{subfigure}
		    \begin{subfigure}{\plotwidth}
	    		\begin{center}
		    		\includegraphics[height=6cm]{fig/Endcap-TOT-Qint-data.png}\\
		        	\caption{\label{Fig:Qint-2016}}
	    		\end{center}
		    \end{subfigure}
	        \caption{\label{Fig:Data-2016} Figure~\ref{Fig:Rates-2016} represent the RPC rate measured in 2016 in $p$-$p$ collision runs as function of the instantaneous luminosity. Every point corresponds to a particular run. Figure~\ref{Fig:Qint-2016} represent the integrated charge for Endcap. The integrated charge in years is shown in blue. The red curve shows the cumulative evolution of the integrated charge in time.}
	    \end{figure}
	    
	    In this perspective, studying the performance of the present system up to an integrated charge of $\sim$\SI{1.2}{C/cm^2}, 3 times higher than what expected for 10 years of operation of HL-LHC, and background hit rates of \SI{1200}{Hz/cm^2}, 3 times stronger than what expected from the designed peak luminosity, and identifying possible long-term aging effects are necessary steps to take to insure that the RPCs will be able to cope with the high radiation conditions.\\

	%****************************************** GAMMA IRRADIATION FACILITIES ********************************************************
	\subsection{The Gamma Irradiation Facilities}
	\label{ssec:Facilities}
		%****************************************** OLD GIF ******************************************************************
		\subsubsection{GIF}
		\label{sssec:GIF}
		
			GIF was a test area located in the SPS West Area at the downstream end of the X5 test beam in which particle detectors were exposed to a particle beam in presence of an adjustable background flux of photons~\cite{GIFacility}. This facility's goal was to simulate background conditions these detectors would suffer in their operating environment at the LHC. A schematic layout of the GIF zone is shown in Figure ~\ref{fig:GIFLayout}. Photons are produced by a strong radioactive source of $^{137}$Cs installed in the upstream part of the zone inside a lead container, which includes a collimator, designed to irradiate a \SIsurface{6}{6}{\meter} area at \SI{5}{\meter} distance from the source. A thin lens-shaped lead filter renders the outcoming flux uniform in the vertical plane orthogonal to the beam direction. Control of the photon rate is achieved by further lead filters allowing the maximum rate to be limited and to vary within a range of four orders of magnitude. The $\gamma$ source is located in a rectangular lead container.
	
			\begin{figure}[!h]
				\begin{center}
					\includegraphics[width = \plotwidth]{fig/GIF.pdf}\\
					\caption{\label{fig:GIFLayout} Layout of the test beam zone called X5c GIF at CERN. Photons from the radioactive source produce a sustained high rate of random hits over the whole area. The zone is surrounded by \SI{8}{\meter} high and \SI{80}{\cm} thick concrete walls. Access is possible through three entry points. Two access doors for personnel and one large gate for material. A crane allows installation of heavy equipment in the area.}
				\end{center}
			\end{figure}
			
			As described on Figure~\ref{fig:CsSource}, the $^{137}$Cs source with a half-life of 30 years and an activity of \SI{740}{\GBq}, as measured on the 5$^{th}$ March 1997, emits a \SI{662}{\keV} photon in 85\% of the decays. The principal collimator hole provides a pyramidal aperture of \SI{74}{\degree}$\times$ \SI{74}{\degree} solid angle. This provides a photon flux in a pyramidal volume of \SI{5}{\meter} maximum length along the beam axis.

				\begin{figure}[!h]
				\begin{center}
					\includegraphics[width = \plotwidth]{fig/Cs137.pdf}\\
					\caption{\label{fig:CsSource} $^{137}$Cs decays by $\beta^-$ emission to the ground state of $^{137}$Ba (BR = 5.64\%) and via the \SI{662}{\keV} isomeric level of $^{137}$Ba (BR = 94.36\%) whose half-life is 2.55 min.}
				\end{center}
			\end{figure}
			
			Particle detectors under test are then placed within the pyramidal volume in front of the source, perpendicularly to the beam line in order to profit from the homogeneous photon flux. Adjusting the background flux of photons can then be done by using the filters and choosing the position of the detectors with respect to the source.
			
		
		%****************************************** NEW GIF ******************************************************************
		\subsubsection{GIF++}
		\label{sssec:GIF++}
		
			GIF++ is a test area located in the SPS North Area at the downstream end of the H4 test beam. In this facility, particle detectors are tested using a particle beam in presence of an adjustable background flux of photons. The goal is to simulate background conditions these detectors would suffer in their operating environment at the HL-LHC. Gamma photons are produced by a strong radioactive $^{137}$Cs source installed in the center part of the zone, with an activity of \SI{13.1}{TBq}, emitting \SI{662}{\keV} photons. A thin lens-shaped lead filter renders the outcoming flux uniform in the vertical plane orthogonal to the beam direction. Control of the photon rate is achieved by using a set of filters allowing the maximum rate to be limited and to vary within a range of four orders of magnitude. The H4 beam, composed of muons with a momentum of about \SI{150}{GeV/c}, passes through the GIF++ zone and is used to study the performance of the detectors. Its flux is of \SI{104}{particles/s/\square\cm} focused in an area similar to \SIsurface{10}{10}{\cm}.
		
		

%****************************************** PRIMARILY TESTS AT GIF *************************************************************************
\section{Preliminary tests at GIF}
\label{sec:GIFtests}
				
	\subsection{\acl{RPC} test setup}
	\label{ssec:RPCSetup}
	
		During summer 2014, preliminary tests have been conducted in the GIF area on a newly produced RE4/2 chamber labelled RE-4-2-BARC-161. This chamber has been placed into a trolley covered with a tent. The position of the RPC inside the tent and of the tent related to the source is described in Figure~\ref{fig:GIFSetup}. To test this CMS RPC, three different absorber settings were used. First of all, measurements were done with fully opened source. Then, to complete this preliminary study, the gamma flux has been attenuated by a factor 2 and a factor 5. The expected gamma flux at the level of our detector will be discussed in subsection~\ref{ssec:gFlux}.

				\begin{figure}[!h]
			\begin{subfigure}{0.5\linewidth}
				\includegraphics[width = 0.5\plotwidth]{fig/position-side.pdf}
				\caption{\label{fig:GIFSetup:A}}
			\end{subfigure}
			\begin{subfigure}{0.5\linewidth}
				\includegraphics[width = 0.5\plotwidth]{fig/position-top.pdf}
				\caption{\label{fig:GIFSetup:B}}
			\end{subfigure}
			\caption{\label{fig:GIFSetup} Description of the RPC setup. Dimensions are given in \si{\mm}. Figure~\ref{fig:GIFSetup:A} provides a side view of the setup while Figure~\ref{fig:GIFSetup:B} shows a top view. A tent containing RPCs is placed at \SI{1720}{\mm} from the source container. The source is situated in the center of the container. RE-4-2-BARC-161 chamber is \SI{160}{\mm} inside the tent. This way, the distance between the source and the chambers plan is \SI{2060}{\mm}.}
		\end{figure}

		\begin{figure}[!h]
			\begin{center}
				\includegraphics[width = \plotwidth]{fig/Setup.jpg}\\
				\caption{\label{fig:RPCSetup} RE-4-2-BARC-161 chamber is inside the tent as described in Figure~\ref{fig:GIFSetup}. In the top right, the two scintillators used as trigger can be seen. This trigger system has an inclination of \SI{10}{\degree} relative to horizontal and is placed above half-partition B2 of the RPCs. PMT electronics are shielded thanks to lead blocks placed in order to protect them without stopping photons from going through the scintillators and the chamber.}
			\end{center}
		\end{figure}
		
		At the time of the tests, the beam not being operationnal anymore, a trigger composed of 2 plastic scintillators has been placed in front of the setup with an inclination of \SI{10}{\deg} \textit{\textbf{(this has to be first confirmed by the simulation - I will adjust in consequence cause it has never been precisely measured)}} with respect to the detector plane in order to look at cosmic muons. Using this particular trigger layout, shown on Figure~\ref{fig:RPCSetup}, leads to a cosmic muon hit distribution into the chamber similar to the one in Figure~\ref{fig:HitProf}. Measured without gamma irradiation, two peaks can be seen on the profil of partition B, centered on strips 52 and 59. Sub-section~\ref{ssec:GeoAcc} will help us undertand that these two peaks are due respectively to forward and backward coming cosmic particles where forward coming particles are first detected by the scintillators and then the RPC while the backward coming muons are first detected in the RPC.

		\begin{figure}[!h]
			\begin{center}
				\includegraphics[width = 1.2\plotwidth]{fig/Data-21-profile.pdf}\\
				\caption{\label{fig:HitProf} Hit distributions over all 3 parttions of RE-4-2-BARC-161 chamber is showed on these plots. Top, middle and bottom figures respectively correspond to partitions A, B, and C. These plots show that some events still occur in other half-partitions than B2, which corresponds to strips 49 to 64, in front of which the trigger is placed, contributing to the inefficiency of detection of cosmic muons. In the case of partitions A and C, the very low amount of data can be interpreted as noise. On the other hand, it is clear that a little portion of muons reach the half-partition B1, corresponding to strips 33 to 48.}
			\end{center}
		\end{figure}
		
	\subsection{\acl{DAQ}}
	\label{ssec:GIFDAQ}
	
		Signals induced by cosmic particle in the RPC strips are shaped by standard CMS RPC \acf{FEE} following the scheme of Figure~\ref{fig:DAQ}. On a first stage, analogic signals are amplified and then sent to the \acf{CFD} described in Figure~\ref{fig:CFD}. At the end of the chain, \SI{100}{ns} long pulses are sent in the LVDS output. These output signal are sent on one side to a V1190A \acf{TDC} module from CAEN and on the other to an OR module to count the number of detected signals. Trigger and hit coïncidences are monitored using scalers. The TDC is used to store the data into ROOT files. These files are thus analysed to understand the detectors performance.

			\begin{figure}[!h]
			\begin{subfigure}{\linewidth}
				\begin{center}
					\includegraphics[width = \plotwidth]{fig/pulse-processing.pdf}\\
					\caption{\label{fig:DAQ:A}}
				\end{center}
			\end{subfigure}
			\begin{subfigure}{\linewidth}
				\begin{center}
					\includegraphics[width = \plotwidth]{fig/pulse-processing-2.pdf}
					\caption{\label{fig:DAQ:B}}
				\end{center}
			\end{subfigure}
			\caption{\label{fig:DAQ} Signals from the RPC strips are shaped by the FEE described on Figure ~\ref{fig:DAQ:A}. Output LVDS signals are then read-out by a TDC module connected to a computer or converted into NIM and sent to scalers. Figure~\ref{fig:DAQ:B} describes how these converted signals are put in coincidence with the trigger.}
		\end{figure}
		
		\begin{figure}[!h]
			\begin{subfigure}{\linewidth}
				\begin{center}
					\includegraphics[width = \plotwidth]{fig/CFD_1.pdf}\\
					\caption{\label{fig:CFD:A}}
				\end{center}
			\end{subfigure}
			\begin{subfigure}{\linewidth}
				\begin{center}
					\includegraphics[width = 1.2\plotwidth]{fig/CFD_2.png}
					\caption{\label{fig:CFD:B}}
				\end{center}
			\end{subfigure}
			\caption{\label{fig:CFD} Description of the principle of a CFD. A comparison of threshold triggering (left) and constant franction triggering (right) is shown in Figure~\ref{fig:CFD:A}. Constant franction triggering is obtained thanks to zero-crossing technique as explained in Figure~\ref{fig:CFD:B}. The signal arriving at the input of the CFD is split into three components. A first one is delayed and connected to the inverting input of a first comparator. A second component is connected to the noninverting input of this first comparator. A third component is connected to the noninverting input of another comparator along with a threshold value connected to the inverting input. Finally, the output of both comparators is fed through an AND gate.}
		\end{figure}

	%****************************************** GIF GEOMETRICAL ACCEPTANCE **********************************************************
	\subsection{Geometrical acceptance of the setup layout to cosmic muons}
	\label{ssec:GeoAcc}
				
		In order to profit from a constant gamma irradiation, the detectors inside of the GIF bunker need to be placed in a plane orthogonal to the beam line. The muon beam that used to be available was meant to test the performance of detectors under test. This beam not being active anymore, another solution to test detector performance had to be used. Thus, it has been decided to use cosmic muons detected through a telescope composed of two scintillators. Lead blocks were used as shielding to protect the photomultipliers from gammas as can be seen from Figure~\ref{fig:RPCSetup}.
					
		An inclination has been given to the cosmic telescope to maximize the muon flux. A good compromise had to be found between good enough muon flux and narrow enough hit distribution to be sure to contain all the events into only one half partitions as required from the limited available readout hardware. Nevertheless, a consequence of the misplaced trigger, that can be seen as a loss of events in half-partition B1 in Figure~\ref{fig:HitProf}, is an inefficiency. Nevertheless, the innefficiency of approximately \SI{20}{\%} highlighted in Figure~\ref{fig:EffCompar} by comparing the performance of chamber BARC-161 in 904 and at GIF without irradiation seems too important to be explained only by the geometrical acceptance of the setup itself. Simulations have been conducted to show how the setup brings inefficiency.
	
		\begin{figure}[!h]
			\begin{center}
				\includegraphics[width = \plotwidth]{fig/Comparison.pdf}
			\end{center}
			\caption{\label{fig:EffCompar} Results are derived from data taken on half-partition B2 only. On the 18$^{th}$ of June 2014, data has been taken on chamber RE-2-BARC-161 at building 904 (Prevessin Site) with cosmic muons providing us a reference efficiency plateau of \numerror{97.54}{0.15}\% represented by a black curve. A similar measurement has been done at GIF on the 21$^{st}$ of July with the same chamber giving a plateau of \numerror{78.52}{0.94}\% represented by a red curve.}
		\end{figure}
		
		\subsubsection{Description of the simulation layout}
		\label{sssec:SimLayout}
		
			The layout of GIF setup has been reproduced and incorporated into a \acf{MC} simulation to study the influence of the disposition of the telescope on the final distribution measured by the RPC. A 3D view of the simulated layout is given into Figure~\ref{fig:SimGIFLay}. Muons are generated randomly in a horizontal plane located at a height corresponding to the lowest point of the PMTs. This way, the needed size of the plane in order to simulate events happening at very big azimutal angles (i.e. $\theta\approx\pi$) can be kept relatively small. The muon flux is designed to follow the usual $cos^2\theta$ distribution for cosmic particle. The goal of the simulation is to look at muons that pass through the muon telescope composed of the two scintilators and define their distribution onto the RPC plane. During the reconstruction, the RPC plane is then divided into its strips and each muon track is assigned to a strip.
		
			\begin{figure}[!h]
				\begin{subfigure}{\linewidth}
					\begin{center}
						\includegraphics[width = \plotwidth]{fig/GIFSetup-SimA.png}\\
						\caption{\label{fig:SimGIFLay:A}}
					\end{center}
				\end{subfigure}
				\begin{subfigure}{\linewidth}
					\begin{center}
						\includegraphics[width = \plotwidth]{fig/GIFSetup-SimB.png}
						\caption{\label{fig:SimGIFLay:B}}
					\end{center}
				\end{subfigure}
				\caption{\label{fig:SimGIFLay} Representation of the layout used for the simulations of the test setup. The RPC is represented as a yellow trapezoid while the two scintillators as blue cuboids looking at the sky. A green plane corresponds to the muon generation plane within the simulation. Figure~\ref{fig:GIFSetup:A} shows a global view of the simulated setup. Figure~\ref{fig:GIFSetup:B} shows a zommed view that allows to see the 2 scintillators as well as the full RPC plane.}
			\end{figure}
			
			In order to further refine the quality of the simulation and understand deeper the results the dependance of the distribution has been studied for a range of telescope inclinations. Moreover, the threshold applied on the PMT signals has been included into the simulation in the form of a cut. In the approximation of uniform scintillators, it has been considered that the threshold can be understood as the minimum distance particles need to travel through the scintillating material to give a strong enough signal. Particles that travel a distance smaller than the set "threshold" are thus not detected by the telescope and cannot trigger the data taking. Finally, the FEE threshold also has been considered in a similar way. The mean momentum of horizontal cosmic rays is higher than those of vertical ones but the stopping power of matter for momenta ranging from \SI{1}{GeV} to \SI{1}{TeV} stays comparable. It is then possible to assume that the mean number of primary $e^-$/ion pairs per unit length will stay similar and thus, depending on the applied discriminator threshold, muons with the shortest path through the gas volume will deposit less charge and induce a smaller signal on the pick-up strips that could eventually not be detected. These two thresholds also restrain the overall geometrical acceptance of the system.
			
		\subsubsection{Simulation procedure}
		\label{sssec:SimProc}
		
			The simulation software has been designed using C++ and the output data is saved into ROOT histograms. Simulations start for a threshold $T_{scint}$ varying in a range from 0 to \SI{45}{mm} in steps of \SI{5}{mm}, where $T_{scint}=$ \SI{0}{mm} corresponds to the case where there isn't any threshold apply on the input signal while $T_{scint}=$ \SI{45}{mm}, which is the scintillator thickness, is the case where muons cannot arrive orthogonally onto the scintillator surface. For a given $T_{scint}$, a set of \acs{RPC} thresholds are considered. The RPC threshold, $T_{RPC}$ varies from \SI{2}{mm}, the thickness of the gas volume, to \SI{3}{mm} in steps of \SI{0.25}{mm}. For each ($T_{scint}$;$T_{RPC}$) pair, $N_{\mu}=10^8$ muons are randomly generated inside the muon plane described in the previous paragraph with an azimutal angle $\theta$ chosen to follow a $cos^2\theta$ distribution. 
			
			Planes are associated to each surface of the scintillators. Knowing muon position into the muon plane and its direction allows us, by assuming that muons travel in a straight line, to compute the intersection of the muon track with these planes. Applying conditions to the limits of the surfaces of the scintillator faces then gives us an answer to weither or not the muon passed through the scintillators. In the case the muon has indeed passed through the telescope, the path through each scintillator is computed and muons whose path was shorter than $T_{scint}$ are rejected and are thus considered as having not interacted with the setup.
			
			On the contrary, if the muon is labeled as good, its position within the RPC plane is computed and the corresponding strip, determined by geometrical tests in the case the distance through the gas volume was enough not to be rejected because of $T_{RPC}$, gets a hit and several histograms are filled in order to keep track of the generation point on the muon plane, the intersection points of the reconstructed muons within the telescope, or on the RPC plane, the path traveled through each individual scintillator or the gas volume, as well as other histograms. Moreover, muons fill different histograms weither they are forward or backward coming muons. They are discriminated according to their direction components. When a muon is generated, an $(x,y,z)$ position is assigned into the muon plane as well as a ($\theta$;$\phi$) pair that gives us the direction it's coming from. This way, muons satisfying the condition $0\leq\phi<\pi$ are designated as backward coming muons while muons satisfying $\pi\leq\phi<2\pi$ as forward coming muons.
			
			This simulation is then repeated for different telescope inclinations ranging in between 4 and \SI{20}{\degree} and varying in steps of \SI{2}{\degree}. Due to this inclination and to the vertical position of the detector under test, the muon distribution reconstructed in the detector plane is asymmetrical. The choice as been made to chose a skew distribution formula to fit the data built as the multiplication of gaussian and sigmoidal curves together. A typical gaussian formula is given as ~\ref{for:gaus} and has three free parameters as $A_g$, its amplitude, $\bar{x}$, its mean value and $\sigma$, its root mean square. Sigmoidal curves as given by formula~\ref{for:sigm} are functions converging to $0$ and $A_s$ as $x$ diverges. The inflexion point is given as $x_i$ and $\lambda$ is proportional to the slope at $x = x_i$. In the limit where $\lambda\longrightarrow\infty$, the sigmoid becomes a step function.
			
			\begin{center}
				\begin{equation}
				\label{for:gaus}
					g(x) = A_g e^{\frac{-(x-\bar{x})^2}{2\sigma^2}}
				\end{equation}
			\end{center}
			
			\begin{center}
				\begin{equation}
				\label{for:sigm}
					s(x) = \frac{A_s}{1+e^{-\lambda(x-x_i)}}
				\end{equation}
			\end{center}
			
		Finally, a possible representation of a skew distribution is given by formula~\ref{for:skew} and is the product of \ref{for:gaus} and \ref{for:sigm}. Naturally, here $A_{sk} = A_g \times A_s$ and represents the theoretical maximum in the limit where the skew tends to a gaussian function.
			
			\begin{center}
				\begin{equation}
				\label{for:skew}
					sk(x) = g(x)\times s(x) = A_{sk}\frac{e^{\frac{-(x-\bar{x})^2}{2\sigma^2}}}{1+e^{-\lambda(x-x_i)}}
				\end{equation}
			\end{center}
			
		\subsubsection{Results}
		\label{sssec:SimRes}
			
			\paragraph{Influence of $\mathbf{T_{scint}}$ on the muon distribution}
			
			\paragraph{Influence of $\mathbf{T_{RPC}}$ on the muon distribution}
		
			\paragraph{Influence of the telescope inclination on the muon distribution}
			
			\paragraph{Comparison to data taken at GIF without irradiation}
		
	%****************************************** PHOTON FLUX AT GIF ******************************************************************
	\subsection{Photon flux at \acs{GIF}}
	\label{ssec:gFlux}
			
		%****************************************** EXPECTATIONS FROM SIMULATIONS ********************************************
		\subsubsection{Expectations from simulations}
		\label{sssec:Simulations}
		
			In order to understand and evaluate the $\gamma$ flux in the GIF area, simulations had been conducted in 1999 and published by S. Agosteo et al ~\cite{AGOSTEO1999}. Table~\ref{tab:Sim1997} presented in this article gives us the $\gamma$ flux for different distances $D$ to the source. This simulation was done using GEANT and a \acf{MCNP} transport code, and the flux $F$ is given in number of $\gamma$ per unit area and unit time along with the estimated error from these packages expressed in \%.
			
			\begin{table}[!h]
				\hspace*{-1.1cm}
				\begin{tabular}{|*{5}{c|}}
					\hline
					Nominal & \multicolumn{4}{c|}{Photon flux $F$ [\si{\second^{-1}\cm^{-2}}]} \\
					\cline{2-5}
					ABS & at $D$ = \SI{50}{\cm} & at $D$ = \SI{155}{\cm} & at $D$ = \SI{300}{\cm} & at $D$ = \SI{400}{\cm} \\
					\hline
					1 & $0.12 \cdot 10^8 \pm 0.2\%$ & $0.14 \cdot 10^7 \pm 0.5\%$ & $0.45 \cdot 10^6 \pm 0.5\%$ & $0.28 \cdot 10^6 \pm 0.5\%$ \\
					\hline
					2 & $0.68 \cdot 10^7 \pm 0.3\%$ & $0.80 \cdot 10^6 \pm 0.8\%$ & $0.25 \cdot 10^6 \pm 0.8\%$ & $0.16 \cdot 10^6 \pm 0.6\%$ \\
					\hline
					5 & $0.31 \cdot 10^7 \pm 0.4\%$ & $0.36 \cdot 10^6 \pm 1.2\%$ & $0.11 \cdot 10^6 \pm 1.2\%$ & $0.70 \cdot 10^5 \pm 0.9\%$ \\
					\hline
				\end{tabular}
				\caption{\label{tab:Sim1997} Total photon flux ($E\gamma \leq$ \SI{662}{\keV}) with statistical error predicted considering a $^{137}$Cs activity of \SI{740}{\GBq} at different values of the distance $D$ to the source along the x-axis of irradiation field~\cite{AGOSTEO1999}.}
			\end{table}
			
			\begin{figure}[!h]
				\begin{center}
					\includegraphics[width = \plotwidth]{fig/simulated_flux.pdf}\\
					\caption{\label{fig:Sim1997} $\gamma$ flux $F(D)$ is plot using values from table~\ref{tab:Sim1997}. As expected, the plot shows similar attenuation behaviours with increasing distance for each absorption factors.}
				\end{center}
			\end{figure}
			
			The simulation doesn't directly provides us with an estimated flux at the level of our RPC. First of all, it is needed to extract the value of the flux from the available data contained in the original paper and then to estimate the flux in 2014 at the time the experimentation took place. Figure~\ref{fig:Sim1997} that contains the data from Table~\ref{tab:Sim1997}. In the case of a pointlike source emiting isotrope and homogeneous gamma radiations, the gamma flux $F$ at a distance $D$ to the source with respect to a reference point situated at $D_0$ where a known flux $F_0$ is measured will be expressed like in Equation~\ref{eq:Flux}, assuming that the flux decreases as $1/D^2$, where $c$ is a fitting factor.
			
			\begin{equation}
				F^{ABS} = F_0^{ABS} \times \left( \frac{c D_0}{D} \right)^2
				\label{eq:Flux}
			\end{equation}
			
			By rewriting Equation~\ref{eq:Flux}, it comes that :
			
			\begin{eqnarray}
				c & = & \frac{D}{D_0}\sqrt{\frac{F^{ABS}}{F_0^{ABS}}}\label{eq:Factor}\\
				\Delta c & = & \frac{c}{2}\left(\frac{\Delta F^{ABS}}{F^{ABS}}+\frac{\Delta F^{ABS}_0}{F^{ABS}_0}\right)\label{eq:FactorErr}
			\end{eqnarray}
			
			Finally, using Equation~\ref{eq:Factor} and the data in Table~\ref{tab:Sim1997} with $D_0=$ \SI{50}{cm} as reference point, we can build Table~\ref{tab:CorrFactor}. It is interesting to note that $c$ for each value of $D$ doesn't depend on the absorption factor.
			
			\begin{table}[!h]
				\begin{center}
					\begin{tabular}{|*{4}{c|}}
						\hline
						Nominal & \multicolumn{3}{c|}{Correction factor $c$} \\
						ABS & at $D$ = \SI{155}{\cm} & at $D$ = \SI{300}{\cm} & at $D$ = \SI{400}{\cm} \\
						\hline
						1 & $1.059 \pm 0.70\%$ & $1.162 \pm 0.70\%$ & $1.222 \pm 0.70\%$ \\
						\hline
						2 & $1.063 \pm 1.10\%$ & $1.150 \pm 1.10\%$ & $1.227 \pm 0.90\%$ \\
						\hline
						5 & $1.056 \pm 1.60\%$ & $1.130 \pm 1.60\%$ & $1.202 \pm 1.30\%$ \\
						\hline
					\end{tabular}
					\caption{\label{tab:CorrFactor} Correction factor c is computed thanks to formulae~\ref{eq:Factor} taking as reference $D_0 =$ \SI{50}{cm} and the associated flux $F_0^{ABS}$ for each absorption factor available in table~\ref{tab:Sim1997}.}
				\end{center}
			\end{table}
			
			For the range of $D/D_0$ values available, it is possible to use a simple linear fit to get the evolution of $c$. The linear fit will then use only 2 free parameters, $a$ and $b$, as written in Equation~\ref{eq:LinearApprox}. This gives us the results showed in Figure~\ref{fig:CorrFactor}. Figure~\ref{fig:CorrFactor:B} confirms that using only a linear fit to extract $c$ is enough as the evolution of the rate that can be obtained superimposes well on the simulation points.
			
			\begin{eqnarray}
				c\left(\frac{D}{D_0}\right) & = & a\frac{D}{D_0} + b\label{eq:LinearApprox}\\
				F^{ABS} & = & F^{ABS}_0 \left( a + \frac{bD_0}{D} \right)^2\label{eq:FluxLinearAp}\\
				\Delta F^{ABS} & = & F^{ABS} \left[\frac{\Delta F^{ABS}_0}{F^{ABS}_0} + 2\frac{\Delta a + \Delta b\frac{D_0}{D}}{a + \frac{bD_0}{D}}\right]\label{eq:FluxLinearApErr}
			\end{eqnarray}
			
			\begin{figure}[!h]
				\begin{subfigure}{\linewidth}
					\begin{center}
						\includegraphics[width = \plotwidth]{fig/flux_correction.pdf}\\
						\caption{\label{fig:CorrFactor:A}}
					\end{center}
				\end{subfigure}
				\begin{subfigure}{\linewidth}
					\begin{center}
						\includegraphics[width = \plotwidth]{fig/correction_model.pdf}
						\caption{\label{fig:CorrFactor:B}}
					\end{center}
				\end{subfigure}
				\caption{\label{fig:CorrFactor} Figure~\ref{fig:CorrFactor:A} shows the linear approximation fit done via formulae~\ref{eq:LinearApprox} on data from table~\ref{tab:CorrFactor}. Figure~\ref{fig:CorrFactor:B} shows a comparison of this model with the simulated flux using a and b given in figure~\ref{fig:CorrFactor:A} in formulae ~\ref{eq:Flux} and the reference value $D_0 = 50 cm$ and the associated flux for each absorption factor $F_0^{ABS}$ from table~\ref{tab:Sim1997}}
			\end{figure}
			
			In the case of the 2014 GIF tests, the RPC plane is located at a distance $D=$\SI{206}{cm} to the source. Moreover, to estimate the strength of the flux in 2014, it is necessary to consider the nuclear decay through time assiciated to the Cesium source whose half-life is well known ($t_{1/2}=$ \SIerror{30.05}{0.08}{y}). The very first source activity measurement has been done on the $5^{th}$ of March 1997 while the GIF tests where done in between the $20^{th}$ and the $31^{st}$ of August 2014, i.e. at a time $t=$ \SIerror{17.47}{0.02}{y} resulting in an attenuation of the activity from \SI{740}{GBq} in 1997 to \SI{494}{GBq} in 2014. All the needed information to extrapolate the flux through our detector in 2014 has now been assembled, leading to the Table~\ref{tab:extra2014}. It isinteresting to note that for a common RPC sensitivity to $\gamma$ of $2 \cdot 10^{-3}$, the order of magnitude of the estimated hit rate per unit area is of the order of the \si{kHz} for the fully opened source. Moreover, taking profit of the two working absorbers, it will be possible to scan background rates at \SI{0}{Hz}, $\sim$\SI{300}{Hz} as well as $\sim$\SI{600}{Hz}. Without source, a good estimate of the intrinsic performance will be available. Then at \SI{300}{Hz}, the goal will be to show that the detectors fulfill the performance certification of CMS RPCs. Then a first idea of the performance of the detectors at higher background will be provided with absorbtion factors 2 ($\sim$\SI{600}{Hz}) and 1 (no absorbtion). \textbf{\textit{[Here I will also put a reference to the plot showing the estimated background rate at the level of RE3/1 in the case of HL-LHC but this one being in another chapter, I will do it later.]}}
			
			\begin{table}[!h]
				\hspace*{-2cm}
				\begin{tabular}{|*{5}{c|}}
					\hline
					Nominal & \multicolumn{3}{c|}{Photon flux $F$ [\si{\second^{-1}\cm^{-2}}]} & Hit rate/unit area [\si{Hz.cm^{-2}}] \\
					ABS & at $D_0^{1997}$ = \SI{50}{\cm} & at $D^{1997}$ = \SI{206}{\cm} & at $D^{2014}$ = \SI{206}{\cm} & at $D^{2014}$ = \SI{206}{\cm} \\
					\hline
					1 & $0.12 \cdot 10^8 \pm 0.2\%$ & $0.84 \cdot 10^6 \pm 0.3\%$ & $0.56 \cdot 10^6 \pm 0.3\%$ & $1129 \pm 32$ \\
					\hline
					2 & $0.68 \cdot 10^7 \pm 0.3\%$ & $0.48 \cdot 10^6 \pm 0.3\%$ & $0.32 \cdot 10^6 \pm 0.3\%$ & $640 \pm 19$ \\
					\hline
					5 & $0.31 \cdot 10^7 \pm 0.4\%$ & $0.22 \cdot 10^6 \pm 0.3\%$ & $0.15 \cdot 10^6 \pm 0.3\%$ & $292 \pm 9$ \\
					\hline
				\end{tabular}
				\caption{\label{tab:extra2014} The data at $D_0$ in 1997 is taken from~\cite{AGOSTEO1999}. In a second step, using Equations~\ref{eq:FluxLinearAp} and~\ref{eq:FluxLinearApErr}, the flux at $D$ can be estimated in 1997. Then, taking into account the attenuation of the source activity, the flux at $D$ can be estimated at the time of the tests in GIF in 2014. Finally, assuming a sensitivity of the RPC to $\gamma$ $s = 2 \cdot 10^{-3}$, an estimation of the hit rate per unit area is obtained.}
			\end{table}

		%****************************************** DOSE MEASUREMENTS ********************************************************
		\subsubsection{Dose measurements}
		\label{sssec:Dose}
			
			\begin{figure}[!h]
				\begin{center}
					\includegraphics[width = \plotwidth]{fig/GIF-fluxes.pdf}\\
					\caption{\label{fig:Dose} Dose measurements has been done in a plane corresponding to the tents front side. This plan is \SI{1900}{\mm} away from the source. As explained in the first chapter, a lens-shaped lead filter provides a uniform photon flux in the vertical plan orthogonal to the beam direction. If the second line of measured fluxes is not taken into account because of lower values due to experimental equipments in the way between the source and the tent, the uniformity of the flux is well showed by the results.}
				\end{center}
			\end{figure}

%****************************************** CONSOLIDATION TESTS AT GIF++ *******************************************************************
\newpage

\section{Longevity tests at \acs{GIF++}}
\label{sec:GIFtests}

		
	    This study implies a monitoring of the performance of the detectors probed using a high intensity muon beam in a irradiated environment by periodically measuring their rate capability, the dark current running through them and the bulk resistivity of the Bakelite composing their electrodes. GIF++, with its very intense $^{137}$Cs source, provides the perfect environment to perform such kind of tests. Assuming a maximum acceleration factor of 3, it is expected to accumulate the equivalent charge in 1.7 years.\\
	    As the maximum background is found in the endcap, the choice naturally was made to focus the GIF++ longevity studies on endcap chambers. Most of the RPC system was installed in 2007. Nevertheless, the large chambers in the fourth endcap (RE4/2 and RE4/3) have been installed during LS1 in 2014. The Bakelite of these two different productions having different properties, four spare chambers of the present system were selected, two RE2,3/2 spares and two RE4/2 spares. Having two chambers of each type allows to always keep one of them non irradiated as reference, the performance evolution of the irradiated chamber being then compared through time to the performance of the non irradiated one.\\
	    The performance of the detectors under different level of irradiation is measured periodically during dedicated test beam periods using the H4 muon beam. In between these test beam periods, the two RE2,3/2 and RE4/2 chambers selected for this study are irradiated by the $^{137}$Cs source in order to accumulate charge and the gamma background is monitored, as well as the currents. The two remaining chambers are kept non-irradiated as reference detectors. Due to the limited gas flow in GIF++, the RE4 chamber remained non-irradiated until end of November 2016 where a new mass flow controller has been installed allowing for bigger volumes of gas to flow in the system.\\
	     Figures~\ref{Fig:Eff-vs-Rate} and \ref{Fig:WP-vs-Rate} give us for different test beam periods, and thus for increasing integrated charge through time, a comparison of the maximum efficiency, obtained using a sigmoid-like function, and of the working point of both irradiated and non irradiated chambers~\cite{SIGMOID2005}. No aging is yet to see from this data, the shifts in $\gamma$ rate per unit area  in between irradiated and non irradiated detectors and RE2 and RE4 types being easily explained by a difference of sensitivity due to the various Bakelite resistivities of the HPL electrodes used for the electrode production.\\
	     Collecting performance data at each test beam period allows us to extrapolate the maximum efficiency for a background hit rate of \SI{300}{Hz/cm^2} corresponding to the expected HL-LHC conditions. Aging effects could emerge from a loss of efficiency with increasing integrated charge over time, thus Figure~\ref{Fig:Eff-vs-Qint} helps us understand such degradation of the performance of irradiated detectors in comparison with non irradiated ones. The final answer for an eventual loss of efficiency is given in Figure~\ref{Fig:Eff-Bef-Aft} by comparing for both irradiated and non irradiated detectors the efficiency sigmoids before and after the longevity study. Moreover, to complete the performance information, the Bakelite resistivity is regularly measured thanks to $Ag$ scans (Figure~\ref{Fig:Res-vs-Qint}) and the noise rate is monitored weekly during irradiation periods (Figure~\ref{Fig:Noise-vs-Qint}). At the end of 2016, no signs of aging were observed and further investigation is needed to get closer to the final integrated charge requirements proposed for the longevity study of the present CMS RPC sub-system.\\
	    
	    
	    \begin{figure}[!h]
	    	\begin{subfigure}{0.5\linewidth}
	    		\includegraphics[width=\linewidth]{fig/Eff-vs-Rate-RE2.png}\\
	        	\caption{\label{Fig:Eff-vs-Rate:RE2}}
	    	\end{subfigure}
	    	\begin{subfigure}{0.5\linewidth}
	    		\includegraphics[width=\linewidth]{fig/Eff-vs-Rate-RE4.png}\\
	        	\caption{\label{Fig:Eff-vs-Rate:RE4}}
	    	\end{subfigure}
	        \caption{\label{Fig:Eff-vs-Rate} Evolution of the maximum efficiency for RE2 (\ref{Fig:Eff-vs-Rate:RE2}) and RE4 (\ref{Fig:Eff-vs-Rate:RE4}) chambers with increasing extrapolated $\gamma$ rate per unit area at working point. Both irradiated (blue) and non irradiated (red) chambers are shown.}
	    \end{figure}
	    
	    \begin{figure}[!h]
	    	\begin{subfigure}{0.5\linewidth}
	    		\includegraphics[width=\linewidth]{fig/WP-vs-Rate-RE2.png}\\
	        	\caption{\label{Fig:WP-vs-Rate:RE2}}
	    	\end{subfigure}
	    	\begin{subfigure}{0.5\linewidth}
	    		\includegraphics[width=\linewidth]{fig/WP-vs-Rate-RE4.png}\\
	        	\caption{\label{Fig:WP-vs-Rate:RE4}}
	    	\end{subfigure}
	        \caption{\label{Fig:WP-vs-Rate} Evolution of the working point for RE2 (\ref{Fig:WP-vs-Rate:RE2}) and RE4 (\ref{Fig:WP-vs-Rate:RE4}) with increasing extrapolated $\gamma$ rate per unit area at working point. Both irradiated (blue) and non irradiated (red) chambers are shown.}
	    \end{figure}
	    
	    \begin{figure}[!h]
	    	\begin{subfigure}{0.5\linewidth}
	    		\includegraphics[width=\linewidth]{fig/Eff-vs-Qint-RE2.png}\\
	        	\caption{\label{Fig:Eff-vs-Qint:RE2}}
	    	\end{subfigure}
	    	\begin{subfigure}{0.5\linewidth}
	    		\includegraphics[width=\linewidth]{fig/Eff-vs-Qint-RE4.png}\\
	        	\caption{\label{Fig:Eff-vs-Qint:RE4}}
	    	\end{subfigure}
	        \caption{\label{Fig:Eff-vs-Qint} Evolution of the maximum efficiency at HL-LHC conditions, i.e. a background hit rate per unit area of \SI{300}{Hz/cm^2}, with increasing integrated charge for RE2 (\ref{Fig:Eff-vs-Qint:RE2}) and RE4 (\ref{Fig:Eff-vs-Qint:RE4}) detectors. Both irradiated (blue) and non irradiated (red) chambers are shown. The integrated charge for non irradiated detectors is recorded during test beam periods and stays small with respect to the charge accumulated in irradiated chambers.}
	    \end{figure}
	    
	    \begin{figure}[!h]
	    	\begin{subfigure}{0.5\linewidth}
	    		\includegraphics[width=\linewidth]{fig/CMSlogo.png}\\
	        	\caption{\label{Fig:Eff-Bef-Af:RE2}}
	    	\end{subfigure}
	    	\begin{subfigure}{0.5\linewidth}
	    		\includegraphics[width=\linewidth]{fig/CMSlogo.png}\\
	        	\caption{\label{Fig:Eff-Bef-Af:RE4}}
	    	\end{subfigure}
	        \caption{\label{Fig:Eff-Bef-Aft} Comparison of the efficiency sigmoid before (triangles) and after (circles) irradiation for RE2 (\ref{Fig:Eff-Bef-Af:RE2}) and RE4 (\ref{Fig:Eff-Bef-Af:RE4}) detectors. Both irradiated (blue) and non irradiated (red) chambers are shown.}
	    \end{figure}
	    
	    \begin{figure}[!h]
	    	\begin{subfigure}{0.5\linewidth}
	    		\includegraphics[width=\linewidth]{fig/Res-vs-Qint-RE2.png}\\
	        	\caption{\label{Fig:Res-vs-Qint:RE2}}
	    	\end{subfigure}
	    	\begin{subfigure}{0.5\linewidth}
	    		\includegraphics[width=\linewidth]{fig/Res-vs-Qint-RE4.png}\\
	        	\caption{\label{Fig:Res-vs-Qint:RE4}}
	    	\end{subfigure}
	        \caption{\label{Fig:Res-vs-Qint} Evolution of the Bakelite resistivity for RE2 (\ref{Fig:Res-vs-Qint:RE2}) and RE4 (\ref{Fig:Res-vs-Qint:RE4}) detectors. Both irradiated (blue) and non irradiated (red) chambers are shown.}
	    \end{figure}
	    
	    \begin{figure}[!h]
	    	\begin{center}
	    		\includegraphics[width=0.5\linewidth]{fig/Noise-vs-Qint-RE2.png}\\
	    	\end{center}
	        \caption{\label{Fig:Noise-vs-Qint} Evolution of the noise rate per unit area for the irradiated chamber RE2-2-BARC-9 only.}
	    \end{figure}

		\subsection{Description of the \acl{DAQ}}

		For the longevity studies, four spare chambers of the present system are used. Two spare RPCs of the RE2,3 stations as well as two spare RPCs from the new RE4 stations have been mounted in a Trolley. Six RE4 gaps are also placed in the trolley. The trolley is placed inside the GIF++ in the upstream region of the bunker, taking the cesium source as a reference. The trolley is oriented for the detection surface of the chambers to be orthogonal to the beam line. The system can be moved along the orthogonal plane in order to have the beam in all $\eta$-partitions. For the aging the trolley is moved outside the beam line and is placed in a distance of \SI{5.2}{m} to the source, which irradiates the bunker using an attenuation filter of 2.2 which corresponds to a fluence of $10^7$\si{gamma/cm\squared}.

		During GIF++ operation, the data collected can be divided into different categories as several parameters are monitored in addition to the usual RPC performance data. On one hand, to know the performance of a chamber, it is need to measure its efficiency and to know the background conditions in which it is operated. To do this, the hit signals from the chamber are recorded and stored in a ROOT file via a \acf{DAQ} software. On the other hand, it is also very important to monitor parameters such as environmental pressure and temperature, gas temperature and humidity, RPC HV, LV, and currents, or even source and beam status. This is done through the GIF++ web \acf{DCS} that stores this information in a database.

			\subsubsection{\acs{GIF++} \acs{RPC} \acs{DAQ}}
            
			Two different types of tests are conducted on RPCs via the DAQ. Indeed, the performance of the detectors is measured periodically during dedicated test beam periods using the H4 muon beam. In between these test beam periods, when the beam is not available, the chambers are irradiated by the $^{137}$Cs in order to accumulate deposited charge and the gamma background is measured.

			RPCs under test are connected through LVDS cables to V1190A \acf{TDC} modules manufactured by CAEN. These modules, located in the rack area outside of the bunker, get the logic signals sent by the chambers and save them into their buffers. Due to the limited size of the buffers, the collected data is regularly erased and replaced. A trigger signal is needed for the TDC modules to send the useful data to the DAQ computer via a V1718 CAEN USB communication module.

			In the case of performance test, the trigger signal used for data acquisition is generated by the coincidence of three scintillators. A first one is placed upstream outside of the bunker, a second one is placed downstream outside of the bunker, while a third one is placed in front of the trolley, close by the chambers. Every time a trigger is sent to the TDCs, i.e. every time a muon is detected, knowing the time delay in between the trigger and the RPC signals, signals located in the right time window are extracted from the buffers and saved for later analysis. Signals are taken in a time window of \SI{400}{ns} centered on the muon peak (here we could show a time spectrum). On the other hand, in the case of background rate measurement, the trigger signal needs to be "random" not to measure muons but to look at gamma background. A trigger pulse is continuously generated at a rate of \SI{300}{Hz} using a dual timer. To integrate an as great as possible time, all signals contained within a time window of 10us prior to the random trigger signal are extracted form the buffers and saved for further analysis (here another time spectrum to illustrate could be useful, maybe even place both spectrum together as a single Figure).

			The signals sent to the TDCs correspond to hit collections in the RPCs. When a particle hits a RPC, it induce a signal in the pickup strips of the RPC readout. If this signal is higher than the detection threshold, a LVDS signal is sent to the TDCs. The data is then organised into 4 branches keeping track of the event number, the hit multiplicity for the whole setup, and the time and channel profile of the hits in the TDCs.

			\subsubsection{\acs{RPC} current, environmental and operation parameter monitoring}
            
			In order to take into account the variation of pressure and temperature between different data taking periods the applied voltage is corrected following the relationship :

\begin{equation}
	HVeff = HVapp\times\left(0.2 + 0.8\cdot\frac{P_0}{P}\times\frac{T}{T_0}\right)
\end{equation}

where $T_0$ (=\SI{293}{K}) and $P_0$ (=\SI{990}{mbar}) are the reference values.

	\subsection{Tools \& Measurements}

	Insert a short description of the online tools (DAQ, DCS, DQM).\\
	Insert a short description of the offline tools : tracking and efficiency algorithm.\\
	Identify long term aging effects we are monitoring the rates per strip.

%****************************************** RESULTS AND DISCUSSIONS ************************************************************************
\section{Results and discussions}
\label{sec:results6}

	%****************************************** RESULTS AT GIF **********************************************************************
	\subsection{Preliminary studies results}
	\label{ssec:GIFResults}
	
		\begin{figure}[!h]
			\begin{subfigure}{0.5\linewidth}
				\begin{center}
					\includegraphics[width = 0.6\plotwidth]{fig/Efficiencies.pdf}
					\caption{\label{fig:GIFResults:E}}
				\end{center}
			\end{subfigure}
			\begin{subfigure}{0.5\linewidth}
				\begin{center}
					\includegraphics[width = 0.6\plotwidth]{fig/Evol_HV.pdf}\\
					\caption{\label{fig:GIFResults:WP}}
				\end{center}
			\end{subfigure}
			\begin{subfigure}{0.5\linewidth}
				\begin{center}
					\includegraphics[width = 0.6\plotwidth]{fig/Clustermultiplicities.pdf}
					\caption{\label{fig:GIFResults:CM}}
				\end{center}
			\end{subfigure}
			\begin{subfigure}{0.5\linewidth}
				\begin{center}
					\includegraphics[width = 0.6\plotwidth]{fig/Clustersizes.pdf}\\
					\caption{\label{fig:GIFResults:CS}}
				\end{center}
			\end{subfigure}
			\caption{\label{fig:GIFResults} }
		\end{figure}
	
	%****************************************** RESULTS AT GIF++ ********************************************************************
	\subsection{Longevity studies results}
	\label{ssec:GIF++Results}

%\renewcommand*{\thesection}{\thechapter.\arabic{section}}       % reset again to chaptnum.sectnum

\clearpage{\pagestyle{empty}\cleardoublepage}

% Header
\renewcommand\evenpagerightmark{{\scshape\small Chapter 6}}
\renewcommand\oddpageleftmark{{\scshape\small Improved RPC investigation and preliminary electronics studies}}

\renewcommand{\bibname}{References}

\hyphenation{}

\chapter[Improved RPC investigation and preliminary electronics studies]%
{Improved RPC investigation and preliminary electronics studies}
\label{chapt6}

	The extension in the endcap of the RPC sub-system towards higher pseudo-rapidity will bring the new detectors to be exposed to much more intense background radiations due to the proximity of the detectors with the beam line (Figure~\ref{fig:Dose}). The challenge will be to produce high counting rate detectors with limited ageing rate to ensure a stable operation of the detector over a period longer than ten years. In Chapter~\ref{chapt4} was discussed the influence of the detector design (number and thickness of gas volumes, OR system, etc...) on the charge deposition and rate capability. Nevertheless, this question can also be addressed from the electronics point of view as a better signal-to-noise ratio would also mean the possibility to greatly lower the charge threshold on the signals to be detected, allowing to use the detector at lower gain, hence lowering the charge deposition per avalanche in the gas volume. Cardarelli showed that the production of low-noise fast FEEs could help decreasing the charge deposition per avalanche at working voltage by an order of magnitude, virtually increasing the life expectancy of such a detector in the same way~\cite{CARDARELLI2012}.

\section{FEE candidates for the production of iRPCs}
\label{chapt6:sec:candidates}

	The extension of the third of fourth endcap disks with \acl{iRPC}s has been presentated in Chapter~\ref{chapt:3} together with the expected background levels (Figure~\ref{fig:iRPC-Rate}). An important piece of these iRPCs will be the \acl{FEE} that will equip the chambers. A fast, low-jitter and low-charge sensitive electronics will help reducing further the charge deposition in the detector by making it possible to operate at lower gain.
	
	In the context of the CMS Muon Upgrade, two FEE solutions have been considered to equip the iRPCs. The baseline for the RPC upgrade is based on the PETIROC ASIC, initially for \acf{ToF} applications. A back-up solution is also under study and focusses on a new low-noise preamplifier designed in INFN laboratories in Rome to replace the preamplification stage of the already existing CMS RPC \acl{FEB}.

	The FEEs that are foreseen to equip the new RPCs need to be able to detect charges as small as \SI{10}{fC}. Not only the new electronics need to be fast and reliable, they also should be able to sustain the high radiation the detectors will be subjected to in the region closest to the beam.
	
	\subsection{CMS RPCROC: the RPC upgrade baseline}
	\label{chapt6:ssec:RPCROC}
	
	Designed by Weeroc, a spin-off company from the french OMEGA laboratory, the PETIROC 2A consits in a fast and low jitter 32-channel ASIC originally developed to read-out \acf{SiPM} in ToF applications and that allows for precise time measurements~\cite{PETIROCIEEE,PETIROCTWEPP}. The ASIC uses an AMS \SI{350}{ns} SiGe technology. The block diagram of the ASIC is showed on Figure~\ref{fig:PETIROCASIC}. A 10-bit DAC allows to adjust the trigger level in a dynamical range spanning from 0.5 to a few tens of photoelectrons and a 6-bit DAC to adjust the response of each individual channel to similar a level.
	
	\begin{figure}[H]
		\centering
		\includegraphics[width = \linewidth]{fig/chapt6/petiroc2.png}\\
		\caption{\label{fig:PETIROCASIC} PETIROC 2A block diagram.}
	\end{figure}
	
	Nevertheless, in order to adapt this ASIC to CMS, modifications were brought to the PETIROC~\cite{PHASEIITP}. In the new CMS RPCROC, the measurement of the charge will be performed by a TimeOverTechnic, taking profit of the capacity the ASIC has in measuring both the leading and trailing edges of the input signals. The dynamic range will be expanded towards lower values to allow for the detection of charges as low as \SI{10}{fC}. Due to the radiation levels that are foreseen at the level of the iRPCs, the SiGe technology will be replaced by the Taiwan Semiconductor Manufacturing Company (TSMC) \SI{130}{nm} CMOS, to increase its radiation hardness while keeping fast pre-amplification and discrimination with a very low jitter that can reach less than \SI{20}{ps} if no internal clock is used, as can be seen from Figure~\ref{fig:jitter}. The ASIC is associated with an FPGA which purpose is to measure time thanks to a TDC with a time resolution of 50-100 \si{ps} developed by Tsinghua University and that will provide a measurement of the signal position along the strip with a precision of a few \si{cm} by measuring the signal timing on both ends of the strips. In order to read-out all 96 strips, 3 ASICs and 3 TDCs, each having an increased number of 64-channels, are hosted on a FEB attached to the chamber.

	\begin{figure}[H]
		\centering
		\includegraphics[width=0.8\plotwidth]{fig/chapt6/jitter-PETIROC.png}
		\caption{\label{fig:jitter} The PETIROC time jitter as a function of the input signal amplitude, measured with and without internal clocks.}
	\end{figure}

	\subsection{INFN \acl{FEE}: a robust back-up solution}
	\label{chapt6:ssec:INFN}

\section{Preliminary tests at CERN}
\label{chapt6:sec:Preliminary}

	\subsection{INFN preamplifiers}
	\label{chapt6:ssec:INFN-Prelim}
	
	INFN electronics were the first ones to be tested by CMS RPC group in collaboration with colleagues from INFN Roma working in the ATLAS RPC group. Indeed, at first the electronics only consisted in a new low-noise preamplifier produced by the team of Cardarelli with the purpose of equipping the new generation of ATLAS RPCs~\cite{CARDARELLI2013}. The tests with CMS RPCs were performed in February 2013 outside of the old GIF facility presented in Chapter~\ref{chapt5:ssec:GIF}. Four preamplifier channels were lended by Cardarelli to equip four CMS RPC channels as presented in Figure~\ref{fig:INFN-preamp}. They were directly connected to the strips for the signals induced by muons passing through the gas volume of the chamber to be amplified. The output was then sent to a discriminator to digitize the signals and filter out the noise by tuning the threshold level.
	
	\begin{figure}[H]
		\centering
		\includegraphics[width=0.8\plotwidth]{fig/chapt6/INFN-Preamp-2013.JPG}
		\caption{\label{fig:INFN-preamp} The four channels of INFN preamplifiers are mounted directly on a CMS RPC and connected to the four outermost read-out strips of the detector.}
	\end{figure}
	
	The performance of the chamber equipped with these new preamplifiers was compared to the performance of CMS FEEs. The experimental setup used is described in Figure~\ref{Setup-GIF}. PMTs a little less wide than four strips were used to trigger the data tacking. Two pairs were used in coïncidence on both the strips connected to the INFN preamplifiers and to the ones connected to the CMS FEEs. An extra PMT, placed perpendicularly to the rest of the setup at the bottom of the setup was used to detect potential showers and send VETO signals if necessary. A last PMT was used close to the power supplies to measure and discard signals due to electromagnetic noise and is not visible on the pictures. Finally, after discrimination, the output of the INFN preamplifiers together with the signals from the CMS FEEs were sent to scalers to count the detected signals versus the number of trigger coïncidences as no DAQ software was available at the time. The full pulse processing for this experiment is shown in Figure~\ref{fig:Pulse-Processing}.
	 
	\begin{figure}[H]
		\begin{subfigure}{\linewidth}
		    \centering
			\includegraphics[width = 0.7\linewidth]{fig/chapt6/Setup-GIF-side.JPG}
			\caption{\label{fig:Setup-GIF:A}}
		\end{subfigure}
		\begin{subfigure}{0.5\linewidth}
		    \centering
			\includegraphics[width = 0.8\linewidth]{fig/chapt6/Setup-GIF-front.JPG}
			\caption{\label{fig:Setup-GIF:B}}
		\end{subfigure}
		\begin{subfigure}{0.5\linewidth}
		    \centering
			\includegraphics[width = 0.8\linewidth]{fig/chapt6/Pulse-processing-GIF.JPG}
			\caption{\label{fig:Setup-GIF:C}}
		\end{subfigure}
		\caption{\label{fig:Setup-GIF} Experimental setup used to test the INFN preamplifier with respect to the CMS FEEs.}
    \end{figure}
	 
	\begin{figure}[H]
		\begin{subfigure}{.5\linewidth}
		    \centering
			\includegraphics[width = 0.9\linewidth]{fig/chapt6/atlas-block-diagram-2013.pdf}
			\caption{\label{fig:Pulse-Processing:A}}
		\end{subfigure}
		\begin{subfigure}{.5\linewidth}
		    \centering
			\includegraphics[width = 0.9\linewidth]{fig/chapt6/cms-block-diagram-2013.pdf}
			\caption{\label{fig:Pulse-Processing:B}}
		\end{subfigure}
		\begin{subfigure}{\linewidth}
		    \centering
			\includegraphics[width = 0.8\linewidth]{fig/chapt6/pulse-processing-2013.pdf}
			\caption{\label{fig:Pulse-Processing:C}}
		\end{subfigure}
		\caption{\label{fig:Pulse-Processing} The block diagrams corresponding to the signal treatment for both INFN preamplifier (Figure~\ref{fig:Pulse-Processing:A}) and CMS FEEs (Figure~\ref{fig:Pulse-Processing:B}) are shown. The digitized signals are then counted in coïncidence with the trigger signals provided by PMTs (Figure~\ref{fig:Pulse-Processing:C}).}
    \end{figure}
    
    The data taking program consisted in \acl{HV} scans. A first point was taken at \SI{0}{V} to only measure noise. Then the HV was increased to an applied value of \SI{7}{kV}. The voltage was increased in steps of \SI{500}{V} until \SI{8}{kV} from where it was increased in steps of \SI{100}{V} until an upper limit of \SI{10}{kV}. After rising the voltage over the electrodes of the RPC, a waiting period of 15 minutes was observed to leave time to the electrodes to charge and to the currents to stabilize. The currents were reported at the moment the data taking was started. At each HV step, except at \SI{0}{V}, approximatively 300 triggers were taken to estimate the efficiency of the detector by counting the number of hits in the system (A or B or C or D), referring to the strips. The noise rate per unit area was measured during the first \SI{100}{s} of data taking by counting the number of hits recieved in each read-out strip. The cluster size, the average number of adjacent strips fired during a muon event, could not be measured due to the lack of available scalers.
    
    During the data acquisition, in addition to counting the number of signals with respect to the number of triggers, the current or the noise rate per unit area as a function of the increasing voltage, the environmental parameters were monitored. Using the information provided by a humidity and temperature sensor on the gas input line together with the environmental pressure given by a weather station, the applied voltage could be corrected following Formula~\ref{eq:PTCMS}. Moreover, the voltage line was filtered to prevent noise and higher currents in the RPC under test.
    
    The results of the preliminary tests are presented in Figure~\ref{fig:INFN-preamp}. More details on the performance shift are provided in Table~\ref{tab:INFN-preamp}. Being able to use electronics with a much higher sensitivity allows for a HV shift of up to \SI{475}{V} with a threshold as low as \SI{3}{fC} corresponding to the lowest threshold available on the discriminator modules used in the experiment. Indeed, the NIM quad discriminator 821 manufactured by LECROY only allows at minimum to set the threshold at a voltage of approximately \SI{30}{mV} on the input signals. Thus, two values of discrimination were used ($\sim$\SI{75}{mV} and $\sim$\SI{30}{mV}).
	
	\begin{figure}[H]
		\begin{subfigure}{.5\linewidth}
		    \centering
			\includegraphics[width=\linewidth]{fig/chapt6/INFN-Preamplifier-Shift.pdf}
			\caption{\label{fig:INFN-preamp:A}}
		\end{subfigure}
		\begin{subfigure}{.5\linewidth}
		    \centering
			\includegraphics[width = \linewidth]{fig/chapt6/INFN-Preamplifier-Rate-Shift.pdf}
			\caption{\label{fig:INFN-preamp:B}}
		\end{subfigure}
		\caption{\label{fig:INFN-preamp} Efficiency (Figure~\ref{fig:INFN-preamp:A}) and noise rate per unit area (Figure~\ref{fig:INFN-preamp:B}) of the CMS RE2-2 detector tested with the standard CMS FEBs (black) and with the INFN preamplifier at different thresholds (red and blue).}
	\end{figure}
	
	\begin{table}[H]
		\caption{\label{tab:INFN-preamp} Results of the sigmoid fit (Formula~\ref{eq:Sigmoid}) performed on the data presented in Figure~\ref{fig:INFN-preamp:A}. The working point and its corresponding efficiency are computed using Formula~\ref{eq:KneeWP}.}
		\footnotesize
		\begin{tabular}{|c|c|c|c|c|c|}
			\hline
			Data & $\epsilon_{max}$ & $\lambda$ ($\times$\Ord{-2} \si{V^{-1}}) & $HV_{50}$ (\si{V}) & $\epsilon_{WP}$ & $HV_{WP}$ (\si{V}) \\ 
			\hline
			CMS FEB, 156fC (2013) & \numerror{0.978}{0.004} & \numerror{1.12}{0.07} & \numerror{9339}{11} & \numerror{0.97}{0.01} & \numerror{9752}{27}\\ 
			\hline
			INFN preamp., 7fC (2013) & \numerror{0.987}{0.003} & \numerror{0.93}{0.05} & \numerror{8907}{11} & \numerror{0.97}{0.01} & \numerror{9374}{27}\\ 
			\hline
			INFN preamp., 3fC (2013) & \numerror{0.991}{0.003} & \numerror{0.86}{0.04} & \numerror{8783}{11} & \numerror{0.98}{0.01} & \numerror{9276}{27}\\ 
			\hline
		\end{tabular}
	\end{table}

	\subsection{INFN preamplifiers mounted onto CMS \acl{FEB}}
	\label{chapt6:ssec:INFN-FEB}
	
	Following the first experiment performed in the experimental hall aside of the old GIF, a new series of tests has been done in the CMS RPC assembly laboratory at CERN. For this purpose, the preamplifiers have been designed to be standalone single channels. To have a consistent comparison with the CMS FEB, a FEB prototype has been built based on the current CMS design. As shown in Figure~\ref{fig:Setup-INFN-904}, the preamplifiers are meant to be plugged in one of the available 16 channels of the board that produces an LVDS output with similar characteristics than the CMS FEB.
	 
	\begin{figure}[H]
		\begin{subfigure}{.5\linewidth}
		    \centering
			\includegraphics[width = \linewidth]{fig/chapt6/ATLAS_FEB.png}
			\caption{\label{fig:Setup-INFN-904:A}}
		\end{subfigure}
		\begin{subfigure}{.5\linewidth}
		    \centering
			\includegraphics[width = 0.56\linewidth]{fig/chapt6/ATLAS_preamp.JPG}
			\caption{\label{fig:Setup-INFN-904:B}}
		\end{subfigure}
		\begin{subfigure}{\linewidth}
		    \centering
			\includegraphics[width = 0.5\linewidth]{fig/chapt6/Setup_ATLAS_PAK.JPG}
			\caption{\label{fig:Setup-INFN-904:C}}
		\end{subfigure}
		\caption{\label{fig:Setup-INFN-904} Figure~ref{fig:Setup-INFN-904:A}: Shielded \acl{FEB} on which the INFN preamplifiers are to be mounted. Figure~ref{fig:Setup-INFN-904:B}: Three INFN preamplifiers connected onto the test FEB. Figure~ref{fig:Setup-INFN-904:C}: Experimental setup used to test the INFN preamplifier single mounted on a FEB similar to the CMS FEB.}
    \end{figure}
	
	At the time of the second experiment, only three channels could be lended by the team of INFN Roma. It was then decided to use the same PMTs than in the first measurements as trigger. This time, they were placed on their side to only cover an area on the detector smaller than three strips. On the data acquisition side, no DAQ software was available yet at the time of experimentation and scalers were once again used. As can be seen from Figure~\ref{Pulse-Processing-904}, the pulse processing has been inspired by the one used in the first experiment. Thanks to the lower number of channels to monitor, the cluster size could be estimated by counting the signals on single channels but also on groups of two and three channels in coincidence with the trigger.
	
	\begin{figure}[H]
		\centering
		\includegraphics[width=.8\linewidth]{fig/chapt6/pulse-processing-2014.pdf}
		\caption{\label{fig:fig:Pulse-Processing-904} Similarly to Figure~\ref{fig:Pulse-Processing:C}, the signals are then counted in coïncidence with the trigger signals provided by PMTs. To estimate the cluster size, the channels are counted by groups of three, two and alone.}
	\end{figure}
	
	\begin{figure}[H]
		\begin{subfigure}{.5\linewidth}
		    \centering
			\includegraphics[width=\linewidth]{fig/chapt6/INFN-LVDS-Eff-Shift.pdf}
			\caption{\label{fig:INFN-FEB:A}}
		\end{subfigure}
		\begin{subfigure}{.5\linewidth}
		    \centering
			\includegraphics[width = \linewidth]{fig/chapt6/INFN-LVDS-ClS-Shift.pdf}
			\caption{\label{fig:INFN-FEB:B}}
		\end{subfigure}
		\begin{subfigure}{\linewidth}
		    \centering
			\includegraphics[width = .5\linewidth]{fig/chapt6/INFN-LVDS-Rate-Shift.pdf}
			\caption{\label{fig:INFN-FEB:C}}
		\end{subfigure}
		\caption{\label{fig:INFN-FEB} Efficiency (Figure~\ref{fig:INFN-FEB:A}), cluster size (Figure~\ref{fig:INFN-FEB:B}) and noise rate per unit area (Figure~\ref{fig:INFN-FEB:C}) of the CMS RE2-2 detector tested with the standard CMS FEBs (black) and with the INFN preamplifier mounted onto the CMS FEB at different thresholds (red, blue, pink, green and purple).}
	\end{figure}
	
	\begin{table}[H]
		\caption{\label{tab:INFN-FEB} Results of the sigmoid fit (Formula~\ref{eq:Sigmoid}) performed on the data presented in Figure~\ref{fig:INFN-FEB:A}. The working point and its corresponding efficiency are computed using Formula~\ref{eq:KneeWP}.}
		\footnotesize
		\begin{tabular}{|c|c|c|c|c|c|}
			\hline
			Data & $\epsilon_{max}$ & $\lambda$ ($\cdot$\Ord{-2} \si{V^{-1}}) & $HV_{50}$ (\si{V}) & $\epsilon_{WP}$ & $HV_{WP}$ (\si{V}) \\ 
			\hline
			CMS FEB, 156fC (2013) & \numerror{0.978}{0.004} & \numerror{1.12}{0.07} & \numerror{9339}{11} & \numerror{0.97}{0.01} & \numerror{9752}{27}\\ 
			\hline
			CMS FEB, 170fC (2014) & \numerror{0.978}{0.003} & \numerror{1.30}{0.06} & \numerror{9364}{9} & \numerror{0.97}{0.01} & \numerror{9740}{19}\\ 
			\hline
			INFN/CMS FEB, 7.2fC (2014) & \numerror{0.973}{0.006} & \numerror{1.26}{0.09} & \numerror{8985}{10} & \numerror{0.97}{0.01} & \numerror{9368}{26}\\ 
			\hline
			INFN/CMS FEB, 6.4fC (2014) & \numerror{0.978}{0.007} & \numerror{1.16}{0.08} & \numerror{8969}{11} & \numerror{0.97}{0.01} & \numerror{9372}{28}\\ 
			\hline
			INFN/CMS FEB, 5.5fC (2014) & \numerror{0.981}{0.005} & \numerror{1.26}{0.09} & \numerror{8973}{12} & \numerror{0.97}{0.01} & \numerror{9357}{28}\\ 
			\hline
			INFN/CMS FEB, 5fC (2014) & \numerror{0.987}{0.004} & \numerror{1.37}{0.10} & \numerror{8976}{12} & \numerror{0.98}{0.01} & \numerror{9342}{28}\\ 
			\hline
		\end{tabular}
	\end{table}

	\subsection{HARDROC 2 readout panel}
	\label{chapt6:ssec:HARDROC2}
	 
	\begin{figure}[H]
		\begin{subfigure}{.5\linewidth}
		    \centering
			\includegraphics[width = \linewidth]{fig/chapt6/HARDROC_PCB.jpg}
			\caption{\label{fig:Setup-HARDROC2-904:A}}
		\end{subfigure}
		\begin{subfigure}{.5\linewidth}
		    \centering
			\includegraphics[width = \linewidth]{fig/chapt6/HARDROC_chip.JPG}
			\caption{\label{fig:Setup-HARDROC2-904:B}}
		\end{subfigure}
		\begin{subfigure}{\linewidth}
		    \centering
			\includegraphics[width = 0.5\linewidth]{fig/chapt6/Setup_HARDROC_PAK.jpg}
			\caption{\label{fig:Setup-HARDROC2-904:C}}
		\end{subfigure}
		\caption{\label{fig:Setup-HARDROC2-904} Figure~ref{fig:Setup-HARDROC2-904:A}: Readout panel with the HARDROC2 ASIC placed on a CMS RPC gap. Figure~ref{fig:Setup-HARDROC2-904:B}: HARDROC2 control chip with its "Mezzanine" used to collect the data from the different HARDROC ASICs and communicate with the computer. Figure~ref{fig:Setup-HARDROC2-904:C}: Experimental setup used to test the HARDROC2 electronics.}
    \end{figure}
	
	\begin{figure}[H]
		\begin{subfigure}{.5\linewidth}
		    \centering
			\includegraphics[width=\linewidth]{fig/chapt6/HARDROC2-Eff-Shift.pdf}
			\caption{\label{fig:HARDROC2:A}}
		\end{subfigure}
		\begin{subfigure}{.5\linewidth}
		    \centering
			\includegraphics[width = \linewidth]{fig/chapt6/HARDROC2-ClS-Shift.pdf}
			\caption{\label{fig:HARDROC2:B}}
		\end{subfigure}
		\begin{subfigure}{\linewidth}
		    \centering
			\includegraphics[width = .5\linewidth]{fig/chapt6/HARDROC2-Rate-Shift.pdf}
			\caption{\label{fig:HARDROC2:C}}
		\end{subfigure}
		\caption{\label{fig:HARDROC2} Efficiency (Figure~\ref{fig:HARDROC2:A}), cluster size (Figure~\ref{fig:HARDROC2:B}) and noise rate per unit area (Figure~\ref{fig:HARDROC2:C}) of the CMS RE4-3 detector tested with the standard CMS FEBs (black) and with the HARDROC 2 readout panel at different thresholds (red, blue and pink).}
	\end{figure}
	
	\begin{table}[H]
		\caption{\label{tab:HARDROC2} Results of the sigmoid fit (Formula~\ref{eq:Sigmoid}) performed on the data presented in Figure~\ref{fig:HARDROC2:A}. The working point and its corresponding efficiency are computed using Formula~\ref{eq:KneeWP}.}
		\footnotesize
		\begin{tabular}{|c|c|c|c|c|c|}
			\hline
			Data & $\epsilon_{max}$ & $\lambda$ ($\cdot$\Ord{-2} \si{V^{-1}}) & $HV_{50}$ (\si{V}) & $\epsilon_{WP}$ & $HV_{WP}$ (\si{V}) \\ 
			\hline
			CMS FEB, 156fC (2014) & \numerror{0.958}{0.000} & \numerror{0.75}{0.00} & \numerror{9174}{1} & \numerror{0.94}{0.00} & \numerror{9716}{2}\\ 
			\hline
			HARDROC 2, 230fC (2015) & \numerror{0.987}{0.002} & \numerror{1.06}{0.04} & \numerror{8905}{8} & \numerror{0.98}{0.01} & \numerror{9333}{17}\\ 
			\hline
			HARDROC 2, 143fC (2015) & \numerror{0.988}{0.001} & \numerror{1.10}{0.04} & \numerror{8826}{8} & \numerror{0.98}{0.01} & \numerror{9243}{17}\\ 
			\hline
			HARDROC 2, 121.4fC (2015) & \numerror{0.987}{0.001} & \numerror{1.07}{0.04} & \numerror{8795}{8} & \numerror{0.98}{0.01} & \numerror{9220}{17}\\ 
			\hline
		\end{tabular}
	\end{table}

\clearpage{\pagestyle{empty}\cleardoublepage}
% Header
\renewcommand\evenpagerightmark{{\scshape\small Chapter 7}}
\renewcommand\oddpageleftmark{{\scshape\small Conclusions and outlooks}}

\hyphenation{}

\chapter[Conclusions and outlooks]{Conclusions and outlooks}
\label{chapt7}
	
	The CMS RPC upgrade has been and will keep on being an exciting scientific research. The collaboration converged towards the solutions that will be adopted in the perspective of HL-LHC. Eventhough the consolidation of the present CMS RPC infrastructure and the certification of the new technologies that will complete the redundancy of the muon system are still ongoing, the future of the experiment from the RPC point of view is now clear.
	
	To reach this point, the contribution of Ghent during the preliminary phase of tests between 2012 and 2015 has been decisive in both the consolidation of the present detectors and in the selection of the \acl{FEE} that will equip the iRPCs. At every step, Ghent University played a leading role in setting up the experiments but also in gathering and analysing the data. First of all, two potential FEE technologies were selected and it was showed that both of them could be used for new CMS detectors. On one side, the INFN amplifier provided a very interesting sensitivity to low charge depositions. On the other hand, the FEEs developed by OMEGA (HARDROC, PETIROC) showed that they could provide a reduction of the working voltage at similar charge deposition ranges. Moreover, this technology had already been certified through multiple experiments using detectors such as scintillators and RPCs. Finally, it had the advantage of proposing a 2D read-out that would greatly improve the spatial resolution of the detectors in the radial direction. The expertise of the Intrumentation group was demonstrated in this campaign.
	
	As a natural continuation, a door was opened to join the GIF++ effort at key positions. A major contribution to the development of the \acl{DAQ}, \acl{DQM} and data analysis tools was provided and will keep helping the collaboration in conducting robust R\&D research in the future. Indeed, new young experts are emerging and taking over the tools to improve them with fresh ideas. So far, the CMS RPC group is on the way of certifying the current RPC system for the HL-LHC period. The Link-system will be upgraded and the present detectors should live through the high-luminosity phase of the LHC without important change in their performance. The RPC that will complete the redundancy of the muon system are being certified as well and show very good performance under high-irradiation that is so far foreseen to stay stable throughout the whole Phase-2.
	
	Nevertheless, the present thesis document only focusses on the R\&D produced by the CMS RPC on the present and new detection technologies that are and will be used at CMS. Few information about the very important research being conducted in order to find a replacement to the standard RPC gas mixture is provided. The outcome of this search for new gases will be of major interest as the restriction for the standard mixture will get harder.
	
	Once the R\&D will be complete, the next phase will consist in the upgrade of the RPC sub-system. Ghent will mainly take part in manufacturing the detectors for the expension of the endcaps as was already the case for the production of the RE4 detectors for the fourth endcap disk of CMS between 2012 and 2013. After LS3, the LHC will finally enter its high-luminosity phase and new breakthrough will be foreseen. The good performance of the RPCs and of all of CMS sub-systems will be important in this reguard and the skills developed during the present R\&D will become an important asset in maintaining the performance of the detectors at their best level.

%\renewcommand*{\thesection}{\thechapter.\arabic{section}}       % reset again to chaptnum.sectnum

\clearpage{\pagestyle{empty}\cleardoublepage}

\newpage

%%%%%%%%%%%%%%%%%%%%%%%%%%%%%%%%%%%%%%%%%%%%%%%%%%%%%%%%%%%%%%%%%%%%%%%%%%%%%%%%%%%%%%%%%
%% SUMMARY IN DUTCH     %%
%%%%%%%%%%%%%%%%%%%%%%%%%%
\selectlanguage{dutch}
\renewcommand{\thesection}{\arabic{section}}    % chapter without number, so don't use chapterno.sectionno

\renewcommand{\bibname}{Referenties}
% Header
\renewcommand\evenpagerightmark{{\scshape\small Nederlandse Samenvatting}}
\renewcommand\oddpageleftmark{{\scshape\small Dutch Summary}}

\chapter[Nederlandse samenvatting]%
{Nederlandse samenvatting \\\makebox[2.82in]{--Dutch Summary--}}

\hyphenation{}
\def\hyph{-\penalty0\hskip0pt\relax}



\clearpage{\pagestyle{empty}\cleardoublepage}




%%%%%%%%%%%%%%%%%%%%%%%%%%%%%%%%%%%%%%%%%%%%%%%%%%%%%%%%%%%%%%%%%%%%%%%%%%%%%%%%%%%%%%%%%
%% ACKNOWLEDGMENT    %%
%%%%%%%%%%%%%%%%%%%%%%%

\selectlanguage{english}
\renewcommand{\thesection}{\arabic{section}}
\renewcommand\evenpagerightmark{{\scshape\small Acknowledgements}}
\renewcommand\oddpageleftmark{{\scshape\small Acknowledgements}}
\chapter{Acknowledgements}
\vspace{0.35in}

\begin{flushright}{\emph{28/11/2019\\
Alexis Fagot}}
\end{flushright}

	The PhD project I worked on for the past seven years is finally reaching an end. It's time to look back and thank all the people without whom it wouldn't have been possible. I thank Gabriella and Nicolas for giving me the opportunity to work on such an interesting project. It has been motivating working in a team with a lot of young researchers in the same situation as me. I mainly think of Andrea, François, Jan, Severiano and Shereen with whom I interacted the most when I was at CERN. Of course, the team also was mentored by great people with a lot of experience like Ian and Salvador with whom I got a lot of pleasure working.\\
	My interest in the R\&D project and my knowledge of detectors opened a few other doors. I got the trust from Anna and Gabriella to become expert-on-call for the RPC sub-system at CMS which was not an easy task but that brought me a lot of experience and confidence. I feel lucky that people such as Salvador and Isabel invited me at more than one occasion to go to Mexico to work with students there. This was very rewarding. Ultimately, I would like to thank my promotor Michael for giving me this chance and Imad for putting us in contact. Michael let me a lot of autonomy to take the PhD project in the direction I wanted and Imad has been a great Master Responsible back in Lyon. He knew his students well and knew where they could flourish.\\
	
	But of course, the completion of this PhD project is also greatly subjected to the social life I developed and the support from my family and friends. I could count on my parents, brother and cousins, Yoan and Laetitia, at any moment throughout these years. Special thanks to Laetitia and Thierry, but also to the kids, for hosting me each time I was at CERN. This was a huge moral help. CERN is a wonderful place to work, but the work load can be overwhelming. Childhood friends also did a lot. Clément and Lolita brought me some light by regularly visiting me. The bond with Mikael and Xavier never weakened even though we could not meet very often anymore.\\
	I wish to thank Nick. We managed to support each other and it turned out quite well in the end. Of course, I can't thank Nick without including Ibby. I wish she didn't leave Ghent so soon, we met a little too late. Our trio was quite fun. I was lucky to have great flatmates. I had first met Bryan in the kot we were sharing during our first year in Ghent. Very friendly and open minded. Paul, I knew already from Lyon but it turned out that he also applied for a PhD in Ghent and got it. It has been nice to share a place with you two. Of course, then came Nina that has been also very nice to live with. I remember nice parties at home and a lot of shared food.\\
	A big thanks to Mostafa and Abbas for the ones that have been here since the very beginning and to Dinka, Amélie and Joe to include me in there groups when I needed it. Jean-François, with his great mind, holds a very special place as he never stopped challenging me with sciences and other topics. It was always a great time meeting with him. Thanks to the ones I met without really expecting it but that naturally integrated me in their circle. Anthony, Charline, Dylan, Florian and Maurane, Guillaume, Roxane, Thomas, both of the Valentins, and Vincent. I probably got some of the most life changing conversations with them. My vision of the world really evolved at their contact. Finally, I would like to thank the people that made me want to build my life here. Alba and Andres, Carmen and Raul, Kajetan, Maren and Prem, Martina, Meta and Vishvas, Nils and Liesl, Ola, and of course Paula, who is sharing my life for a while now.\\
	
	To some of you, even if our paths are now separate, I don't forget what you brought me. I am thankful for that. To the rest of you that I have the pleasure on meeting often, you make of every day a good moment. I hope it goes on for as long as possible. Take care.

\clearpage{\pagestyle{empty}\cleardoublepage}

%%%%%%%%%%%%%%%%%%%%%%%%%%%%%%%%%%%%%%%%%%%%%%%%%%%%%%%%%%%%%%%%%%%%%%%%%%%%%%%%%%%%%%%%%
%% APPENDICES        %%
%%%%%%%%%%%%%%%%%%%%%%%

\appendix
\graphicspath{{chapt_dutch/}{intro/}{chapt2/}{chapt3/}{chapt4/}{chapt5/}{chapt6/}{chapt7/}}

% Header
\renewcommand\evenpagerightmark{{\scshape\small Appendix A}}
\renewcommand\oddpageleftmark{{\scshape\small A data acquisition software for VME CAEN TDCs}}

\renewcommand{\bibname}{References}

\hyphenation{}

\chapter[A data acquisition software for VME CAEN TDCs]%
{A data acquisition software for VME CAEN TDCs}
\label{app1}

Certifying detectors in the perspective of HL-LHC required to develop tools for the GIF++ experiment. One of them was the \acf{DAQ} software that allows to make the communications in between the computer and the TDC modules in order to retrieve the RPC data. In this appendix, details about the software, as of how the software was written, how it functions and how it can be exported to another similar setup.

\section{Introduction}


\clearpage{\pagestyle{empty}\cleardoublepage}

\newfloat{algo}{h}{alg}

\graphicspath{{chapt_dutch/}{intro/}{chapt2/}{chapt3/}{chapt4/}{chapt5/}{chapt6/}{chapt7/}}

% Header
\renewcommand\evenpagerightmark{{\scshape\small Appendix B}}
\renewcommand\oddpageleftmark{{\scshape\small Details on the online analysis package}}

\renewcommand{\bibname}{References}

\hyphenation{}

\chapter[Details on the offline analysis package]%
{Details on the offline analysis package}
\label{app2}

The data collected in GIF++ thanks to the DAQ described in Appendix~\ref{app1} is difficult to interpret by a human user that doesn't have a clear idea of the raw data architecture of the CMS RPC ROOT files. In order to render the data human readable, a C++ offline analysis tool was designed to provide users with detector by detector histograms that give a clear overview of the parameters monitored during the data acquisition~\cite{GIFOffline}. In this appendix, details about this software, as of how the software was written and how it functions will be given.

\section{GIF++ Offline Analysis file tree}
\label{app2:sec:code}

	GIF++ Offline Analysis source code is fully available on github at \url{https://github.com/afagot/GIF_OfflineAnalysis}. The software requires \href{https://root.cern.ch/downloading-root}{ROOT} as non-optionnal dependency as it takes ROOT files in input and write an output ROOT file containing histograms. To compile the GIF++ Offline Analysis project is compiled with cmake. To compile, first a \verb+build/+ directory must be created to compile from there:

	\begin{minted}[bgcolor=bg]{bash}
mkdir build
cd build
cmake ..
make
make install
	\end{minted}
	
	To clean the directory and create a new build directory, the bash script \verb+cleandir.sh+ can be used:
	
	\begin{minted}[bgcolor=bg]{bash}
./cleandir.sh
	\end{minted}

	The source code tree is provided below along with comments to give an overview of the files' content. The different objects created for this project (\cppinline{Infrastructure}, \cppinline{Trolley}, \cppinline{RPC}, \cppinline{Mapping}, \cppinline{RPCHit}, \cppinline{RPCCluster} and \cppinline{Inifile}) will be described in details in the following sections.\\
	
	\newpage

	\dirtree{%
	 .1 GIF\_OfflineAnalysis.
	 .2 bin.
	 .3 offlineanalysis\DTcomment{executable}.
	 .2 build\DTcomment{cmake compilation directory}.
	 .3 ....
	 .2 include\DTcomment{list of C++ header files}.
	 .3 Cluster.h\DTcomment{declaration of object RPCCluster}.
	 .3 Current.h\DTcomment{declaration of GetCurrent analysis macro}.
	 .3 GIFTrolley.h\DTcomment{declaration of object Trolley}.
	 .3 Infrastructure.h\DTcomment{declaration of object Infrastructure}.
	 .3 IniFile.h\DTcomment{declaration of object IniFile for ini parser}.
	 .3 Mapping.h\DTcomment{declaration of object Mapping}.
	 .3 MsgSvc.h\DTcomment{declaration of offline log messages}.
	 .3 OfflineAnalysis.h\DTcomment{declaration of data analysis macro}.
	 .3 RPCDetector.h\DTcomment{declaration of object RPC}.
	 .3 RPCHit.h\DTcomment{declararion of object RPCHit}.
	 .3 types.h\DTcomment{definition of useful variable types}.
	 .3 utils.h\DTcomment{declaration of useful functions}.
	 .2 obj\DTcomment{binary files created by compiler}.
	 .3 ....
	 .2 src\DTcomment{list of C++ source files}.
	 .3 Cluster.cc\DTcomment{definition of object RPCCluster}.
	 .3 Current.cc\DTcomment{definition of GetCurrent analysis macro}.
	 .3 GIFTrolley.cc\DTcomment{definition of object Trolley}.
	 .3 Infrastructure.cc\DTcomment{definition of object Infrastructure}.
	 .3 IniFile.cc\DTcomment{definition of object IniFile for ini parser}.
	 .3 main.cc\DTcomment{main file}.
	 .3 Mapping.cc\DTcomment{definition of object Mapping}.
	 .3 MsgSvc.cc\DTcomment{definition of offline log messages}.
	 .3 OfflineAnalysis.cc\DTcomment{definition of data analysis macro}.
	 .3 RPCDetector.cc\DTcomment{definition of object RPC}.
	 .3 RPCHit.cc\DTcomment{definition of object RPCHit}.
	 .3 utils.cc\DTcomment{definition of useful functions}.
	 .2 cleandir.sh\DTcomment{bash script to clean build directory}.
	 .2 CMakeLists.txt\DTcomment{set of instructions for cmake}.
	 .2 config.h.in\DTcomment{definition of version number}.
	 .2 README.md\DTcomment{REAMDE file for github}.
	}
	
\section{Usage of the Offline Analysis}
\label{app2:sec:usage}

	In order to use the Offline Analysis tool, it is mandatory to know the Scan number and the HV Step of the run that needs to be analysed. This information needs to be written in the following format:
	
	\begin{minted}[bgcolor=bg]{bash}
Scan00XXXX_HVY
	\end{minted}

	where \verb+XXXX+ is the scan ID and \verb+Y+ is the high voltage step (in case of a high voltage scan, data will be taken for several HV steps). This format corresponds to the data file base name. Usually, the offline analysis tool is automatically called by the WebDCS of GIF++ but, nontheless, to locally start the analysis for tests, simply type:
	
	\begin{minted}[bgcolor=bg]{bash}
bin/offlineanalysis /path/to/Scan00XXXX_HVY
	\end{minted}

	and the offline tool will by itself take care of finding the data ROOT files, as listed bellow:

	\begin{itemize}
		\item[•] Scan00XXXX\_HVY\_DAQ.root containing the TDC data (events, hit and time lists)
		\item[•] Scan00XXXX\_HVY\_CAEN.root containing the CAEN mainframe data (HVs and currents of every HV channels)
	\end{itemize}
	
	The analysed output ROOT datafiles are saved into the data folder and called \verb+Scan00XXXX_HVY_Offline.root+. Inside those, a list of \cppinline{TH1} histograms can be found. Its size will vary as a function of the number of detectors in the setup as each set of histograms is produced detector by detector. For each partition of each chamber, you will find:

	\begin{itemize}
		\item[•] \cppinline{Time_Profile_Tt_Sc_p} shows the time profile of all recorded events,
		\item[•] \cppinline{Hit_Profile_Tt_Sc_p} shows the hit profile of all recorded events,
		\item[•] \cppinline{Hit_Multiplicity_Tt_Sc_p} shows the hit multiplicity of all recorded events (number of hits per event),
		\item[•] \cppinline{Strip_Mean_Noise_Tt_Sc_p} shows noise/gamma rate for each strip in a selected time range,
		\item[•] \cppinline{Strip_Activity_Tt_Sc_p} shows noise/gamma activity for each strip (normalised version of previous histogram - strip activity = strip rate / average partition rate),
		\item[•] \cppinline{Strip_Homogeneity_Tt_Sc_p} shows the homogeneity $h$ of a given partition ($h = e^{-StdDev(strip rates in partition)/(average partition rate)}$)
		\item[•] \cppinline{mask_Strip_Mean_Noise_Tt_Sc_p} shows noise/gamma rate for each masked strip in a selected time range,
		\item[•] \cppinline{mask_Strip_Activity_Tt_Sc_p} shows noise/gamma activity for each masked strip with repect to the average rate of active strips,
		\item[•] \cppinline{NoiseCSize_H_Tt_Sc_p} shows noise/gamma cluster size,
		\item[•] \cppinline{NoiseCMult_H_Tt_Sc_p} shows noise/gamma cluster multiplicity (number of reconstructed clusters per event),
		\item[•] \cppinline{Chip_Mean_Noise_Tt_Sc_p} shows the same information than \cppinline{Strip_Mean_Noise_Tt_Scp} using a different binning (1 chip corresponds to 8 strips),
		\item[•] \cppinline{Chip_Activity_Tt_Sc_p} shows the same information than \cppinline{Strip_Activity_Tt_Scp} using a different binning,
		\item[•] \cppinline{Chip_Homogeneity_Tt_Sc_p} shows the homogeneity of a given partition using chip binning,
		\item[•] \cppinline{Beam_Profile_Tt_Sc_p} shows the estimated beam profile when taking efficiency scan (constructed with the hits contained in the muon peak where the noise/gamma background has been subtracted),
		\item[•] \cppinline{L0_Efficiency_Tt_Sc_p} shows the level 0 efficiency that was estimated \textbf{without} muon tracking,
		\item[•] \cppinline{MuonCSize_H_Tt_Sc_p} shows the level 0 muon cluster size that was estimated \textbf{without} muon tracking, and
		\item[•] \cppinline{MuonCMult_H_Tt_Sc_p} shows the level 0 muon cluster multiplicity that was estimated \textbf{without} muon tracking.
	\end{itemize}

	In the histogram labels, \verb+t+ stands for the trolley number (1 or 3), \verb+c+ for the chamber slot label in trolley \verb+t+ and \verb+p+ for the partition label (A, B, C or D depending on the chamber layout).\\

	Moreover, up to 3 CSV files can be created depending on which ones of the 3 input files were in the data folder:

	\begin{itemize}
		\item[•] \verb+Offline-Rate.csv+ : contains the summary of the noise/gamma rates and clusters,
		\item[•] \verb+Offline-Current.csv+ : contains the summary of the currents and voltages applied on the RPCs,
		\item[•] \verb+Offline-L0-EffCl.csv+ : contains the summary of the level 0 efficiency and muon cluster information without tracking.
	\end{itemize}
	
	Note that these 3 CSV files are created along their \textit{headers} (file containing the names of the data columns) and are automatically merged together when the offline analysis is used via the RunDQM button of the webDCS. Thus, the resulting files are:

	\begin{itemize}
		\item[•] \verb+Rate.csv+ ,
		\item[•] \verb+Current.csv+ ,
		\item[•] \verb+L0-EffCl.csv+ .
	\end{itemize}

\clearpage{\pagestyle{empty}\cleardoublepage}

%%%%%%%%%%%%%%%%%%%%%%%%%%%%%%%%%%%%%%%%%%%%%%%%%%%%%%%%%%%%%%%%%%%%%%%%%%%%%%%%%%%%%%%%%
%% CONTENT FOR THE READER %%
%%%%%%%%%%%%%%%%%%%%%%%%%%%%

\frontmatter
\cleardoublepage
\renewcommand\evenpagerightmark{{\scshape\small List of Figures}}
\renewcommand\oddpageleftmark{{\scshape\small List of Figures}}
\addcontentsline{toc}{chapter}{List of Figures}
\listoffigures
\clearpage{\pagestyle{empty}\cleardoublepage}

\cleardoublepage
\renewcommand\evenpagerightmark{{\scshape\small List of Tables}}
\renewcommand\oddpageleftmark{{\scshape\small List of Tables}}
\addcontentsline{toc}{chapter}{List of Tables}
\listoftables
\clearpage{\pagestyle{empty}\cleardoublepage}

% List of Acronyms
%%%%%%%%%%%%%%%%%%%%%%%%%%%%%%%%%%%%%%%%%%%%%%%%%%%%%%%%%%%%%%%%%%%%%

\chapter*{List of Acronyms}
     %\@mkboth{\MakeUppercase List of Acronyms}%
      %        {\MakeUppercase List of Acronyms}%

\begin{center}
	\textbf{\Large List of Acronyms}
\end{center}

\begin{acronym_expdlist}
    \acro{AFL}{Almost Full Level}
    \acro{ALCTs}{ anode local charged track boards}
	\acro{BARC}{Bhabha Atomic Research Centre}
	\acro{BCS}{Bardeen–Cooper–Schrieffer}
	\acro{BLT}{Block Transfer}
	\acro{BMTF}{Barrel Muon Track Finder}
	\acro{BNL}{Brookhaven National Laboratory}
	\acro{BSM}{Physics beyond the Standard Model}
	\acro{BR}{Branching Ratio}
	\acro{CAEN}{Costruzioni Apparecchiature Elettroniche Nucleari S.p.A.}
	\acro{CERN}{European Organization for Nuclear Research}
	\acro{CFD}{Constant Fraction Discriminator}
	\acro{CFEBs}{cathode front-end boards}
	\acro{CKM}{Cabibbo–Kobayashi–Maskawa}
	\acro{CMB}{Cosmic Microwave Background}
	\acro{CMS}{Compact Muon Solenoid}
	\acro{CSC}{Cathode Strip Chamber}
	\acro{CuOF}{copper-to-optical-fiber translators}
	\acro{DAQ}{Data Acquisition}
	\acro{DCS}{Detector Control Software}
	\acro{DQM}{Data Quality Monitoring}
	\acro{DT}{Drift Tube}
	\acro{ECAL}{electromagnetic calorimeter}
	\acro{EMTF}{Endcap Muon Track Finder}
	\acro{FCC}{Future Circular Collider}
	\acro{FEB}{Front-End Board}
	\acro{FEE}{Front-End Electronics}
	\acro{FWHM}{full-width-at-half-maximum}
	\acro{GEANT}{GEometry ANd Tracking - a series of software toolkit platforms developed by CERN}
	\acro{GEB}{GEM Electronics board}
	\acro{GEM}{Gas Electron Multiplier}
	\acro{GIF}{Gamma Irradiation Facility}
	\acro{GIF++}{new Gamma Irradiation Facility}
	\acro{GWP}{Global Warming Potential}
	\acro{HCAL}{hadron calorimeter}
	\acro{HEP}{High-Energy Physics}
	\acro{HL-LHC}{High Luminosity LHC}
	\acro{HPL}{High-pressure laminate}
	\acro{HSCPs}{Heavy Stable Charged Particles}
	\acro{HV}{High Voltage}
	\acro{ICRU}{International Commission on Radiation Units \& Measurements}
	\acro{iRPC}{improved RPC}
	\acro{IRQ}{Interrupt Request}
	\acro{ISR}{Intersecting Storage Rings}
	\acro{LEIR}{Low Energy Ion Ring}
	\acro{LEP}{Large Electron-Positron}
	\acro{LHC}{Large Hadron Collider}
	\acro{LS1}{First Long Shutdown}
	\acro{LS2}{Second Long Shutdown}
	\acro{LS3}{Third Long Shutdown}
	\acro{LSP}{lightest supersymmetric particle}
	\acro{LV}{Low Voltage}
	\acro{LVDS}{Low-Voltage Differential Signaling}
	\acro{MC}{Monte Carlo}
	\acro{MCNP}{Monte Carlo N-Particle}
	\acro{MiC1}{first version of Minicrate electronics}
	\acro{mip's}{minimum ionizing particles}
	\acro{MRPC}{Multigap RPC}
	\acro{MSSM}{Minimal Supersymmetric Standard Model}
	\acro{mSUGRA}{minimal SUper GRAvity}
	\acro{NIM}{Nuclear Instrumentation Module logic signals}
	\acro{OH}{Optohybrid Board}
	\acro{OMTF}{Overlap Muon Track Finder}
	\acro{PAI}{Photo-Absorption Ionisation}
	\acro{PAIR}{Photo-Absorption Ionisation with Relaxation}
	\acro{PMT}{PhotoMultiplier Tube}
	\acro{PS}{Proton Synchrotron}
	\acro{PU}{pile-up}
	\acro{QCD}{Quantum Chromodynamics}
	\acro{QED}{Quantum Electrodynamics}
	\acro{RADMON}{Radiation Monitoring}
	\acro{RMS}{Root Mean Square}
	\acro{ROOT}{a framework for data processing born at CERN}
	\acro{RPC}{Resistive Plate Chamber}
	\acro{SC}{Synchrocyclotron}
	\acro{SiPM}{Silicon Photomultiplier}
	\acro{SLAC}{Stanford Linear Accelerator Center}
	\acro{SM}{Standard Model}
	\acro{SPS}{Super Proton Synchrotron}
	\acro{SUSY}{supersymmetry}
	\acro{TDC}{Time-to-Digital Converter}
	\acro{TDR}{Technical Design Report}
	\acro{ToF}{Time-of-flight}
	\acro{TPG}{trigger primitives}
	\acro{webDCS}{Web Detector Control System}
	\acro{WIMPs}{Weakly Interacting Massive Particles}
	\acro{YETS}{Year End Technical Stop}
\end{acronym_expdlist}

% End of list of Acronyms
%%%%%%%%%%%%%%%%%%%%%%%%%%%%%%%%%%%%%%%%%%%%%%%%%%%%%%%%%%%%%%%%%%%%%%



%%%%%%%%%%%%%%%%%%%%%%%%%%%%%%%%%%%%%%%%%%%%%%%%%%%%%%%%%%%%%%%%%%%%%%%%%%%%%%%%%%%%%%%%%
%% BIBLIOGRAPHY           %%
%%%%%%%%%%%%%%%%%%%%%%%%%%%%

\pagenumbering{gobble}
\cleardoublepage
\renewcommand\evenpagerightmark{{\scshape\small Bibliography}}
\renewcommand\oddpageleftmark{{\scshape\small Bibliography}}
\addcontentsline{toc}{chapter}{References}
\printbibliography

\end{document}